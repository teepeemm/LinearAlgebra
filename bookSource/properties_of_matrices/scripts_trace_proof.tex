
\subsection*{Proof Explanation}

%%%Insert this to get the typewriter font so it looks like a real movie script
{\ttfamily
\fontdimen2\font=0.4em
\fontdimen3\font=0.2em
\fontdimen4\font=0.1em
\fontdimen7\font=0.1em
\hyphenchar\font=`\-


\hypertarget{scripts_properties_of_matrices_trace_proof}{In this} 
video we will talk through the steps required to prove
$$
\tr MN=\tr NM\, .
$$
There are some useful things to remember, first we can write
$$
M=(m^i_j)\, \qquad \mbox{and}\qquad N=(n^i_j)
$$
where the upper index labels rows and the lower one columns.
Then 
$$
MN=\big(\sum_l m^i_l n^l_j\big)\, ,
$$
where the ``open'' indices $i$ and $j$ label rows and columns, but the index $l$ is a ``dummy'' index because it is summed over. (We could have given it any name we liked!).

Finally the trace is the sum over diagonal entries for which the 
row and column numbers must coincide
$$
\tr M = \sum_i m^i_i\, .
$$
Hence starting from the left of the statement we want to prove, we have
$$
\mbox{LHS}= \tr MN = \sum_i \sum_l m^i_l n^l_i\, .
$$
Next we do something obvious, just change the order of the entries $m^i_l$ and $n^l_i$ (they are just numbers) so
$$
 \sum_i \sum_l m^i_l n^l_i= \sum_i \sum_l n^l_i m^i_l\, . 
$$
Equally obvious, we now rename $i\to l$ and $l\to i$ so
$$
 \sum_i \sum_l m^i_l n^l_i = \sum_l \sum_i n^i_l m^l_i\, . 
$$
Finally, since we have finite sums it is legal to change the order of summations
$$
 \sum_l \sum_i n^i_l m^l_i= \sum_i \sum_l n^i_l m^l_i\, . 
$$
This expression is the same as the one on the line above where we 
started except the $m$ and $n$ have been swapped so
$$
 \sum_i \sum_l m^i_l n^l_i= \tr NM =\mbox{RHS}\, . 
$$
This completes the proof. $\square$
} % Closing braket for font

\newpage
