


\begin{enumerate}

\item 
Compute the following matrix products
\[
\begin{pmatrix}1&2&1\\[1mm]4&5&2\\[1mm]7&8&2\end{pmatrix}
\begin{pmatrix}-2&\frac43&-\frac{1}{3}\\[1mm]2&-\frac53&\frac23\\[1mm]-1&2&-1\end{pmatrix}\, ,
\qquad
\begin{pmatrix}1&2&3&4&5\end{pmatrix}
\begin{pmatrix}1\\[1mm]2\\[1mm]3\\[1mm]4\\[1mm]5\end{pmatrix}\, ,
\]
\[
\begin{pmatrix}1\\[1mm]2\\[1mm]3\\[1mm]4\\[1mm]5\end{pmatrix}\begin{pmatrix}1&2&3&4&5\end{pmatrix}\, ,\qquad
\begin{pmatrix}1&2&1\\[1mm]4&5&2\\[1mm]7&8&2\end{pmatrix}
\begin{pmatrix}-2&\frac43&-\frac{1}{3}\\[1mm]2&-\frac53&\frac23\\[1mm]-1&2&-1\end{pmatrix}
\begin{pmatrix}1&2&1\\[1mm]4&5&2\\[1mm]7&8&2\end{pmatrix}\, ,
\]
\[
\begin{pmatrix}x & y &z\end{pmatrix}
\begin{pmatrix}
2& 1& 1 \\ 1 & 2 & 1 \\ 1 & 1 & 2
\end{pmatrix}
\begin{pmatrix}x\\[1mm]y\\[1mm]z\end{pmatrix}\, ,\qquad
\begin{pmatrix}2&1&2&1&2\\[1mm]0&2&1&2&1\\[1mm]0&1&2&1&2\\[1mm]0&2&1&2&1\\[1mm]0&0&0&0&2\end{pmatrix}
\begin{pmatrix}1&2&1&2&1\\[1mm]0&1&2&1&2\\[1mm]0&2&1&2&1\\[1mm]0&1&2&1&2\\[1mm]0&0&0&0&1\end{pmatrix}\, ,
\]
\vspace{.2cm}
\[
\begin{pmatrix}-2&\frac43&-\frac{1}{3}\\[1mm]2&-\frac53&\frac23\\[1mm]-1&2&-1\end{pmatrix}
\begin{pmatrix}4&\frac23&-\frac23\\[1mm]6&\frac53&-\frac23\\[1mm]12&-\frac{16}3&\frac{10}3\end{pmatrix}
\begin{pmatrix}1&2&1\\[1mm]4&5&2\\[1mm]7&8&2\end{pmatrix}\, .
\]

\phantomnewpage

\item \label{MN=NTMT} Let's prove the theorem $(MN)^T = N^TM^T$.

Note: the following is a common technique for proving matrix identities.
\begin{enumerate}
\item Let \(M=(m^i_j)\) and let \(N=(n^i_j)\). Write out a few of the entries of each matrix in the form given at the \hyperref[matrixnotation]{beginning of section}~\ref{properties_matrices}.

\item Multiply out \(MN\) and write out a few of its entries in the same form as in part (a). In terms of the entries of \(M\) and the entries of \(N\), what is the entry in row \(i\) and column \(j\) of \(MN\)?

\item Take the transpose \((MN)^T\) and write out a few of its entries in the same form as in part (a). In terms of the entries of \(M\) and the entries of \(N\), what is the entry in row \(i\) and column \(j\) of \((MN)^T\)?

\item Take the transposes \(N^T\) and \(M^T\) and write out a few of their entries in the same form as in part (a).

\item Multiply out \(N^TM^T\) and write out a few of its entries in the same form as in part a. In terms of the entries of \(M\) and the entries of \(N\), what is the entry in row \(i\) and column \(j\) of \(N^TM^T\)?

\item Show that the answers you got in parts (c) and (e) are the same.
\end{enumerate}

\phantomnewpage

\item  
\begin{enumerate}
\item Let $A=\begin{pmatrix}
1 & 2 & 0\\
3 & -1 & 4\\
\end{pmatrix}$.  Find $AA^T$ and $A^TA$ and their traces.  \\
%What can you say about matrices $MM^T$ and $M^TM$ in general?  Explain.
\item Let $M$ be any $m\times n$ matrix.  Show that $M^TM$ and $MM^T$ are symmetric. (Hint: use the result of the previous problem.)  What are their sizes? What is the relationship between their traces? 
\end{enumerate}

\phantomnewpage

\item \label{mat_prob1} %\hyperlink{dotprod}{
Let $x = \ccolvec{x_1 \\ \vdots \\ x_n}$ and $y = \ccolvec{y_1 \\ \vdots \\ y_n}$ be column vectors.  Show that the dot product $x \dotprod y = x^T\  I\  y$.

\Videoscriptlink{matrices_hint1.mp4}{Hint}{scripts_matrices_hint1}

\phantomnewpage

\item \label{mat_prob} \hyperlink{leftmult}{Above}, we showed that \emph{left} multiplication by an $r \times s$ matrix $N$  was a linear transformation $M^s_k \stackrel{N}{\longrightarrow} M^r_k$.  Show that \emph{right} multiplication by a $k \times m$ matrix $R$ is a linear transformation $M^s_k \stackrel{R}{\longrightarrow} M^s_m$.  In other words, show that right matrix multiplication obeys linearity.

\Videoscriptlink{matrices_hint.mp4}{Hint}{scripts_matrices_hint}

\phantomnewpage

\item Let the $V$ be a vector space where $B=(v_1,v_2)$ is an ordered basis. Suppose
\[
L:V\stackrel{\rm linear}{-\!\!-\!\!\!\longrightarrow} V
\]
and \[L(v_1)=v_1+v_2 \, ,\quad L(v_2)=2v_1+v_2\, .\] Compute the matrix of $L$ in the basis $B$ and then compute the trace of this matrix.
Suppose that $ad-bc\neq 0$ and consider now the new basis
\[
B'=(av_1+b v_2,cv_1+dv_2)\, .
\]
Compute the matrix of $L$ in the basis $B'$. Compute the trace of this matrix. What do you find? What do you conclude about the trace of a matrix? Does it make sense to talk about the ``trace of a linear transformation'' without reference to any bases? 

\item Explain what happens to a matrix when:
\begin{enumerate}
\item You multiply it on the left by a diagonal matrix.
\item You multiply it on the right by a diagonal matrix.
\end{enumerate}
Give a few simple examples before you start explaining.


%%%%%%%%%%%%%%%%%%%%%%%%%%%%%%%%%


\phantomnewpage

\item  \label{expprob}Compute $\exp (A)$ for the following matrices:
\begin{itemize}
\item $A = \begin{pmatrix}
\lambda & 0 \\
0 & \lambda \\
\end{pmatrix}$
\item $A = \begin{pmatrix}
1 & \lambda \\
0 & 1 \\
\end{pmatrix}$
\item $A = \begin{pmatrix}
0 & \lambda \\
0 & 0 \\
\end{pmatrix}$
\end{itemize}

\Videoscriptlink{properties_of_matrices_exponents.mp4}{Hint}{properties_of_matrices_exponents}

%\phantomnewpage

%\item Suppose $ad-bc\neq 0$, and let $M=\begin{pmatrix}
%a & b \\
%c & d \\
%\end{pmatrix}$.  
%\begin{enumerate}
%\item Find a matrix $M^{-1}$ such that $MM^{-1}=I$.  
%\item Explain why your result explains what you found in a \hyperref[inverserowops]{previous homework exercise}.
%\item Compute $M^{-1}M$.
%\end{enumerate}

\phantomnewpage

\item Let $M = \begin{pmatrix}
1 & 0 & 0 & 0 & 0 & 0 & 0 & 1 \\
0 & 1 & 0 & 0 & 0 & 0 & 1 & 0 \\
0 & 0 & 1 & 0 & 0 & 1 & 0 & 0 \\
0 & 0 & 0 & 1 & 1 & 0 & 0 & 0 \\
0 & 0 & 0 & 0 & 2 & 1 & 0 & 0 \\
0 & 0 & 0 & 0 & 0 & 2 & 0 & 0 \\
0 & 0 & 0 & 0 & 0 & 0 & 3 & 1 \\
0 & 0 & 0 & 0 & 0 & 0 & 0 & 3 \\
\end{pmatrix}$.  Divide $M$ into named blocks, with one block the $4\times4$ identity matrix, and then multiply blocks to compute $M^2$.

\phantomnewpage

\item A matrix $A$ is called \emph{anti-symmetric}\index{Anti-symmetric matrix} (or skew-symmetric\index{Skew-symmetric matrix|see {Anti-symmetric matrix}}) if $A^T = -A$. Show that for every $n \times n$ matrix $M$, we can write $M = A + S$ where $A$ is an anti-symmetric matrix and $S$ is a symmetric matrix.

\emph{Hint: What kind of matrix is $M + M^T$? How about $M - M^T$?}

\item An example of an operation which is not associative is the cross product. 
\begin{enumerate}
\item Give a simple example of three vectors from 3-space $u,v,w$ such that $u\times (v\times w) \neq (u\times v)\times w$. 
\item We saw in \hyperlink{crossmat}{Chapter~\ref{warmup}} that the operator $B=u\times$ (cross product with a vector) is a linear operator. It can therefore be written as a matrix (given an ordered basis such as the standard basis). How is it that composing such linear operators is non-associative even though matrix multiplication is associative? 

\end{enumerate}

\end{enumerate}

\phantomnewpage