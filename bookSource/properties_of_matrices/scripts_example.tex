
\subsection*{Matrix Exponential Example}

{\ttfamily
\fontdimen2\font=0.4em
\fontdimen3\font=0.2em
\fontdimen4\font=0.1em
\fontdimen7\font=0.1em
\hyphenchar\font=`\-

\hypertarget{properties_of_matrices_example}{This video} shows you
how to compute
\[
\exp\begin{pmatrix}0&\theta\\-\theta & 0\end{pmatrix}
\, .
\]
For this we need to remember that the matrix exponential is defined by its power series
\[
\exp M := I + M + \frac1{2!} M^2 + \frac1{3!} M^3+\cdots\, .
\]
Now lets call 
\[\begin{pmatrix}0&\theta\\-\theta & 0\end{pmatrix}=i\theta\]
where the {\it matrix} 
\[
i:=\begin{pmatrix}0&1\\-1 & 0\end{pmatrix}
\]
and by matrix multiplication is seen to obey
\[
i^2 =-I\, ,\qquad i^3=-i\, , i^4 = I\, .
\]
Using these facts we compute by organizing terms according to whether they have an $i$ or not:
\begin{eqnarray*}
\exp i\theta &=& I + \frac 1{2!}\, \theta^2 (-I) + \frac 1{4!} \,(+I) + \cdots \\
&+&i \theta + \frac1{3!} \,\theta^3 (-i) + \frac 1{5!} \,i+\cdots\\[2mm]
&=& I ( 1-\frac 1{2!}\, \theta^2  + \frac 1{4!}\,\theta^4  + \cdots)\\
&+&i( \theta- \frac1{3!}\, \theta^3  + \frac 1{5!}\,\theta^5 +\cdots)\\[2mm]
&=&I\cos\theta + i \sin\theta \\[2mm]
&=&\begin{pmatrix}\cos\theta & \sin\theta \\ -\sin\theta & \cos\theta\end{pmatrix}\, .
\end{eqnarray*}
Here we used the familiar Taylor series for the cosine and sine functions. A fun thing to think about is how the above matrix
acts on vector in the plane.


} % Closing brace for the font

%\newpage
