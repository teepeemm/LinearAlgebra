
\subsection{More on the Trace Function}

%%%Insert this to get the typewriter font so it looks like a real movie script
{\ttfamily
\fontdimen2\font=0.4em
\fontdimen3\font=0.2em
\fontdimen4\font=0.1em
\fontdimen7\font=0.1em
\hyphenchar\font=`\-


\hypertarget{scripts_properties_of_matrices_trace}{This seemingly boring} function which extracts a single real number does not seem immediately useful, however it is an example of an element in the \href{http://en.wikipedia.org/wiki/Dual_space}{\emph{dual-space}} of all $n \times n$ matrices since it is a \href{http://en.wikipedia.org/wiki/Bounded_operator}{bounded linear operator} to the underlying field $\mathbb{F}$. By a bounded operator, I mean it will at most scale the length of the matrix (think of it as a vector in $\mathbb{F}^{n^2}$) by some fixed constant $C > 0$ (this can depend upon $n$), and for example if the length of a matrix $M$ is $d$, then $\tr(M) \leq C d$ (I believe $C = 1$ should work).

Some other useful properties is for block matrices, it should be clear that we have
\[
\tr \begin{pmatrix} A & B \\ C & D \end{pmatrix} = \tr A + \tr D.
\]
and that
\[
\tr(P A P^{-1}) = \tr\bigl(P(AP^{-1})\bigr) = \tr\bigl((AP^{-1})P \bigr) = \tr(A P^{-1}P) = \tr(A)
\]
so the trace function is conjugate (i.e. similarity) invariant. Using a concept from Chapter~\ref{basisdimension}, it is basis invariant. Additionally in later chapters we will see that the trace function can be used to calculate the \hyperref[elementarydeterminants]{determinant} (in a sense it is the derivative of the determinant, see \hyperref[prob_derivative_determinant]{Lecture~\ref*{elementarydeterminantsII} Problem~\ref*{prob_derivative_determinant}}) and \hyperref[eigenvalseigenvects]{eigenvalues}.

Additionally we can define the set $\mathfrak{sl}_n$ as the set of all $n \times n$ matrices with trace equal to 0, and since the trace is linear and $a \cdot 0 = 0$, we note that $\mathfrak{sl}_n$ is a vector space. Additionally we can use the fact $\tr(MN) = \tr(NM)$ to define an operation called bracket
\[
[M,N] = MN - NM,
\]
and we note that $\mathfrak{sl}_n$ is closed under bracket since
\[
\tr(MN - NM) = \tr(MN) - \tr(NM) = \tr(MN) - \tr(MN) = 0.
\]

} % Closing braket for font

\newpage
