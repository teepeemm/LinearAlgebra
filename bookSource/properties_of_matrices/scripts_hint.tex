
\subsection*{Hint for Review Question~\ref{expprob}}

{\ttfamily
\fontdimen2\font=0.4em
\fontdimen3\font=0.2em
\fontdimen4\font=0.1em
\fontdimen7\font=0.1em
\hyphenchar\font=`\-

\hypertarget{properties_of_matrices_exponents}{This is a hint for} computing exponents of matrices. So what is $e^A$ if $A$ is a matrix? We remember that the Taylor series for
\[
e^x = \sum^{\infty}_{n=0} \frac{x^n}{n!}.
\]
So as matrices we can think about
\[
e^A = \sum^{\infty}_{n=0} \frac{A^n}{n!}.
\]
This means we are going to have an idea of what $A^n$ looks like for any $n$. Lets look at the example of one of the matrices in the problem. Let
\[
A = 
\left( 
\begin{array}{cc}
1 & \lambda\\
0 & 1\\
\end{array}
\right).
\]

Lets compute $A^n$ for the first few $n$. 
\begin{align*}
A^0 & =
\left(
\begin{array}{cc}
1 & 0\\
0 & 1\\
\end{array}
\right)
\\ A^1 & = 
\left(
\begin{array}{cc}
1 & \lambda \\
0 & 1\\
\end{array}
\right)
\\ A^2 & = A\cdot A = 
\left(
\begin{array}{cc}
1 & 2 \lambda \\
0 & 1\\
\end{array}
\right)
\\ A^3 & = A^2\cdot A = 
\left(
\begin{array}{cc}
1 & 3 \lambda \\
0 & 1\\
\end{array}
\right).
\end{align*}
There is a pattern here which is that 
\[ A^n = 
\left(
\begin{array}{cc}
1 & n \lambda \\
0 & 1\\
\end{array}
\right),
 \]
then we can think about the first few terms of the sequence
\[
e^A = \sum^{\infty}_{n=0} \frac{A^n}{n!} = A^0 + A + \frac{1}{2!�} A^2 + \frac{1}{3!} A^3 + \ldots.
\]
Looking at the entries when we add this we get that the upper left-most entry looks like this:
\[
1 + 1 + \frac{1}{2} + \frac{1}{3!} + \ldots = \sum^{\infty}_{n=0} \frac{1}{n!} = e^1.
\]
Continue this process with each of the entries using what you know about Taylor series expansions to find the sum of each entry.
 
%  \[e^A = 
% \left(
%\begin{array}{cc}
%1 + 1 + \frac{1}{2} + \frac{1}{3!} + \ldots  & 0 + \lambda +   \frac{1}{2!} 2 \lambda + \frac{1}{3!} 3  \lambda + \ldots \\
%0 & 1 + 1 + \frac{1}{2!} + \frac{1}{3!}\\
%\end{array}
%\right)
% \]
% 
 
} % Closing brace for the font

%\newpage
