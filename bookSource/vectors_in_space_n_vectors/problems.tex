

\begin{enumerate}

\item When he was young, Captain Conundrum\index{Captain Conundrum} mowed lawns on weekends to help pay his college tuition bills. He charged his customers according to the size of their lawns at a rate of 5\textcent \ per square foot and meticulously kept a record of the areas of their lawns in an ordered list:
\[
A=(200,300,50,50,100,100,200,500,1000,100)\, .
\]
He also listed the number of times he mowed each lawn in a given year, for the year 1988 that ordered list was
\[
f=(20,1,2,4,1,5,2,1,10,6)\, .
\]
\begin{enumerate}
\item
Pretend that $A$ and $f$ are vectors and compute $A\dotprod f$.
\item
What quantity does the dot product $A\dotprod f$ measure?
\item\label{cost}
How much did Captain Conundrum earn from mowing lawns in 1988? Write an expression
for this amount in terms of the vectors $A$ and $f$.
\item Suppose Captain Conundrum charged different customers different rates. How could
you modify the expression in part~\ref{cost} to compute the Captain's earnings?
\end{enumerate}


\item \begin{itemize}
	\item[(2)] Find the angle between the diagonal of the unit square in $\mathbb{R}^2$ and one of the coordinate axes.

	\item[(3)] Find the angle between the diagonal of the unit cube in $\mathbb{R}^3$ and one of the coordinate axes.

	\item[(n)] Find the angle between the diagonal of the unit (hyper)-cube in $\mathbb{R}^n$ and one of the coordinate axes.

	\item[($\infty$)] What is the limit as $n \to \infty$ of the angle between the diagonal of the unit (hyper)-cube in $\mathbb{R}^n$ and one of the coordinate axes?
\end{itemize}

\phantomnewpage

\item\label{rotate}\hypertarget{rotationprob}{Consider} the matrix 
$M = \begin{pmatrix}
\cos \theta & \sin \theta \\
-\sin \theta & \cos \theta \\
\end{pmatrix}
$ and the vector $X = \colvec{x\\y}$.
\begin{enumerate}
	\item Sketch $X$ and $MX$ in $\mathbb{R}^2$ for several values of $X$ and $\theta$.
	\item Compute $\frac{||MX||}{||X||}$ for arbitrary values of $X$ and $\theta$.
	\item Explain your result for (b) and describe the action of $M$ geometrically.
\end{enumerate}

%I can't like this problem -Cherney
%\item Suppose in $\mathbb{R}^2$ I measure the $x$ direction in inches and the $y$ direction in miles.  Approximately what is the real-world angle between the vectors $\colvec{0\\1}$ and $\colvec{1\\1}$?  What is the angle between these two vectors according to the dot-product?  Give a definition for an inner product so that the angles produced by the inner product are the actual angles between vectors.

\phantomnewpage

\item \label{lorentz}(Lorentzian Strangeness).  For this problem, consider $\mathbb{R}^n$ with the Lorentzian inner product  defined in example~\ref{lorentzex} \hyperlink{lorentzian_metric}{above}. 
\begin{enumerate}
	\item Find a non-zero vector in two-dimensional Lorentzian space-time with zero length.

	\item Find and sketch the collection of all vectors in two-dimensional Lorentzian space-time with zero length.

	\item Find and sketch the collection of all vectors in three-dimensional Lorentzian space-time with zero length. 
	\item Replace the word ``zero" with the word ``one" in the previous two parts. 
\end{enumerate}
\Videoscriptlink{vectors_in_space_n_vectors_hint.mp4}{The Story of Your Life}{vectors_in_space_n_vectors_hint}

\item Create a system of equations whose solution set is a 99 dimensional hyperplane in 
$\mathbb{R}^{101}$. 

\item Recall that a plane in $\mathbb{R}^3$ can be described by the equation 
\[n \cdot \colvec{x\\ y\\ z}=n\cdot p\] 
where the vector $p$ labels a given point on the plane and $n$ is a vector normal to the plane. 
Let  $N$ and $P$ be vectors in $\mathbb{R}^{101}$ and 
\[X=\ccolvec{x^1\\x^2\\ \vdots\\ x^{101}}.\]
What kind of geometric object does $N\cdot X= N\cdot P$ describe?

\item 
Let
\[
u=\ccolvec{1\\1\\1\\ \vdots \\ 1} \text{~and~} v= \ccolvec{1\\2\\3\\ \vdots\\ \! 101\!} 
\]
Find the projection of $v$ onto $u$ and the projection of $u$ onto $v$.
({\itshape Hint:} Remember that  two vectors $u$ and $v$ define a plane, so first work out how to project one vector onto another
in a plane. The \hyperlink{projectionpic}{picture} from Section~\ref{gramschmidt} could help.)

\item If the solution set to the equation $A(x)=b$ is the set of vectors whose tips lie on the paraboloid $z=x^2+y^2$, then what can you say about the function $A$? 


\item Find a system of equations whose solution set is 
\[
\left\{ \colvec{1\\1\\2\\0} +c_1 \colvec{-1\\-1\\0\\1} +c_2 \colvec{0\\0\\-1\\-3}  \middle| \,c_1,c_2\in \mathbb{R} 
\right\}.
\]
Give a general procedure for going from a parametric description of a hyperplane to a system of equations with  that hyperplane as a solution set.


\item If $A$ is a linear operator and both  $v$ and $cv$ (for any real number $c$) are  solutions to $Ax=b$, then what can you say about $b$? 


\end{enumerate}

\phantomnewpage