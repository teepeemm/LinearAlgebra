\subsection*{A $4\times4$ Gram Schmidt Example}

%%%Insert this to get the typewriter font so it looks like a real movie script
{\ttfamily
\fontdimen2\font=0.4em
\fontdimen3\font=0.2em
\fontdimen4\font=0.1em
\fontdimen7\font=0.1em
\hyphenchar\font=`\-


%%%%put a hypertarget around the opening bit of text
\hypertarget{scripts_gram_schmidt_and_orthogonal_complements_4by4_example}{Lets do an example} of how to "Gram-Schmidt" some vectors in $\mathbb{R}^4$. Given the following vectors
\[
v_1 = \colvec{o\\1 \\ 0\\ 0}, \, v_2 = \colvec{0\\1\\ 1\\ 0}, \, v_3 = \colvec{3\\0 \\ 1\\ 0}, \, \text{ and } v_4 = \colvec{1\\ 1\\ 0\\2},
\]
we start with $v_1$
\[
v_1^\bot = v_1 =\colvec{0\\1 \\ 0\\ 0}.
\]
Now the work begins
\begin{eqnarray*}
v_2^\bot &=& v_2 -\frac{(v_1^\bot \cdot v_2) }{\norm{v_1^\bot}^2} v_1^\bot \\
&=&\colvec{0\\1\\ 1\\ 0} - \frac{1}{1}  \colvec{0\\1 \\ 0\\ 0}\\
&=&  \colvec{0\\0\\ 1\\ 0}  \\
\end{eqnarray*}


This gets a little longer with every step.
\begin{eqnarray*}
v_3^\bot &=& v_3  -\frac{(v_1^\bot \cdot v_3 )}{\norm{v_1^\bot}^2} v_1^\bot  -\frac{(v_2^\bot \cdot v_3 )}{\norm{v_2^\bot}^2} v_2^\bot\\
&=&\colvec{3\\0 \\ 1\\ 0}- \frac{0}{1} \colvec{0\\1 \\ 0\\ 0} - \frac{1}{1}  \colvec{0\\0\\ 1\\ 0}  = \colvec{3\\0 \\ 0\\ 0}\\
\end{eqnarray*} 
This last step requires subtracting off the term of the form $\frac{u\cdot v}{u \cdot u} \mathbf{u}$ for each of the previously defined basis vectors.

\begin{eqnarray*}
v_4^\bot &=& v_4  - \frac{(v_1^\bot \cdot v_4 )}{\norm{v_1^\bot}^2} v_1^\bot  -\frac{(v_2^\bot \cdot v_4 )}{\norm{v_2^\bot}^2} v_2^\bot  -\frac{(v_3^\bot \cdot v_4 )}{\norm{v_3^\bot}^2} v_3^\bot \\
&=& \colvec{1\\ 1\\ 0\\2}  -\frac{1}{1}  \colvec{0\\1 \\ 0\\ 0}  - \frac{0}{1} \colvec{0\\0\\ 1\\ 0}  -\frac{3}{9}  \colvec{3\\0 \\ 0\\ 0} \\
&=& \colvec{0\\ 0\\ 0\\2}
\end{eqnarray*}
Now $v_1^\bot$,  $v_2^\bot$, $v_3^\bot$, and $v_4^\bot$ are an orthogonal basis. Notice that even with very, very nice looking vectors we end up having to do quite a bit of arithmetic. This a good reason to use programs like matlab to check your work.
%%%%don't forget to close the bracket so the stuff after your file doesn't look like a movie!
}

%\newpage