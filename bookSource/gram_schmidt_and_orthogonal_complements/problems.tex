


\begin{enumerate}

\item Let $D=\begin{pmatrix}
\lambda_1 & \mc0 \\
\mc0 & \lambda_2 \\
\end{pmatrix}$.
\begin{enumerate}
\item Write $D$ in terms of the vectors $e_1$ and $e_2$, and their transposes.
\item Suppose $P=\begin{pmatrix}
a & b \\
c & d \\
\end{pmatrix}$ is invertible.  Show that $D$ is similar to
\[
M=\frac{1}{ad-bc}\begin{pmatrix}
\lambda_1ad-\lambda_2bc & -(\lambda_1-\lambda_2)ab \\[1mm]
(\lambda_1-\lambda_2)cd & -\lambda_1bc + \lambda_2ad
\end{pmatrix}.
\]
\item Suppose the vectors $\rowvec{a,b}$ and $\rowvec{c,d}$ are orthogonal.  What can you say about $M$ in this case? (Hint: think about what \(M^T\) is equal to.)
\end{enumerate}

\phantomnewpage

\item \label{orthogprob} Suppose $S=\{v_1, \ldots, v_n \}$ is an \emph{orthogonal} (not orthonormal) basis for~$\Re^n$.  Then we can write any vector $v$ as $v=\sum_ic^iv_i$ for some constants $c^i$.  Find a formula for the constants $c^i$ in terms of $v$ and the vectors in~$S$.

\Videoscriptlink{orthonormal_bases_hint.mp4}{Hint}{scripts_orthonormal_bases_hint}
\phantomnewpage

\item \label{orthogprojprob} Let $u,v$ be linearly independent vectors in $\Re^3$, and $P=\spa \{ u,v\}$ be the plane spanned by $u$ and $v$.  
\begin{enumerate}
\item Is the vector $v^\bot := v-\frac{u\cdot v}{u\cdot u}u$ in the plane $P$?
\item  What is the (cosine of the) angle between $v^\bot$ and $u$?
\item %Given your solution to the above, 
How can you find a third vector perpendicular to both $u$ and $v^\bot$?
\item  Construct an orthonormal basis for $\Re^3$ from $u$ and $v$.
\item  Test your abstract formul\ae\ starting with 
\[
u=\rowvec{1 , 2 , 0} \text{ and } v=\rowvec{0 , 1 , 1}.
\]
\end{enumerate}

\Videoscriptlink{orthonormal_bases_hint3.mp4}{Hint}{scripts_orthonormal_bases_hint3}

\phantomnewpage



\item Find an orthonormal  basis for $\Re^4$ which includes $(1,1,1,1)$ using the following procedure:\\
\begin{enumerate} 
\item Pick a vector perpendicular to the vector 
\[v_1 =\colvec{1\\1\\1\\1}\] from the solution set of the matrix equation \[v_1^Tx=0\, .\] Pick the vector $v_2$ obtained from the standard Gaussian elimination procedure which is the coefficient of $x_2$.
\item Pick a vector perpendicular to both $v_1$ and $v_2$ from the solutions set of the matrix equation \[\colvec{v_1^T\\[1mm]v_2^T}x=0\, .\] Pick the vector $v_3$ obtained from the standard Gaussian elimination procedure with $x_3$ as the coefficient. 
\item Pick a vector perpendicular to $v_1,v_2,$ and $v_3$ from the solution set of the matrix equation \[\colvec{v_1^T\\[1mm]v_2^T\\[1mm]v_3^T}x=0\, .\]  Pick the vector $v_4$ obtained from the standard Gaussian elimination procedure with $x_3$ as the coefficient. 
\item Normalize the four vectors obtained   above.
\end{enumerate}


\item Use the inner product \[f\cdot g := \int_0^1 f(x)g(x)dx\]  on the vector space $V=\span \{1,x,x^2,x^3\}$ to perform the Gram-Schmidt procedure on the set of vectors $\{1,x,x^2,x^3\}$. 

\item Use the inner product \[f\cdot g := \int_0^{2\pi} f(x)g(x)dx\]  on the vector space $V=\span\{\sin(x),\sin(2x),\sin(3x) \}$ to perform the Gram-Schmidt procedure on the set of vectors $\{\sin(x),\sin(2x),\sin(3x) \}$. \\
Try to build an orthonormal basis for the vector space \[\spa \{ \sin(nx)~| ~n\in \N \}\, .\]
%What do you suspect about the vector space $\spa \{ \sin(nx)~| ~n\in \N \}$?\\
%What do you suspect about the vector space $\spa \{ \sin(ax)~|~ a \in \Re \}$?
\item 
\begin{enumerate}
\item
Show that if $Q$ is an orthogonal $n\times n$ matrix, then \[u\dotprod v = (Qu)\dotprod (Qv)\, ,\] for any $u,v\in \Re^n$. That is, $Q$ preserves the inner product. 
\item Does $Q$ preserve the outer product? 
\item  If the set of vectors $\{ u_1,\dots,u_n\}$ is orthonormal and $\{ \lambda_1,\cdots,\lambda_n\}$ is a set of numbers, 
then what are the eigenvalues and eigenvectors of the matrix
$M=\sum_{i=1}^n \lambda_i u_i u_i^T$? 
\item How would the eigenvectors and eigenvalues of this matrix change if we replaced  $\{ u_1,\dots,u_n\}$ by $\{ Qu_1,\dots,Q u_n\}$?
\end{enumerate}


\item Carefully write out the Gram-Schmidt procedure for the set of vectors 
\[\left\{ \colvec{1\\1\\1}, \colvec{1\\-1\\1}, \colvec{1\\1\\-1} \right\} \, .\] Is it possible to rescale the second vector obtained in the procedure to a vector with integer components? 


\item 
\label{basisortho}
\begin{enumerate}
\item Suppose $u$ and $v$ are linearly independent.  Show that $u$ and $v^\perp$ are also linearly independent.  Explain why $\{u, v^\perp\}$ is a basis for $\spa \{u,v\}$.



\Videoscriptlink{gram_schmidt_and_orthogonal_complements_hint.mp4}{Hint}{gram_schmidt_and_orthogonal_complements_hint}

\item Repeat the previous problem, but with three independent vectors $u,v,w$
 where $v^\perp$ and $w^\perp$ are as defined by the Gram-Schmidt procedure. 
\end{enumerate}

\phantomnewpage


\item \label{QRprob} Find the $QR$ factorization of
\[
M=\begin{pmatrix}1&0&\phantom{\!-}2\\-1&2&0\\-1&-2&2
\end{pmatrix}\, .
\]

\phantomnewpage

\item Given any three vectors $u,v,w$, when do $v^\perp$ or $w^\perp$ of the Gram--Schmidt procedure vanish?

\phantomnewpage

\item For $U$ a subspace of $W$, use the subspace theorem to check that $U^\perp$ is a subspace of $W$.

\phantomnewpage


\phantomnewpage

\item %(Extra Credit) 
Let $S_n$ and $A_n$ define the space of $n \times n$ symmetric and anti-symmetric matrices, respectively. These are subspaces of the vector space $M^n_n$ of all $n\times n$ matrices. What is $\dim M^n_n$, $\dim S_n$, and $\dim A_n$? Show that $M^n_n = S_n + A_n$. Define an inner product on square matrices
\[
M\cdot N =\tr MN\, .
\]
Is $A_n^{\perp}=S_n$? Is $M^n_n = S_n \oplus A_n$?

%\emph{Hint: Note that $\dim S_n = \dim U_n$ where $U_n$ is the vector space of all $n \times n$ upper triangular matrices, and also note that $\dim A_n = \dim \widetilde{U}_n$ where $\widetilde{U}_n$ is the vector space of all strictly $n \times n$ upper triangular matrices (\emph{i.e.} the diagonal entries are all 0).}

\item The vector space $V=\span\{ \sin(t),\sin(2t), \sin(3t) , \sin(3t)\}$ has an inner product: 
\[f\cdot g:=\int _0^{2\pi}f(t)g(t) dt\, .\] Find the orthogonal compliment to $U=\span\{ \sin(t)+\sin(2t) \}$ in $V$. Express $\sin(t)-\sin(2t)$ as  the sum of vectors from $U$ and $U^\perp$.

\end{enumerate}

\phantomnewpage