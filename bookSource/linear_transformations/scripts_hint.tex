
\subsection*{Hint for Review Question~\ref{polyprob}}

{\ttfamily
\fontdimen2\font=0.4em
\fontdimen3\font=0.2em
\fontdimen4\font=0.1em
\fontdimen7\font=0.1em
\hyphenchar\font=`\-

\hypertarget{scripts_linear_transformations_hint}{The first thing we see in the problem} is a definition of this new space $P_n$. Elements of $P_n$ are polynomials that look like \[a_0 + a_1 t +a_2 t^2 + \ldots + a_n t^n\] where the $a_i$'s are constants. So this means if $L$ is a linear transformation from $P_2 \to P_3$ that the inputs of $L$ are degree two polynomials which look like \[a_0 + a_1 t +a_2 t^2\]
and the output will have degree three and look like \[b_0 + b_1 t +b_2 t^2 + b_3 t^3\]

We also know that $L$ is a linear transformation, so what does that mean in this case? Well, by linearity we know that we can separate out the sum, and pull out the constants so we get
\[ L (a_0 + a_1 t +a_2 t^2) = a_0L(1) + a_1L( t) +a_2 L( t^2)\]
Just this should be really helpful for the first two parts of the problem. The third part of the problem is asking us to think about this as a linear algebra problem, so lets think about how we could write this in the vector notation we use in the class. We could write 
\[a_0 + a_1 t +a_2 t^2 \text{ as } \colvec{a_0 \\ a_1 \\ a_2} \]
And think for a second about how you add polynomials, you match up terms of the same degree and add the constants component-wise. So it makes some sense to think about polynomials this way, since vector addition is also component-wise.

We could also write the output 
\[  b_0 + b_1 t +b_2 t^2 + b_3 t^3 \text{ as }  \colvec{b_0 \\ b_1 \\ b_2} \\ b_3\]

Then lets look at the information given in the problem and think about it in terms of column vectors 
\begin{itemize}
\item $L(1) = 4$ but we can think of the input $1= 1+ 0t + 0t^2 $ and the output  $4 = 4+ 0t + 0t^2 0t^3$ and write this as $ L (\colvec{1\\0\\0}) = \colvec{4\\0\\0\\0}$

\item$L(t) = t^3$ This can be written as 
$ L (\colvec{0\\1\\0}) = \colvec{0\\0\\0\\1}$

\item$L(t^2) = t-1$ It might be a little trickier to figure out how to write $t-1$ but if we write the polynomial out with the terms in order and with zeroes next to the terms that do not appear, we can see that 
\[ t-1 = -1 + t +0t^2 + 0t^3 \text{ corresponds to } \colvec{-1 \\ 1\\ 0 \\ 0}\]
So this can be written as 
$ L (\colvec{0\\0\\1}) = \colvec{-1\\1\\0\\0}$

Now to think about how you would write the linear transformation $L$ as a matrix, first think about what the dimensions of the matrix would be. Then look at the first two parts of this problem to help you figure out what the entries should be.


\end{itemize}


} % Closing brace for the font

%\newpage
