
\subsection{\linTransTitle: Derivative and Integral of (Real) Polynomials of Degree at Most 3}

%%%Insert this to get the typewriter font so it looks like a real movie script
{\ttfamily
\fontdimen2\font=0.4em
\fontdimen3\font=0.2em
\fontdimen4\font=0.1em
\fontdimen7\font=0.1em
\hyphenchar\font=`\-


\hypertarget{scripts_linear_transformations_deriv_int}{For this, we} consider the vector space $\mathbb{P}^{\mathbb{R}}_3$ of real coefficient polynomials $p$ such that the degree of $\deg{p}$ is at most 3. Let $D$ denote the usual derivative operator and we note that \hyperlink{derivative_linear}{it is linear}, and we can write this as the matrix
\[
D = \begin{pmatrix}
0 & 1 & 0 & 0 \\
0 & 0 & 2 & 0 \\
0 & 0 & 0 & 3 \\
0 & 0 & 0 & 0
\end{pmatrix}.
\]

Similarly now consider the map $I$ where $I(f) = \int f(x) \, dx$ is the \emph{indefinite} integral on any integrable function $f$. Now we first note that for any $\alpha, \beta \in \mathbb{R}$, we have
\begin{align*}
I(\alpha \cdot p + \beta \cdot q) & = \int \bigl( \alpha \cdot p(x) + \beta \cdot q(x) \bigr) \, dx
\\ & = \alpha \int p(x) \, dx + \beta \int q(x) \, dx = \alpha I(p) + \beta I(q),
\end{align*}
so $I$ is a linear map on functions. However we note that this is not a well-defined map on vector spaces since the additive constant states the image is not unique. For example $I(3x^2) = x^3 + c$ where $c$ can be \emph{any} constant. Therefore we have to perform a \emph{definite} integral instead, so we define $I(f) := \int_0^x f(y) \, dy$. The other thing we could do is explicitly choose our constant, and we note that this does not necessarily give the same map (ex. take the constant to be non-zero with polynomials which in-fact will make it \hyperlink{non_linear_example}{non-linear}).

Now going to our vector space $\mathbb{P}^{\mathbb{R}}_3$, if we take any $p(x) = \alpha x^3 + \beta x^2 + \gamma x + \delta$, then we have
\[
I(p) = \frac{\alpha}{4} x^4 + \frac{\beta}{3} x^3 + \frac{\gamma}{2} x^2 + \delta x,
\]
and we note that this is outside of $\mathbb{P}^{\mathbb{R}}_3$. So to make our image in $\mathbb{P}^{\mathbb{R}}_3$, we formally set $I(x^3) = 0$. Thus we can now (finally) write this as the linear map $I \colon \mathbb{P}^{\mathbb{R}}_3 \rightarrow \mathbb{P}^{\mathbb{R}}_3$ as the matrix:
\[
I = \begin{pmatrix}
0 & 0 & 0 & 0 \\
1 & 0 & 0 & 0 \\
0 & \frac{1}{2} & 0 & 0 \\
0 & 0 & \frac{1}{3} & 0
\end{pmatrix}.
\]

Finally we have
\[
I D = \begin{pmatrix}
0 & 0 & 0 & 0 \\
0 & 1 & 0 & 0 \\
0 & 0 & 1 & 0 \\
0 & 0 & 0 & 1
\end{pmatrix}
\]
and
\[
D I = \begin{pmatrix}
1 & 0 & 0 & 0 \\
0 & 1 & 0 & 0 \\
0 & 0 & 1 & 0 \\
0 & 0 & 0 & 0
\end{pmatrix},
\]
and note the subspaces that are preserved under these compositions.

} % Closing braket for font

\newpage
