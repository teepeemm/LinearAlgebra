


\begin{enumerate}

\item While performing  Gaussian elimination on these augmented matrices write the full system of equations describing the new rows in terms of the old rows above each equivalence symbol as in  \hyperlink{Keeping track of EROs with equations between rows}{Example}~\ref{Rsystem}. 
$$
\begin{amatrix}{2} 
2 & 2 & 10 \\
1 & 2 & 8 \\
\end{amatrix}
,~
\begin{amatrix}{3} 
1 & 1 & 0 & 5 \\
1 & 1 & \!\!-1& 11 \\
-1 & 1 & 1 & -5 \\ 
\end{amatrix}
$$

%%%%%%%%%%%%%%%%%%%

\item Solve the vector equation by applying ERO matrices to each side of the equation to perform elimination. Show each matrix explicitly as in \hyperlink{Undoing}{Example~\ref{slowly}}.

\begin{eqnarray*}
\begin{pmatrix}
3	&6 	&2 \\ %-3
5 	&9 	&4 \\ %1
2	&4	&2 \\ %0
\end{pmatrix} 
\begin{pmatrix}
 x \\ 
y \\
z 
\end{pmatrix} 
=
\begin{pmatrix}
-3 \\ 
1  \\
0  \\
\end{pmatrix} 
\end{eqnarray*}

%%%%%%%%%%%%%%%%%%%

\item Solve this vector equation by finding the inverse of the matrix through $(M|I)\sim (I|M^{-1})$ and then applying $M^{-1}$ to both sides of the equation. 
\begin{eqnarray*}
\begin{pmatrix}
2	&1 	&1 \\ %9
1 	&1 	&1 \\ %6
1	&1	&2 \\ %7
\end{pmatrix} 
\begin{pmatrix}
 x \\ 
y \\
z 
\end{pmatrix} 
=
\begin{pmatrix}
9 \\ 
6  \\
7  \\
\end{pmatrix} 
\end{eqnarray*}


%%%%%%%%%%%%%%%%%%%

\item Follow the method of  \hyperlink{elldeeeww}{Examples~\ref{factorize} and~\ref{factorizes}} to find the $LU$ and $LDU$ factorization of 
\begin{eqnarray*}
\begin{pmatrix}
3	&3 	&6 \\ %0 %2
3 	&5 	&2 \\ %1 %1
6	&2	&5 \\ %0 %1
\end{pmatrix} .
\end{eqnarray*}



%%%%%%%%%%%%%%%%%%%%

\item 
Multiple matrix equations with the same matrix can be solved simultaneously. 
\begin{enumerate}
\item Solve both systems by performing elimination on just one augmented matrix.
\begin{eqnarray*}
\begin{pmatrix}
2	&-1 	&-1 \\ %0 %2
-1 	&1 	&1 \\ %1 %1
1	&-1	&0 \\ %0 %1
\end{pmatrix} 
\begin{pmatrix}
 x \\ 
y \\
z 
\end{pmatrix} 
=
\begin{pmatrix}
0\\ 
1  \\
0  \\
\end{pmatrix} 
,~
\begin{pmatrix}
2	&-1 	&-1 \\ %0 %2
-1 	&1 	&1 \\ %1 %1
1	&-1	&0 \\ %0 %1
\end{pmatrix} 
\begin{pmatrix}
 a \\ 
b \\
c 
\end{pmatrix} 
=
\begin{pmatrix}
2\\ 
1  \\
1  \\
\end{pmatrix} 
\end{eqnarray*}
\item Give an interpretation of the columns of $M^{-1}$ in $(M|I)\sim (I|M^{-1})$ in terms of solutions to certain systems of linear equations.
\end{enumerate}

%%%%%%%%%%%%%%%%%%%%%%%%

\item How can you convince your fellow students to never make this mistake?
\begin{eqnarray*}
\begin{amatrix}{3} 
1 & 0 & 2 & 3 \\ 
0 & 1 & 2& 3 \\
2 & 0 & 1 & 4 \\
\end{amatrix} 
& 
\stackrel{R_1'=R_1+R_2}{
\stackrel{R_2'=R_1-R_2}{ 
\stackrel{\ R_3'= R_1+2R_2}{\sim}}}
&
\begin{amatrix}{3} 
1 & 1 & 4 & 6 \\
1 & \!\!-1 & 0& 0 \\
1 & 2 & 6 & 9 
\end{amatrix}
\end{eqnarray*}

\item Is $LU$ factorization of a matrix unique?  Justify your answer.


\item[$\infty$.] If you randomly create a matrix by picking numbers out of the blue, it will probably be difficult to perform elimination or factorization; fractions and large numbers will probably be involved. To invent simple problems it is better to start with a simple answer:
\begin{enumerate}
\item Start with any augmented matrix in RREF. Perform EROs to make most of the components non-zero. Write the result on a separate piece of paper and give it to your friend. Ask that friend to find RREF of the augmented matrix you gave them. Make sure they get the same augmented matrix you started with.  
\item Create  an upper triangular matrix $U$ and a lower triangular matrix~$L$ with only $1$s on the diagonal. Give the result to a friend to factor into $LU$ form. 
\item Do the same with an $LDU$ factorization. 
\end{enumerate}
\end{enumerate}

\phantomnewpage

