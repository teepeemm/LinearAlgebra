
\subsection{\elemRowOpsTitle: Example}

%%%Insert this to get the typewriter font so it looks like a real movie script
{\ttfamily
\fontdimen2\font=0.4em
\fontdimen3\font=0.2em
\fontdimen4\font=0.1em
\fontdimen7\font=0.1em
\hyphenchar\font=`\-


%%%%put a hypertarget around the opening bit of text
\hypertarget{script_elementary_row_operations_example}{We have three basic rules} 
\begin{enumerate}
\item Row Swap
\item Scalar Multiplication
\item Row Sum
\end{enumerate}

Lets look at an example. The system 
\begin{align*}
 3x + y & = 7\\
 x + 2y & = 4
\end{align*}

is something we learned to solve in high school algebra. Now we can write it in augmented matrix for this way
\[
\left( \begin{array}{cc|c}
3 & 1 & 7 \\
1 & 2 & 4  
\end{array} \right)\, .
\]

We can see what these operations allow us to do:
\begin{enumerate}
\item Row swap allows us to switch the order of rows. In this example there are only two equations, so I will switch them. This will work with a larger system as well, but you have to decide which equations to switch. So we get
\begin{align*}
 x + 2y &= 4 \\
3x + y &= 7
\end{align*}
The augmented matrix looks like 
\[
\left( \begin{array}{cc|c}
1 & 2 & 4 \\
3 & 1 & 7
\end{array} \right)\, .
\]
Notice that this won't change the solution of the system, but the augmented matrix will look different. This is where we can say that the original augmented matrix is equivalent to the one with the rows swapped.
This will work with a larger system as well, but you have to decide which equations, or rows to switch. Make sure that you don't forget to switch the entries in the right-most column.

\item Scalar multiplication allows us to multiply both sides of an equation by a non-zero constant.
So if we are starting with
\begin{align*}
 x + 2y &= 4 \\
3x + y &= 7
\end{align*}
Then we can multiply the first equation by $-3$ which is a non-zero scalar. This operation will give us 
 \begin{align*}
 -3x + -6y &= -12 \\
3x + y &= 7
\end{align*}
which has a corresponding augmented matrix
\[
\left( \begin{array}{cc|c}
-3 & -6 & -12 \\
3 & 1 & 7
\end{array} \right)\, .
\]
Notice that we have multiplied the entire first row by $-3$, and this changes the augmented matrix, but not the solution of the system. We are not allowed to multiply by zero because it would be like replacing one of the equations with $0=0$, effectively destroying the information contained in the equation. 

\item Row summing allows us to add one equation to another. In our example we could start with 
\begin{align*}
 -3x + -6y &= -12 \\
3x + y &= 7
\end{align*} 
and replace the first equation with the sum of both equations. So we get 
 \begin{align*}
 -3x + 3x + -6y + y &= -12 + 7 \\
3x + y &= 7,
\end{align*} 
which after some simplification is translates to 
\[
\left( \begin{array}{cc|c}
0 & -5 & -5 \\
3 & 1 & 7
\end{array} \right).
\]

When using this row operation make sure that you end up with as many equations as you started with. Here we replaced the first equation with a sum, but the second equation remained untouched.

In the example, notice that the $x$-terms in the first equation disappeared, which makes it much easier to solve for $y$. Think about what the next steps for solving this system would be using the language of elementary row operations. 

\end{enumerate}


%%%%don't forget to close the bracket so the stuff after your file doesn't look like a movie!
}

\newpage

