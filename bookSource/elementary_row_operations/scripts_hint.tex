
\subsection*{Hint for Review Question~\ref*{Ihavenoslons}}

%%%Insert this to get the typewriter font so it looks like a real movie script
{\ttfamily
\fontdimen2\font=0.4em
\fontdimen3\font=0.2em
\fontdimen4\font=0.1em
\fontdimen7\font=0.1em
\hyphenchar\font=`\-





%%%%put a hypertarget around the opening bit of text
\hypertarget{video_elementary_row_operations_hint}{The first part} for Review Question~\ref{Ihavenoslons} is simple--just 
write out the associated linear system and you will find the equation $0=6$ which is inconsistent. Therefore we learn 
that we must avoid a row of zeros preceding a non-vanishing entry after the vertical bar.

Turning to the system of equations, we first write out the augmented matrix and then perform two row operations
\begin{eqnarray*}
&&\left(\begin{array}{ccc|c}1&-3&0&6\\1&0&3&-3\\2&k&3-k&1\end{array}\right)\\[2mm]
&\stackrel{R_2-R_1;R_3-2R_1}\sim&
\left(\begin{array}{ccc|c}1&-3&0&6\\0&3&3&-9\\0&k+6&3-k&-11\end{array}\right)\, .
\end{eqnarray*}
Next we would like to subtract some amount of $R_2$ from $R_3$ to achieve a zero in the third entry of the second column. But if 
$$k+6=3-k\Rightarrow k=-\frac32\, ,$$
this would produce zeros in the third row before the vertical line.
You should also check that this does not make the whole third line zero. You now have enough information to write a complete solution.


%%%%don't forget to close the bracket so the stuff after your file doesn't look like a movie!
}

%\newpage
