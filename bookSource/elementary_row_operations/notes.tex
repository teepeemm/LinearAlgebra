
\chapter{\elemRowOpsTitle}

%\hypertarget{Elementary Row Operations} %what is this? 

Elementary row operations are  systems of linear equations between old and new  rows. 


\begin{example} (\hypertarget{Keeping track of EROs with equations between rows}{Keeping track of EROs with equations between rows})\\
We will refer to the new $k$th row as $R'_k$ and the old $k$th row as $R_k$.
\begin{eqnarray*}
\begin{amatrix}{3} 
0 & 1 & 1 & 7 \\ 
2 & 0 & 0& 4 \\
0& 0 & 1 & 4 \\
\end{amatrix} 
& \stackrel{R_1'=0R_1+\phantom{1}R_2+0R_3}{\stackrel{R_2'=\phantom{1}R_1+0R_2+0R_3}{ \stackrel{R_3'= 0R_1+0R_2+\phantom{1}R_3}{\sim}}}&
\begin{amatrix}{3} 
2 & 0 & 0 & 4 \\
0 & 1 & 1& 7 \\
0 & 0 & 1 & 4 \\ 
\end{amatrix}
\\[.2cm]
& \stackrel{R_1'=\frac12 R_1+0R_2+0R_3}{\stackrel{R_2'=0R_1+\phantom{1}R_2+0R_3}{ \stackrel{R_3'= 0R_1+0R_2+\phantom{1}R_3}{\sim}} }&
\begin{amatrix}{3} 
1 & 0 & 0 & 2 \\
0 & 1 & 1& 7 \\
0 & 0 & 1 & 4 \\ 
\end{amatrix}
\\[.2cm]
& \stackrel{R_1'= \phantom{1}R_1+0R_2+0R_3}{\stackrel{R_2'=0R_1+\phantom{1}R_2-\phantom{1}R_3}{ \stackrel{R_3'=0R_1+0R_2+\phantom{1} R_3}{\sim}} }&
\begin{amatrix}{3} 
1 & 0 & 0 & 2 \\
0 & 1 & 0& 3 \\
0 & 0 & 1 & 4 \\ 
\end{amatrix}
\end{eqnarray*}
\end{example}

%\videoscriptlink{elementary_row_operations_example.mp4}{Example}{script_elementary_row_operations_example}

%\begin{center}\href{\webworkurl ReadingHomework3/1/}{Reading homework: problem 3.1}\end{center}
%

%%%%%%%%%%%%%%%%%%%%%%%%%%%%%%%%%%%%%%

\section{EROs as Matrices}
The matrix describing the system of equations between rows performs the corresponding ERO on the augmented matrix:
\begin{example} (Performing EROs with Matrices)\\
\begin{eqnarray*}
\begin{pmatrix}
0 & 1 & 0  \\ 
1 & 0 & 0 \\
0& 0 & 1  
\end{pmatrix} 
\begin{amatrix}{3} 
0 & 1 & 1 & 7 \\ 
2 & 0 & 0& 4 \\
0& 0 & 1 & 4 
\end{amatrix} 
=
\begin{amatrix}{3} 
2 & 0 & 0 & 4 \\
0 & 1 & 1& 7 \\
0 & 0 & 1 & 4 \\ 
\end{amatrix}
\\ %second line
\begin{pmatrix}
\frac12 & 0 & 0  \\ 
0 & 1 & 0 \\
0& 0 & 1  \\
\end{pmatrix}
\begin{amatrix}{3} 
2 & 0 & 0 & 4 \\
0 & 1 & 1& 7 \\
0 & 0 & 1 & 4 \\ 
\end{amatrix}
=
\begin{amatrix}{3} 
1 & 0 & 0 & 2 \\
0 & 1 & 1& 7 \\
0 & 0 & 1 & 4 \\ 
\end{amatrix}
\\ %third line
\begin{pmatrix}
1&0&0\\
0&1&-1\\
0&0&1
\end{pmatrix}
\begin{amatrix}{3} 
1 & 0 & 0 & 2 \\
0 & 1 & 1& 7 \\
0 & 0 & 1 & 4 \\ 
\end{amatrix} 
=
\begin{amatrix}{3} 
1 & 0 & 0 & 2 \\
0 & 1 & 0& 3 \\
0 & 0 & 1 & 4 \\ 
\end{amatrix}
\end{eqnarray*}
\end{example}

This obviously involved more writing than Gaussian elimination. 
The point here is that realizing EROs as matrices allows us to make concrete the notion of ``dividing by a matrix'' \hyperlink{ch1divide}{alluded to in chapter 1}; we can now perform manipulations on both sides of an equation in a familiar way.

\begin{example} (Undoing $A$ in $Ax=b$ slowly, using $A=6=3\cdot2$)
\begin{eqnarray*}
6x&=&\phantom{ 3^{-1}} 12 \\
3^{-1}6x&=&3^{-1}12 \\
 2x&=&\phantom{3^{-1}~}4  \\
 2^{-1}2x&=&2^{-1}~4\\ %  \Leftrightarrow 
  1x&=&\phantom{3^{-1}~} 2
\end{eqnarray*}
\end{example}

\noindent
In particular, matrices corresponding to EROs undo a matrix step by step.
\begin{example} (\hypertarget{Undoing} $A$ in $Ax=b$ slowly, Using $A=M=...$)
\begin{eqnarray*}
\begin{pmatrix}
0 & 1 & 1  \\ 
2 & 0 & 0 \\
0& 0 & 1  \\
\end{pmatrix} 
\begin{pmatrix}
 x \\ 
y \\
z 
\end{pmatrix} 
&=&
\phantom{
\begin{pmatrix}
0 & 1 & 0  \\ 
1 & 0 & 0 \\
0& 0 & 1  \\
\end{pmatrix} 
}
~
\begin{pmatrix}
 7 \\ 
4 \\
4\\
\end{pmatrix} 
\\\nn %second line
%
\begin{pmatrix}
0 & 1 & 0  \\ 
1 & 0 & 0 \\
0& 0 & 1  \\
\end{pmatrix} 
%
\begin{pmatrix}
0 & 1 & 1  \\ 
2 & 0 & 0 \\
0& 0 & 1  \\
\end{pmatrix} 
\begin{pmatrix}
 x \\ 
y \\
z 
\end{pmatrix} 
&=&
\begin{pmatrix}
0 & 1 & 0  \\ 
1 & 0 & 0 \\
0& 0 & 1  \\
\end{pmatrix} 
~
\begin{pmatrix}
 7 \\ 
4 \\
4\\
\end{pmatrix} 
\\\nn %third line
\phantom{-}
\begin{pmatrix}
2 & 0 & 0 \\
0 & 1 & 1  \\ 
0& 0 & 1  \\
\end{pmatrix} 
\begin{pmatrix}
 x \\ 
y \\
z 
\end{pmatrix} 
&=&
\phantom{
\begin{pmatrix}
0 & 1 & 0  \\ 
1 & 0 & 0 \\
0& 0 & 1  \\
\end{pmatrix} 
}
~
\begin{pmatrix}
4 \\
7 \\ 
4\\
\end{pmatrix} 
\\ \nn %fourth line
\begin{pmatrix}
\frac12 & 0 & 0  \\ 
0 & 1 & 0 \\
0& 0 & 1  \\
\end{pmatrix} 
\begin{pmatrix}
2 & 0 & 0 \\
0 & 1 & 1  \\ 
0& 0 & 1  \\
\end{pmatrix} 
\begin{pmatrix}
 x \\ 
y \\
z 
\end{pmatrix} 
&=&
%\phantom{
\begin{pmatrix}
\frac12 & 0 & 0  \\ 
0 & 1 &  0\\
0& 0 & 1  \\
\end{pmatrix} 
%}
~
\begin{pmatrix}
4 \\
7 \\ 
4\\
\end{pmatrix} 
\\\nn %fifth line
\phantom{
\begin{pmatrix}
\frac12 & 0 & 0  \\ 
0 & 1 & 0 \\
0& 0 & 1  \\
\end{pmatrix} 
}
\begin{pmatrix}
1 & 0 & 0 \\
0 & 1 & 1  \\ 
0& 0 & 1  \\
\end{pmatrix} 
\begin{pmatrix}
 x \\ 
y \\
z 
\end{pmatrix} 
&=&
\phantom{
\begin{pmatrix}
\frac12 & 0 & 0  \\ 
0 & 1 &  0\\
0& 0 & 1  \\
\end{pmatrix} 
}
~
\begin{pmatrix}
2 \\
7 \\ 
4\\
\end{pmatrix} 
\\ \nn %sixth line
%\phantom{
\begin{pmatrix}
1 & 0 & 0  \\ 
0 & 1 & -1 \\
0& 0 & 1  \\
\end{pmatrix} 
%}
\begin{pmatrix}
1 & 0 & 0 \\
0 & 1 & 1  \\ 
0& 0 & 1  \\
\end{pmatrix} 
\begin{pmatrix}
 x \\ 
y \\
z 
\end{pmatrix} 
&=&
%\phantom{
\begin{pmatrix}
1 & 0 & 0  \\ 
0 & 1 &  -1\\
0& 0  &  1  \\
\end{pmatrix} 
%}
\! \!
\begin{pmatrix}
2 \\
7 \\ 
4\\
\end{pmatrix} 
\\\nn%%%%%%%%%%%%%%%%%%%%%%%7th line
\phantom{
\begin{pmatrix}
1 & 0 & 0  \\ 
0 & 1 & -1 \\
0& 0 & 1  \\
\end{pmatrix} 
}
\begin{pmatrix}
1 & 0 & 0 \\
0 & 1 & 0  \\ 
0& 0 & 1  \\
\end{pmatrix} 
\begin{pmatrix}
 x \\ 
y \\
z 
\end{pmatrix} 
&=&
\phantom{
\begin{pmatrix}
1 & 0 & 0  \\ 
0 & 1 &  -1\\
0& 0  &  1  \\
\end{pmatrix} 
}
\! 
\begin{pmatrix}
2 \\
3 \\ 
4\\
\end{pmatrix} 
\end{eqnarray*}
\end{example}



\noindent
This is another way of thinking about what is happening in the process of elimination which feels more like elementary algebra in the sense that you ``do something to both sides of an equation" until you have a solution. 


%%%%%%%%%%%%%%%%%%

\section{Recording EROs in $[M | I ] $}
Just as we put together $3^{-1}2^{-1}=6^{-1}$ to get a single thing to apply to both sides of $6x=12$ to undo $6$, we should put together multiple EROs  to get a single thing that undoes our matrix. 
To do this, augment by the identity matrix (not just a single column) and then perform Gaussian elimination. 
There is no need to write the EROs as systems of equations or as matrices while doing this. 
%That is, perform the EROs without the their corresponding matrices.

\begin{example} (Collecting EROs that \hypertarget{undo a matrix}{undo a matrix})
\begin{eqnarray*}
\left(\begin{array}{ccc|ccc}
0 & 1 & 1 &1 &0 &0\\ 
2 & 0 & 0 &0&1&0\\
0& 0 & 1   &0  &0 &1\\
\end{array}  \right)
\sim
&\left(\begin{array}{ccc|ccc}
2 & 0 & 0 &0&1&0\\
0 & 1 & 1 &1 &0 &0\\ 
0& 0 & 1   &0  &0 &1\\
\end{array}  \right)&
\\
\sim
&\left(\begin{array}{ccc|ccc}
1 & 0 & 0 &0&\frac12&0\\
0 & 1 & 1 &1 &0 &0\\ 
0& 0 & 1   &0  &0 &1\\
\end{array}  \right)&
\sim
\left(\begin{array}{ccc|ccc}
1 & 0 & 0 &0&\frac12&0\\
0 & 1 & 0 &1 &0 &-1\\ 
0& 0 & 1   &0  &0 &1\\
\end{array}  \right)
\end{eqnarray*}
\end{example}
\noindent
As we changed the left slot from the matrix $M$ to the identity matrix, the right slot changed from the identity matrix to the matrix which undoes $M$: 
\begin{example} (Checking that \hypertarget{inversie}{one matrix undoes another})
\begin{eqnarray*}
\left(\begin{array}{ccc}
0&\frac12&0\\
1 &0 &-1\\ 
0  &0 &1\\
\end{array}  \right)
\left(\begin{array}{ccc}
0&1&1\\
2 &0 &0\\ 
0  &0 &1\\
\end{array}  \right)
=
\left(\begin{array}{ccc}
1  &0 &0\\
0  &1 &0\\ 
0  &0 &1\\
\end{array}  \right) \, .
\end{eqnarray*}
If the matrices are composed in the opposite order, the result is the same.
\begin{eqnarray*}
\left(\begin{array}{ccc}
0&1&1\\
2 &0 &0\\ 
0  &0 &1\\
\end{array}  \right)
\left(\begin{array}{ccc}
0&\frac12&0\\
1 &0 &-1\\ 
0  &0 &1\\
\end{array}  \right)
=
\left(\begin{array}{ccc}
1  &0 &0\\
0  &1 &0\\ 
0  &0 &1\\
\end{array}  \right) \, .
\end{eqnarray*}
\end{example}

%\reading{3}{1}


In abstract generality, let $M$ be some matrix and, as always, let $I$ stand for the identity matrix. Imagine the process of performing elementary row operations to bring $M$ to the identity matrix. 
\begin{equation*}
(M | I) \sim ( E_1M| E_1)\sim (E_2E_1 M | E_2 E_1) \sim \cdots \sim (I | \cdots E_2E_1 )
\end{equation*}
Ellipses stand for additional EROs. The result is a product of matrices that form a matrix which undoes $M$
\begin{equation*}
\cdots E_2 E_1 M =  I 
\end{equation*}
This is only true if RREF of $M$ is the identity matrix.  
In that case, we say $M$ is {\itshape invertible}\index{invertiblech3}. 


Much use is made of the fact that invertible matrices can be undone with EROs. 
To begin with, since each  elementary row operation has an inverse, 
\[
M= E_1^{-1} E_2^{-1} \cdots
\]
while the inverse of $M$ is 
\begin{equation*}
M^{-1}=\cdots E_2 E_1 
\end{equation*}
This is symbolically verified as
\begin{equation*}
M^{-1}M=\cdots E_2 E_1\, E_1^{-1} E_2^{-1} \cdots
=\cdots E_2 \, E_2^{-1} \cdots = \cdots = I
\end{equation*}
Thus, if $M$ is invertible then  $M$ and can be expressed as the product of EROs. (The same is true for its inverse.) This has the feel of the fundamental theorem of arithmetic (integers can be expressed as the product of primes) or the fundamental theorem of algebra (polynomials can be expressed as the product of first order polynomials); EROs are the building blocks of invertible matrices. 




\section{Three Kinds of EROs, Three Kinds of Matrices}

%To use this in concrete examples, one uses the fact that i
We now work toward concrete examples and applications. 
It is surprisingly easy to translate between EROs as descriptions of rows and as matrices.
The matrices corresponding to these kinds are close in form to the identity matrix:
\begin{itemize}
\item Row Swap: Identity matrix with two rows swapped
\item Scalar Multiplication:  Identity matrix with one diagonal entry not 1.
\item Row Sum: The identity matrix with one off diagonal entry not 0.
\end{itemize}


\begin{example} (Correspondences between EROs and their matrices)
\begin{itemize}
\item The row swap matrix that swaps the 2nd and 4th row is the identity matrix with the 2nd and 4th row swapped; 
\[
\begin{pmatrix}
1&0&0&0&0\\
0&0&0&1&0\\
0&0&1&0&0\\
0&1&0&0&0\\
0&0&0&0&1\\
\end{pmatrix}
\]
\item
The scalar multiplication matrix that replaces the 3rd row with 7 times the 3rd row is the identity matrix with 7 in the 3rd row instead of 1; 
\[
\begin{pmatrix}
1&0&0&0\\
0&1&0&0\\
0&0&7&0\\
0&0&0&1\\
\end{pmatrix}
\]

\item The row sum matrix that replaces the 4th row with the 4th row plus 9 times the 2nd row is the identity matrix with a 9 in the  4th row, 2nd column
\[
\begin{pmatrix}
1&0&0&0&0&0&0\\
0&1&0&0&0&0&0\\
0&0&1&0&0&0&0\\
0&9&0&1&0&0&0\\
0&0&0&0&1&0&0\\
0&0&0&0&0&1&0\\
0&0&0&0&0&0&1\\
\end{pmatrix}
\]
\end{itemize}
\end{example}

We can write an explicit factorization of a matrix into EROs by keeping track of the EROs used in getting to RREF.

\begin{example} (Express $M$ from the  
\hyperlink{undo a matrix}{earlier example} as a product of EROs)\\
Note that in the previous example 
one of each of the kinds of EROs is used, in the order just given.
Elimination looked like 
\begin{eqnarray*}
M=
\left(\begin{array}{ccc}
0 & 1 & 1 \\ 
2 & 0 & 0 \\\
0& 0 & 1  
\end{array}  \right)
\stackrel{E_1}{\sim}
\left(\begin{array}{ccc}
2 & 0 & 0 \\
0 & 1 & 1 \\ 
0& 0 & 1  
\end{array}  \right)
\stackrel{E_2}{\sim}
\left(\begin{array}{ccc}
1 & 0 & 0 \\
0 & 1 & 1 \\ 
0& 0 & 1  
\end{array}  \right)
\stackrel{E_3}{\sim}
\left(\begin{array}{ccc}
1 & 0 & 0 \\
0 & 1 & 0 \\ 
0& 0 & 1  
\end{array}  \right)
=I
\end{eqnarray*}
where the EROs matrices are 
\begin{eqnarray*}
E_1
= \left(\begin{array}{ccc}
0  &1 &0\\
1  &0 &0\\ 
0  &0 &1\\
\end{array}  \right)
,~
E_2
= \left(\begin{array}{ccc}
\frac12  &0 &0\\
0  &1 &0\\ 
0  &0 &1\\
\end{array}  \right) , ~
E_3
= \left(\begin{array}{ccc}
1  &0 &0\\
0  &1 & -1\\ 
0  &0 &1\\
\end{array}  \right) \,.
\end{eqnarray*}
%Composing these gives (by matrix multiplication rules worked out in %\hyperref{}
%\begin{eqnarray*}
%E_3E_2E_1
%&= &
% \left(\begin{array}{ccc}
%1  &0 &0\\
%0  &1 & -1\\ 
%0  &0 &1\\
%\end{array}  \right)
% \left(\begin{array}{ccc}
%\frac12  &0 &0\\
%0  &1 &0\\ 
%0  &0 &1\\
%\end{array}  \right)
%\left(\begin{array}{ccc}
%0  &1 &0\\
%1  &0 &0\\ 
%0  &0 &1\\
%\end{array}  \right)
%\\ %2nd line
%&=& \left(\begin{array}{ccc}
%1 &0 &0\\
%0  &1 & -1\\ 
%0  &0 &1\\
%\end{array}  \right) 
%\left(\begin{array}{ccc}
%0  	&\frac12 	& 0\\ 
%1  	&0	 	&0\\
%0  	&0 		&1
%\end{array}  \right) 
%=%%%%%%%%%%%%%%%%
%\left(\begin{array}{ccc}
%0  	&\frac12 	& 0\\ 
%1  	&0	 	&-1\\
%0  	&0 		&1
%\end{array}  \right) 
% \, .
%\end{eqnarray*}
%We showed this was $M^{-1}$ \hyperlink{inversie}{earlier}. 
The inverse of the ERO matrices (corresponding to the description of the reverse row maniplulations)
\begin{eqnarray*}
E_1^{-1}
= \left(\begin{array}{ccc}
0  &1 &0\\
1  &0 &0\\ 
0  &0 &1\\
\end{array}  \right)
,~
E_2^{-1}
= \left(\begin{array}{ccc}
2  &0 &0\\
0  &1 &0\\ 
0  &0 &1\\
\end{array}  \right) , ~
E_3^{-1}
= \left(\begin{array}{ccc}
1  &0 &0\\
0  &1 & 1\\ 
0  &0 &1\\
\end{array}  \right) \,.
\end{eqnarray*}
Multiplying these gives 
\begin{eqnarray*}
E_1^{-1}E_2^{-1}E_3^{-1}
&=& 
\left(\begin{array}{ccc}
0  &1 &0\\
1  &0 &0\\ 
0  &0 &1\\
\end{array}  \right)
 \left(\begin{array}{ccc}
2  &0 &0\\
0  &1 &0\\ 
0  &0 &1\\
\end{array}  \right) 
\left(\begin{array}{ccc}
1  &0 &0\\
0  &1 & 1\\ 
0  &0 &1\\
\end{array}  \right) 
\\ %2nd line
&=&
\left(\begin{array}{ccc}
0  &1 &0\\
1  &0 &0\\ 
0  &0 &1\\
\end{array}  \right)
 \left(\begin{array}{ccc}
2  &0 &0\\
0  &1 &1\\ 
0  &0 &1\\
\end{array}  \right) 
= %%%%%%%
\left(\begin{array}{ccc}
 0 &1 &1\\
2  &0 &0\\ 
0  &0 &1\\
\end{array}  \right)  = M \, .
\end{eqnarray*}
\end{example}

\section{$LU$, $LDU$, and $LDPU$ Factorizations}
This process of elimination can be stopped half way to obtain decompositions frequently used in large computations in sciences and engineering. 
The first half of the elimination process is to eliminate entries below the diagonal. 
leaving a matrix which is called {\itshape upper triangular}\index{upper triangular}. The ERO matrices which do this part of the elimination are {\itshape lower triangular}\index{lower triangular}, as are their inverses. But putting together the upper triangular and lower triangular parts one obtains the so called $LU$ factorization.

\newpage
\begin{example} ($LU$ \hypertarget{elldeeeww}{ factorization})
\begin{eqnarray*}
M=
\begin{pmatrix}
2&0&-3&1\\
0&1&2&2\\
-4&0&9&2\\
0&-1&1&-1\\
\end{pmatrix}
&\stackrel{E_1}{\sim}&
\begin{pmatrix}
2&0&-3&1\\
0&1&2&2\\
0&0&3&4\\
0&-1&1&-1\\
\end{pmatrix}
\\
&\stackrel{E_2}{\sim}&
~~\begin{pmatrix}
2&0&-3&1\\
0&1&2&2\\
0&0&3&4\\
0&0&3&1\\
\end{pmatrix}
~~\stackrel{E_3}{\sim}~
\begin{pmatrix}
2&0&-3&1\\
0&1&2&2\\
0&0&3&4\\
0&0&0&-3\\
\end{pmatrix}
:=U
\end{eqnarray*}

%%%%%%%%%%%%%%%%%%%%%%%%%%
where the EROs and their inverses are 
%%%%%%%%%%%%%%%
\begin{eqnarray*}
E_1=
\begin{pmatrix}
1&0&0&0\\
0&1&0&0\\
2&0&1&0\\
0&0&0&1\\
\end{pmatrix} \, ,~~~~
E_2=
\begin{pmatrix}
1&0&0&0\\
0&1&0&0\\
0&0&1&0\\
0&1&0&1\\
\end{pmatrix} \, ,~~
E_3=
\begin{pmatrix}
1&0&0&0\\
0&1&0&0\\
0&0&1&0\\
0&0&-1&1\\
\end{pmatrix} \, 
\\ %%%%%%%%second line
E_1^{-1}=
\begin{pmatrix}
1&0&0&0\\
0&1&0&0\\
-2&0&1&0\\
0&0&0&1\\
\end{pmatrix}  , \,
E_2^{-1}=
\begin{pmatrix}
1&0&0&0\\
0&1&0&0\\
0&0&1&0\\
0&-1&0&1\\
\end{pmatrix}  , \,
E_3^{-1}=
\begin{pmatrix}
1&0&0&0\\
0&1&0&0\\
0&0&1&0\\
0&0&1&1\\
\end{pmatrix} \, .
\end{eqnarray*}
applying the inverses of the EROs to both sides of the equality  $U=E_3E_2E_1M$ gives 
$M=E_1^{-1}E_2^{-1}E_3^{-1}U$ or 
\begin{eqnarray*}
\!\!\! %oi vey
\begin{pmatrix}
2&0&-3&1\\
0&1&2&2\\
-4&0&9&2\\
0&-1&1&-1\\
\end{pmatrix}
\!\!\!\!\!\!&=&\!\!\!\!\!\!
\begin{pmatrix}
1&0&0&0\\
0&1&0&0\\
-2&0&1&0\\
0&0&0&1\\
\end{pmatrix} \!\!\!
\begin{pmatrix}
1&0&0&0\\
0&1&0&0\\
0&0&1&0\\
0&-1&0&1\\
\end{pmatrix}\!\!\!
\begin{pmatrix}
1&0&0&0\\
0&1&0&0\\
0&0&1&0\\
0&0&1&1\\
\end{pmatrix}\!\!\!
\begin{pmatrix}
2&0&-3&1\\
0&1&2&2\\
0&0&3&4\\
0&0&0&-3\\
\end{pmatrix}
\\
&=&\!\!\!\!
\begin{pmatrix}
1&0&0&0\\
0&1&0&0\\
-2&0&1&0\\
0&0&0&1\\
\end{pmatrix} 
%%%%%%%%
\begin{pmatrix}
1&0&0&0\\
0&1&0&0\\
0&0&1&0\\
0&-1&1&1\\
\end{pmatrix}
%%%%%%%%%%
\begin{pmatrix}
2&0&-3&1\\
0&1&2&2\\
0&0&3&4\\
0&0&0&-3\\
\end{pmatrix}
\\
&=&\!\!\!\!
\begin{pmatrix}
1&0&0&0\\
0&1&0&0\\
-2&0&1&0\\
0&-1&1&1\\
\end{pmatrix} 
%
\begin{pmatrix}
2&0&-3&1\\
0&1&2&2\\
0&0&3&4\\
0&0&0&-3\\
\end{pmatrix} \, .
\end{eqnarray*}
This is a lower triangular matrix times an upper triangular matrix. 
\end{example}

\newpage
What if we stop at a different point in elimination? 
We could multiply rows so that the entries in the diagonal are 1 next. Note that the EROs that do this are diagonal. This gives a slightly different factorization.
\begin{example} ($LDU$ factorization building from previous example)
\begin{eqnarray*}
M=
\begin{pmatrix}
2&0&-3&1\\
0&1&2&2\\
-4&0&9&2\\
0&-1&1&-1\\
\end{pmatrix}
\stackrel{E_3E_2E_1}{\sim}
\begin{pmatrix}
2&0&-3&1\\
0&1&2&2\\
0&0&3&4\\
0&0&0&-3\\
\end{pmatrix}
\stackrel{E_4}{\sim}
\begin{pmatrix}
1&0&\frac{-3}{2}&\frac{1}{2}\\
0&1&2&2\\
0&0&3&4\\
0&0&0&-3\\
\end{pmatrix}
\\
\stackrel{E_5}{\sim}
\begin{pmatrix}
1&0&\frac{-3}{2}&\frac{1}{2}\\
0&1&2&2\\
0&0&1&\frac43\\
0&0&0&-3\\
\end{pmatrix}
\stackrel{E_6}{\sim}
\begin{pmatrix}
1&0&\frac{-3}{2}&\frac{1}{2}\\
0&1&2&2\\
0&0&1&\frac43\\
0&0&0&1\\
\end{pmatrix}
=:U
\\
%%%%%%%%the EROs
E_4=
\begin{pmatrix}
\frac12&0&0&0\\
0&1&0&0\\
0&0&1&0\\
0&0&0&1\\
\end{pmatrix} , \, ~~
E_5=
\begin{pmatrix}
1&0&0&0\\
0&1&0&0\\
0&0&\frac13&0\\
0&0&0&1\\
\end{pmatrix} , \, ~~
E_6=
\begin{pmatrix}
1&0&0&0\\
0&1&0&0\\
0&0&1&0\\
0&0&0&-\frac13\\
\end{pmatrix} , \, 
\\
%%%%%%%%the ERO inverses
E_4^{-1}=
\begin{pmatrix}
2&0&0&0\\
0&1&0&0\\
0&0&1&0\\
0&0&0&1\\
\end{pmatrix} , \, 
E_5^{-1}=
\begin{pmatrix}
1&0&0&0\\
0&1&0&0\\
0&0&3&0\\
0&0&0&1\\
\end{pmatrix} , \, 
E_6^{-1}=
\begin{pmatrix}
1&0&0&0\\
0&1&0&0\\
0&0&1&0\\
0&0&0&-3\\
\end{pmatrix} , \, 
\end{eqnarray*}
The equation $U=E_6E_5E_4E_3E_2E_1 M$ can be rearranged into  
\[M=(E_1^{-1}E_2^{-1}E_3^{-1})(E_4^{-1}E_5^{-1}E_6^{-1})U.\] 
We calculated the product of the first three factors in the previous example; it was named $L$ there, and we will reuse that name here. The product of the next three factors is diagonal and we wil name it $D$. The last factor we named $U$ (the name means something different in this example than the last example.) The $LDU$ factorization of our matrix is
\begin{eqnarray*}
\begin{pmatrix}
2&0&-3&1\\
0&1&2&2\\
-4&0&9&2\\
0&-1&1&-1\\
\end{pmatrix}
=
\begin{pmatrix}
1&0&0&0\\
0&1&0&0\\
-2&0&1&0\\
0&-1&1&1\\
\end{pmatrix} 
\begin{pmatrix}
2&0&0&0\\
0&1&0&0\\
0&0&3&0\\
0&0&1&-3\\
\end{pmatrix} 
\begin{pmatrix}
1&0&\frac{-3}{2}&\frac{1}{2}\\
0&1&2&2\\
0&0&1&\frac43\\
0&0&0&1\\
\end{pmatrix}
\end{eqnarray*}
\end{example}

The $LDU$ factorization of a matrix is a factorization into blocks of EROs of a various types: $L$ is the product of the inverses of EROs which eliminate below the diagonal by row addition, $D$ the product of inverses of EROs which set the diagonal elements to 1 by row multiplication, and $U$ is the product of inverses of EROs which eliminate above the diagonal by row addition.

\hypertarget{LDPU}{
You} may notice that one of the three kinds of row operation is missing from this story. 
Row exchange my very well be necessary to obtain RREF. Indeed, 
so far in this chapter we have been working under the tacit assumption that 
$M$ can be brought to the identity by just row multiplication and row addition. 
If row exchange is necessary, the resulting factorization is $LDPU$ where $P$ is the product of inverses of EROs that perform row exchange. 

\begin{example} ($LDPU$ factorization, building from previous examples)
\begin{eqnarray*}
M=
\begin{pmatrix}
0&1&2&2\\
2&0&-3&1\\
-4&0&9&2\\
0&-1&1&-1\\
\end{pmatrix}
\stackrel{E_7}{\sim}
\begin{pmatrix}
0&1&2&2\\
2&0&-3&1\\
-4&0&9&2\\
0&-1&1&-1\\
\end{pmatrix}
\stackrel{E_6E_5E_4E_3E_2E_1}{\sim} L
\end{eqnarray*}
\begin{eqnarray*}
E_7=
\begin{pmatrix}
0&1&0&0\\
1&0&0&0\\
0&0&1&0\\
0&0&0&1\\
\end{pmatrix}
=E_7^{-1}
\end{eqnarray*}
\begin{eqnarray*}
M=(E_1^{-1}E_2^{-1}E_3^{-1})(E_4^{-1}E_5^{-1}E_6^{-1}) (E_7^{-1}) U=LDPU\\
\end{eqnarray*}
%%%%%%%last line!
\begin{eqnarray*}
\!\!\!\begin{pmatrix}
0&1&2&2\\
2&0&-3&1\\
-4&0&9&2\\
0&-1&1&-1\\
\end{pmatrix}
\!\!\!=\!\!\!
\begin{pmatrix}
1&0&0&0\\
0&1&0&0\\
-2&0&1&0\\
0&-1&1&1\\
\end{pmatrix} 
\!\!\!
\begin{pmatrix}
2&0&0&0\\
0&1&0&0\\
0&0&3&0\\
0&0&1&-3\\
\end{pmatrix} 
\!\!\!
\begin{pmatrix}
0&1&0&0\\
1&0&0&0\\
0&0&1&0\\
0&0&0&1\\
\end{pmatrix}
\!\!\!
\begin{pmatrix}
1&0&\frac{-3}{2}&\frac{1}{2}\\
0&1&2&2\\
0&0&1&\frac43\\
0&0&0&1\\
\end{pmatrix}
\end{eqnarray*}

\end{example}

%\References{
%Hefferon, Chapter One, Section 1.1 and 1.2
%\\
%Beezer, Chapter SLE, Section RREF
%\\
%Wikipedia, \href{http://en.wikipedia.org/wiki/Row_echelon_form}{Row Echelon Form}
%\\
%Wikipedia, \href{http://en.wikipedia.org/wiki/Elementary_matrix_transformations}{Elementary Matrix Operations}}

\section{Review Problems}




\begin{enumerate}
\item \label{det33} Let $M=\begin{pmatrix}
m^1_1 & m^1_2 & m^1_3\\
m^2_1 & m^2_2 & m^2_3\\
m^3_1 & m^3_2 & m^3_3\\
\end{pmatrix}$.  Use row operations to put $M$ into \emph{row echelon form}.  For simplicity, assume that $m_1^1\neq 0 \neq m^1_1m^2_2-m^2_1m^1_2$.

Prove that $M$ is non-singular if and only if:
\[
m^1_1m^2_2m^3_3 
- m^1_1m^2_3m^3_2 
+ m^1_2m^2_3m^3_1 
- m^1_2m^2_1m^3_3 
+ m^1_3m^2_1m^3_2
- m^1_3m^2_2m^3_1
\neq 0
\]

\phantomnewpage

\item 
\begin{enumerate}
\item What does the matrix $E^1_2=\begin{pmatrix}
0 & 1 \\
1 & 0
\end{pmatrix}$ do to $M=\begin{pmatrix}
a & b \\
d & c
\end{pmatrix}$ under left multiplication?  What about right multiplication?
\item Find elementary matrices $R^1(\lambda)$ and $R^2(\lambda)$ that respectively multiply rows $1$ and $2$ of $M$ by $\lambda$ but otherwise leave $M$ the same under left multiplication.
\item Find a matrix $S^1_2(\lambda)$ that adds a multiple $\lambda$ of row $2$ to row $1$ under left multiplication.
\end{enumerate}

\phantomnewpage

\item Let $M$ be a matrix and $S^i_jM$ the same matrix with rows \(i\) and \(j\) switched.  Explain every line of the 
\hyperlink{rowswap}{series of equations} proving that $\det M = -\det (S^i_jM)$.

\phantomnewpage

%\item \label{prob_inversion_number} This problem is a ``hands-on'' look at why \hyperlink{permutation_parity}{the property} describing the parity of permutations is true.
%
%\hypertarget{inversion_number}{The \emph{inversion number}}\index{Permutation!Inversion number} of a permutation $\sigma$ is the number of pairs $i<j$ such that $\sigma(i)>\sigma(j)$; it's the number of ``numbers that appear left of smaller numbers'' in the permutation.  For example, for the permutation $\rho = [4,2,3,1]$, the inversion number is $5$. The number $4$ comes before $2,3,$ and $1$, and $2$ and $3$ both come before $1$.
%
%Given a permutation $\sigma$, we can make a new permutation $\tau_{i,j} \sigma$ by exchanging the $i$th and $j$th entries of $\sigma$.
%
%\begin{enumerate}
%\item What is the inversion number of the permutation \(\mu=[1,2,4,3]\) that exchanges 4 and 3 and leaves everything else alone? Is it an even or an odd permutation?
%
%\item What is the inversion number of the permutation \(\rho=[4,2,3,1]\) that exchanges 1 and 4 and leaves everything else alone? Is it an even or an odd permutation?
%
%\item What is the inversion number of the permutation \(\tau_{1,3} \mu\)? Compare the parity\footnote{The \emph{parity} of an integer refers to whether the integer is even or odd. Here the parity of a permutation $\mu$ refers to the parity of its inversion number.} of \(\mu\) to the parity of \(\tau_{1,3} \mu.\)
%
%\item What is the inversion number of the permutation \(\tau_{2,4} \rho\)? Compare the parity of \(\rho\) to the parity of \(\tau_{2,4} \rho.\)
%
%\item What is the inversion number of the permutation \(\tau_{3,4} \rho\)? Compare the parity of \(\rho\) to the parity of \(\tau_{3,4} \rho.\)
%\end{enumerate}
%
%\videoscriptlink{elementary_matrices_determinant_hint.mp4}{Problem~\ref{prob_inversion_number} hints}{scripts_elementary_matrices_determinants_hint}

\phantomnewpage

%\item \label{problem_permutation} (Extra credit) Here we will examine a (very) small set of the general properties about permutations and their applications. In particular, we will show that one way to compute the sign of a permutation is by finding the \hyperlink{inversion_number}{inversion number} $N$ of $\sigma$ and we have
%\[
%\sgn(\sigma) = (-1)^N.
%\]
%
%For this problem, let $\mu = [1,2,4,3]$.
%
%\begin{enumerate}
%\item Show that every permutation $\sigma$ can be sorted by only taking simple (adjacent) transpositions\index{Permutation!Simple transposition} $s_i$ where $s_i$ interchanges the numbers in position $i$ and $i+1$ of a permutation $\sigma$ (in our other notation $s_i = \tau_{i,i+1}$). For example $s_2 \mu = [1, 4, 2, 3]$, and to sort $\mu$ we have $s_3 \mu = [1, 2, 3, 4]$.
%
%\item \label{prob_part_relations} We can compose simple transpositions together to represent a permutation (note that the sequence of compositions is not unique), and these are associative, we have an identity (the trivial permutation where the list is in order or we do nothing on our list), and we have an inverse since it is clear that $s_i s_i \sigma = \sigma$. Thus permutations of $[n]$ under composition are an example of a \hyperref[groups]{group}. However note that not all simple transpositions commute with each other since
%\begin{align*}
%s_1 s_2 [1, 2, 3] & = s_1 [1, 3, 2] = [3, 1, 2]
%\\ s_2 s_1 [1, 2, 3] & = s_2 [2, 1, 3] = [2, 3, 1]
%\end{align*}
%(you will prove here when simple transpositions commute). When we consider our initial permutation to be the trivial permutation $e = [1, 2, \dotsc, n]$, we do not write it; for example $s_i \equiv s_i e$ and $\mu = s_3 \equiv s_3 e$. This is analogous to not writing 1 when multiplying. Show that $s_i s_i = e$ (in shorthand $s_i^2 = e$), $s_{i+1} s_i s_{i+1} = s_i s_{i+1} s_i$ for all $i$, and $s_i$ and $s_j$ commute for all $|i - j| \geq 2$.
%
%\item Show that every way of expressing $\sigma$ can be obtained from using the relations proved in part~\ref{prob_part_relations}. In other words, show that for any expression $w$ of simple transpositions representing the trivial permutation $e$, using the proved relations.
%
%\emph{Hint: Use induction on $n$. For the induction step, follow the path of the $(n+1)$-th strand by looking at $s_n s_{n-1} \cdots s_k s_{k\pm1} \cdots s_n$ and argue why you can write this as a subexpression for any expression of $e$. Consider using diagrams of these paths to help.}
%
%\item The simple transpositions \hyperlink{action}{acts on} an $n$-dimensional vector space $V$ by $s_i v = E^i_{i+1} v$ (where $E^i_j$ is \hyperlink{elem_matrix_row_swap}{an elementary matrix}) for all vectors $v \in V$. Therefore we can just represent a permutation $\sigma$ as the matrix $M_{\sigma}$\footnote{Often people will just use $\sigma$ for the matrix when the context is clear.}, and we have $\det(M_{s_i}) = \det(E^i_{i+1}) = -1$. Thus prove that $\det(M_{\sigma}) = (-1)^N$ where $N$ is a number of simple transpositions needed to represent $\sigma$ as a permutation. You can assume that $M_{s_i s_j} = M_{s_i} M_{s_j}$ (it is not hard to prove) and that $\det(A B) = \det(A) \det(B)$ \hyperref[detmultiplicative]{from Chapter~\ref*{elementarydeterminantsII}}.
%
%\emph{Hint: You to make sure $\det(M_{\sigma})$ is well-defined since there are infinite ways to represent $\sigma$ as simple transpositions.}
%
%\item Show that $s_{i+1} s_i s_{i+1} = \tau_{i, i+2}$, and so give one way of writing $\tau_{i, j}$ in terms of simple transpositions? Is $\tau_{i,j}$ an even or an odd permutation? What is $\det(M_{\tau_{i,j}})$? What is the inversion number of $\tau_{i,j}$?
%
%\item The minimal number of simple transpositions needed to express $\sigma$ is called the \emph{length}\index{Permutation!Length} of $\sigma$; for example the length of $\mu$ is 1 since $\mu = s_3$. Show that the length of $\sigma$ is equal to the inversion number of $\sigma$.
%
%\emph{Hint: Find an procedure which gives you a new permutation $\sigma^{\prime}$ where $\sigma = s_i \sigma^{\prime}$ for some $i$ and the inversion number for $\sigma^{\prime}$ is 1 less than the inversion number for $\sigma$.}
%
%\item Show that $(-1)^N = \sgn(\sigma) = \det(M_{\sigma})$, where $\sigma$ is a permutation with $N$ inversions. Note that this immediately implies that $\sgn(\sigma \rho) = \sgn(\sigma) \sgn(\rho)$ for any permutations $\sigma$ and $\rho$.
%\end{enumerate}

\item Let $M'$ be the matrix obtained from $M$ by swapping two columns $i$ and $j$. Show that $\det M'=-\det M $.

\item The scalar triple product of three vectors $u,v,w$ from $\Re^3$ is $u\cdot(v\times w)$. Show that this product is the same as the determinant of the matrix whose columns are $u,v,w$ (in that order). What happens to the scalar triple product when the factors are permuted? 

\item Show that if $M$ is a $3\times 3$ matrix whose third row is a sum of multiples of the other rows ($R_3=aR_2+bR_1$) then $\det M=0$. Show that the same is true if one of the columns is a sum of multiples of the others. 

\end{enumerate}

\phantomnewpage

\newpage

