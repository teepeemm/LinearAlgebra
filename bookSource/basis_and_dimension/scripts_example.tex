
\subsection*{Worked Example}

%%%Insert this to get the typewriter font so it looks like a real movie script
{\ttfamily
\fontdimen2\font=0.4em
\fontdimen3\font=0.2em
\fontdimen4\font=0.1em
\fontdimen7\font=0.1em
\hyphenchar\font=`\-


\hypertarget{basis_and_dimension_example}{In this} 
video we will work through an example of how to extend a set of linearly independent vectors to a basis. For fun, we will take
the vector space 
\[
V=\{(x,y,z,w)|x,y,z,w\in {\mathbb Z}^5\}\, .
\]
This is like four dimensional space ${\mathbb R}^4$ except that the numbers can only be $\{0,1,2,3,4\}$. This is like bits, but now the rule is
\[
0=5\, .
\]
Thus, for example,  $\frac14=4$ because $4\time 4=16=1+3\times 5=1$. Don't get too caught up on this aspect, its a choice of base field designed
to make computations go quicker!

Now, here's the problem we will solve:

\begin{center}
{\bf Find a basis for $V$ that includes the vectors $\begin{pmatrix}1\\2\\3\\4\end{pmatrix}$ and $\begin{pmatrix}0\\3\\2\\1\end{pmatrix}$.}
\end{center}

The way to proceed is to add a known (and preferably simple) basis to the vectors given, thus we consider
\[
v_1=\begin{pmatrix}1\\2\\3\\4\end{pmatrix},\
v_2=\begin{pmatrix}0\\3\\2\\1\end{pmatrix},\
e_1=\begin{pmatrix}1\\0\\0\\0\end{pmatrix},\
e_2=\begin{pmatrix}0\\1\\0\\0\end{pmatrix},\
e_3=\begin{pmatrix}0\\0\\1\\0\end{pmatrix},\
e_4=\begin{pmatrix}0\\0\\0\\1\end{pmatrix}.
\]
The last four vectors are clearly a basis (make sure you understand this....) and are called the {\it canonical basis}\index{Canonical basis}.
We want to keep $v_1$ and $v_2$ but find a way to turf out two of the vectors in the canonical basis leaving us
a basis of four vectors. To do that, we have to study linear independence, or in other words a linear system problem
defined by
\[
0=\alpha_1 e_1 + \alpha_2 e_2 + \alpha_3 v_1 + \alpha_4 v_2 + \alpha_5 e_3 + \alpha_6 e_4 \, .
\]
We want to find solutions for the $\alpha's$ which allow us to determine two of the $e's$.
For that we use an augmented matrix
\[
\left(\begin{array}{cccccc|c}
1&0&1&0&0&0&0\\
2&3&0&1&0&0&0\\
3&2&0&0&1&0&0\\
4&1&0&0&0&1&0
\end{array}\right)\, .
\]
Next comes a bunch of row operations. Note that we have dropped the last column of zeros since it has no information--you can fill in the 
row operations used above the $\sim$'s as an exercise:
\[
\begin{pmatrix}
1&0&1&0&0&0\\
2&3&0&1&0&0\\
3&2&0&0&1&0\\
4&1&0&0&0&1
\end{pmatrix}\sim
\begin{pmatrix}
1&0&1&0&0&0\\
0&3&3&1&0&0\\
0&2&2&0&1&0\\
0&1&1&0&0&1
\end{pmatrix}
\]
\[
\sim
\begin{pmatrix}
1&0&1&0&0&0\\
0&1&1&2&0&0\\
0&2&2&0&1&0\\
0&1&1&0&0&1
\end{pmatrix}
\sim
\begin{pmatrix}
1&0&1&0&0&0\\
0&1&1&2&0&0\\
0&0&0&1&1&0\\
0&0&0&3&0&1
\end{pmatrix}
\]
\[
\sim
\begin{pmatrix}
1&0&1&0&0&0\\
0&1&1&0&3&0\\
0&0&0&1&1&0\\
0&0&0&0&2&1
\end{pmatrix}
\sim
\begin{pmatrix}
1&0&1&0&0&0\\
0&1&1&0&3&0\\
0&0&0&1&1&0\\
0&0&0&0&1&3
\end{pmatrix}
\]
\[
\sim
\begin{pmatrix}
\underline1&0&1&0&0&0\\
0&\underline1&1&0&0&1\\
0&0&0&\underline1&0&2\\
0&0&0&0&\underline1&3
\end{pmatrix}
\]
The pivots are underlined.
The columns corresponding to non-pivot variables are the ones that can be eliminated--their coefficients (the $\alpha$'s)
will be arbitrary, so set them all to zero save for the one next to the vector you are solving for which can be taken to be unity.
Thus that vector can certainly be expressed in terms of previous ones. Hence, altogether, our basis is
\[
\left\{
\begin{pmatrix}1\\2\\3\\4\end{pmatrix} \, , \ 
\begin{pmatrix}0\\3\\2\\1\end{pmatrix} ,\
\begin{pmatrix}0\\1\\0\\0\end{pmatrix} ,\
\begin{pmatrix}0\\0\\1\\0\end{pmatrix}
\right\}\, .
\]
Finally, as a check, note that $e_1=v_1+v_2$ which explains why we had to throw it away.



} % Closing bracket for font

%\newpage
