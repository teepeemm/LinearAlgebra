
\subsection*{Do Matrices Commute?}

{\ttfamily
\fontdimen2\font=0.4em
\fontdimen3\font=0.2em
\fontdimen4\font=0.1em
\fontdimen7\font=0.1em
\hyphenchar\font=`\-

\hypertarget{script_matrices_commute}{This video} shows you a funny property of matrices.
Some matrix properties look just like those for numbers. For example numbers obey
\[
a(bc) = (ab)c\, 
\]
and so do matrices:
\[
A(BC)=(AB)C.
\]
This says the order of bracketing does not matter and is called associativity.
Now we ask ourselves whether the basic property of numbers
\[ab=ba\, ,\]
holds for matrices
\[AB\stackrel?=BA\, .\]
For this, firstly note that we need to work with square matrices even for both orderings to even make sense.
Lets take a simple $2\times 2$ example, let
\[
A=\begin{pmatrix}1&a\\0&1\end{pmatrix}\, ,\qquad
B=\begin{pmatrix}1&b\\0&1\end{pmatrix}\, ,\qquad
C=\begin{pmatrix}1&0\\a&1\end{pmatrix}\, .
\]
In fact, computing $AB$ and $BA$ we get the same result
\[AB=BA=\begin{pmatrix}1&a+b\\0&1\end{pmatrix}\, ,\]
so this pair of matrices do commute. Lets try $A$ and $C$:
\[
AC=\begin{pmatrix}1+a^2&a\\a&1\end{pmatrix}\, ,\qquad\mbox{and}\qquad CA =\begin{pmatrix}1&a\\a&1+a^2\end{pmatrix}
\]
so \[AC\neq CA\]
and this pair of matrices does {\itshape not} commute. Generally, matrices usually do not commute, and the problem of finding those that do is a very interesting one.

} % Closing brace for the font

%\newpage
