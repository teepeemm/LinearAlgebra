
\subsection*{$2\times2$ Example}

%%%Insert this to get the typewriter font so it looks like a real movie script
{\ttfamily
\fontdimen2\font=0.4em
\fontdimen3\font=0.2em
\fontdimen4\font=0.1em
\fontdimen7\font=0.1em
\hyphenchar\font=`\-


\hypertarget{scripts_diagonalization_example}{Lets diagonalize the} matrix M from a previous example
\Videoscriptlink{eigenvectors_and_eigenvalues_example.mp4}{Eigenvalues and Eigenvectors: $2\times 2$ Example}{scripts_eigenvalseigenvects_example}
\[
M = 
\begin{pmatrix}
4 & 2\\
1 &3 
\end{pmatrix}
\]
We found the eigenvalues and eigenvectors of $M$, our solution was \[
\lambda_1 = 5, \, \mathbf{v}_1= \colvec{2\\1} \, \text{ and } \lambda_2 = 2, \, \mathbf{v}_2=\colvec{1\\-1}.
\]
So we can diagonalize this matrix using the formula $D = P^{-1}MP$ where $P= (\mathbf{v}_1, \mathbf{v}_2)$. This means
\[
P = \begin{pmatrix}
2 &1\\
1 &-1 
\end{pmatrix}
\text{ and }
P^{-1} = -\frac{1}{3}\begin{pmatrix}
-1 &-1\\
-1 &2 
\end{pmatrix}
\]
The inverse comes from the formula for inverses of $2\times2$ matrices:
\[\begin{pmatrix}
a & b \\
c & d \\
\end{pmatrix}^{-1}=\frac{1}{ad-bc}\begin{pmatrix}
d & -b \\
-c & a \\
\end{pmatrix} \text{, so long as } ad-bc\neq 0.
\] So we get:
\[
D = -\frac{1}{3}\begin{pmatrix}
-1 &-1\\
-1 &2 
\end{pmatrix}
\begin{pmatrix}
4 & 2\\
1 &3 
\end{pmatrix}
 \begin{pmatrix}
2 &1\\
1 &-1 
\end{pmatrix}
=\begin{pmatrix}
5 &0\\
0 & 2
\end{pmatrix}
\]

But this does not really give any intuition into why this happens. Let look at what happens when we apply this matrix $D = P^{-1}MP$ to a vector $v = \colvec{x \\ y}$. Notice that applying $P$ translates $v = \colvec{x \\ y}$ into $x\mathbf{v}_1+ y\mathbf{v}_2$.

\begin{eqnarray*}
P^{-1}MP
\colvec{x \\ y} &=& P^{-1}M \colvec{2x+y \\ x-y} \\
&=& P^{-1}M \left[\colvec{2x \\ x} + \colvec{y \\ -y}\right] \\
&=& P^{-1}\left[ x\, M \colvec{2 \\ 1} + y\,  M\colvec{1 \\ -1}\right]\\
&=& P^{-1}\left[ x\, M\mathbf{v}_1 + y\,   M\mathbf{v}_2\right]\\
\end{eqnarray*}

Remember that we know what $M$ does to $\mathbf{v}_1$ and $\mathbf{v}_2$, so we get
\begin{eqnarray*}
P^{-1}[ x\, M\mathbf{v}_1 + y\, M\mathbf{v}_2] &=& P^{-1}[ x \lambda_1\,  \mathbf{v}_1 + y \lambda_2 \,  \mathbf{v}_2] \\
&=& 5x\,   P^{-1}\mathbf{v}_1 + 2y \,   P^{-1} \mathbf{v}_2 \\
&=& 5x\,   \colvec{1 \\ 0} + 2y \,  \colvec{0 \\ 1} \\
&=& \colvec{5x \\ 2y}
\end{eqnarray*}
Notice that multiplying by $P^{-1}$ converts $\mathbf{v}_1$ and $\mathbf{v}_2$ back in to $ \colvec{1 \\ 0}$ and $ \colvec{0 \\ 1}$ respectively. This shows us why $D = P^{-1}MP$ should be the diagonal matrix:
\[ D = \begin{pmatrix}
\lambda_1 &\mc0\\
\mc0 & \lambda_2
\end{pmatrix}
=\begin{pmatrix}
5 &0\\
0 & 2
\end{pmatrix}
\]
} % Closing bracket for font

%\newpage
