
\subsection{\eigenTitle: Worked Example}

%%%Insert this to get the typewriter font so it looks like a real movie script
{\ttfamily
\fontdimen2\font=0.4em
\fontdimen3\font=0.2em
\fontdimen4\font=0.1em
\fontdimen7\font=0.1em
\hyphenchar\font=`\-


\hypertarget{scripts_eigenvalseigenvects_matrix}{Lets consider }
a linear transformation 
\[L:V\longrightarrow W\]
where a basis for $V$ is the pair of vectors $\{\to,\uparrow\}$
and a basis for $W$ is given by some other pair of vectors $\{\nearrow,\nwarrow\}$. (Don't be afraid that we are using arrows instead of latin letters to denote vectors!) To test your understanding, see if you know what $\dimension V$ and $\dimension W$ are. Now suppose that $L$ does the 
following to the basis vectors in $V$
\[
\to \ \stackrel{L}\mapsto\  a\nearrow + \ c\nwarrow =: L(\to)\, ,\qquad \uparrow \ \stackrel{L}\mapsto \ 
b\nearrow +\  d\nwarrow=: L(\uparrow)\, .
\]
Now arrange $L$ acting on the basis vectors in a row vector (this will be a row vector whose entries are vectors).
\[
\big(L(\to)\quad L(\uparrow)\big)
=
\big(a\nearrow + \ c\nwarrow\quad b\nearrow + \ d\nwarrow\big)\, .
\]
Now we rewrite the right hand side as a matrix acting from the right on the basis vectors in $W$:
\[
\big(L(\to)\quad L(\uparrow)\big)
=
\big(\nearrow\quad \nwarrow\big)
\begin{pmatrix}
a&b\\c&d
\end{pmatrix}\, .
\]
The matrix on the right is the matrix of $L$ with respect to this pair of bases.

We can also write what happens when $L$ acts on a general vector $v\in V$. Such a $v$ can be written
\[
v=x \to \ + \ y \uparrow\, .
\]
First we compute $L$ acting on this using linearity of $L$
\[
L(v) = L(x \to \ + \ y \uparrow)
\]
and then arrange this as a row vector (whose entries are vectors) times a column vector of numbers
\[
L(v) = \big(L(\to)\quad  L(\uparrow)\big)\begin{pmatrix}x\\y\end{pmatrix}\, .
\] 
Now we use our result above for the row vector$(L(\to)\ L(\uparrow))$
and obtain
\[
L(v) = \big(\nearrow\quad \nwarrow\big)
\begin{pmatrix}
a&b\\c&d
\end{pmatrix}
\begin{pmatrix}x\\y\end{pmatrix}\, .
\]

Finally, as a fun exercise, suppose that you want to make a change of basis in $W$ via
\[
\nearrow \ = \ \to + \uparrow\, \mbox{ and } \, \nwarrow\ =\  -\to + \uparrow\, .
\]
Can you compute what happens to the matrix of $L$?

} % Closing bracket for font

\newpage
