%%%%%%%%%%%%%%%%%% Latex %%%%%%%%%%%%%%%%%%%%%
\documentclass[12pt]{article}

\usepackage{amsmath}
\usepackage{shadow}
%\usepackage{bbold}
\usepackage{amssymb}
\usepackage{slashed}
%\usepackage{graphics}
\usepackage{graphicx}
%\usepackage{pstricks,pst-node,pst-tree}


%Sets
\def\A{{\mathbb{A}}}
\def\C{{\mathbb{C}}}
\def\F{{\cal F}}
\def\H{{\mathbb{H}}}
\def\O{{\cal O}}
\def\N{{\mathbb{N}}}
\def\Q{{\mathbb{Q}}}
\def\R{{\mathbb{R}}}
\def\Real{\mathbb R}
%\def\R{{\cal R}}
\def\Z{{\mathbb{Z}}}

%greek
\def\a{\alpha}
\def\b{\beta}
\def\de{\delta} 
\def\De{\Delta}
\def\ga{\gamma} 
\def\g{\gamma}
\def\si{\sigma} 
\def\Si{{\Sigma}}
\def\e{\epsilon}
\def\eps{\epsilon}
\def\La{\Lambda}
\def\r{\rho}
\def\Th{\Theta}
\def\th{\theta} 
\def\k{\kappa}
\def\l{\lambda}
\def\m{\mu}
\def\n{\nu}
\def\s{\sigma}
\def\t{\tau}
\def\z{\zeta}

%equation 
\def\be{\begin{equation}}
\def\ee{\end{equation}}
\def\bea{\begin{align}}
\def\eea{\end{align}} 
\def\nn{\nonumber}

%matricies
\def\ba{\left(\begin{array}{cc}}
\def\ea{\end{array}\right) }
\def\baa{\left(\begin{array}{ccc}}
\def\eaa{\end{array}\right) }
\def\baaa{\left(\begin{array}{cccc}}
\def\eaaa{\end{array}\right) }

%vectors
\def\v{\vec}
\def\bv{\left(\begin{array}{c}}
\def\ev{\end{array}\right) }


%quantum operators
%\def\d{{\bfseries d}}
\def\D{{\bfseries {\cal D}}}
\def\Del{\bm \delta}
\def\Delb{\bar{\bm \delta}}
\def\DIV{{\bfseries div}}
\def\G{{\bfseries g}}
\def\GRAD{{\bfseries grad}}
%\def\N{{\bfseries N}}
\def\ord{{\bfseries ord}}
\def\Pa{\bm \partial}
\def\Partial{\bm \partial}
\def\Pab{\bar{\bm \partial}}
\def\TR{{\bfseries tr}}
\def\dsl{\slashed{\partial}} 
\def\psl{\slashed{p}} 

%misc
\def\la{\langle}
\def\ra{\rangle}
\def\scirc{\!{\scriptstyle \circ}}
\def\pa{\partial}
\def\bra{\langle} \def\ket{\rangle}
\def\ie{{i.e., }}
\def\eg{{e.g.\ }}
\def\cl{\centerline}
\def\noi{\noindent}
\def\f{\frac}
\def\LR{\Leftrightarrow}
\def\d{\frac{d}{dx}}
\def\span{ \operatorname{span}}

\def\ni{\noindent}

%\newcommand{\comment}[1]{{\bfseries [#1]}}
%\newcommand{\ul}[1]{\underline{#1}}
%\newcommand{\bm}[1]{\mbox{\boldmath $ #1 $}}

%%%%%%%%%%%%%%%%%%%%%%%%%%%%%%%%%%%

\begin{document}

\thispagestyle{empty}
~
\vspace{-2cm}

\begin{center}
\vspace{-1.5cm}
{\Large{\bfseries  
Written Exercise part I\\
Vectors and Vector Spaces}
 }  \\[9mm]
% {\sc \small 22A, Summer 2014, \\
% Prof. Cherney}
\end{center}




\section{Instructions:} 
Invent a unique applications style exercise and response. 
Use the example provided below as an indicator of the mathematical ideas your example should cover, but your presentation should be self contained and original; it should not refer to this document. 

Your work must be typed. While it can be difficult to get started typing mathematics, the idea of taking years of college level math and being unable to type about it should strike you as absurd. Most word processing software has the capacity to insert matrices, subscripts, etc. Use of LaTeX is recommended. Consult the internet for getting started with LaTeX. 

Note that the example provided contains considerable detail. You should not expect to be able to sit down and immediately write an exercise and response with the same level of detail; you will need to go through your presentation and add detail several times. There is no length requirement for this assignment; a perfunctory job will be easy to spot. 

If your example involves complicated calculations (involving difficult fractions etc) then you may run your calculations on software and report the results in your work, but you might benefit from playing with the numbers until you obtain an easy calculation. 


\section{Purpose}

Feel out the boundaries of the applicability of the mathematical objects from linear algebra. If you find that your initial choice of exercise  makes difficult the process of utilizing such objects, then you are doing the right thing: getting a feel for when use of these objects is natural, and less than natural. 

Think about how the simple set structure of vector spaces and the idea of bases facilitate the organization of information using n-vectors and matrices. 
Also think about how one must compromise between 1) the desire to work with vectors spaces and linear functions so that calculations may be done is a systematic way and 2) the naturalness of modeling with these mathematical structures. 


\newpage

\section{Prerequisites}
Before you begin this project you should have a good understanding of 
\begin{enumerate}
\item functions as three things (domain, codomain, rule of correspondence) 
\item n-vectors and vectors that are not n-vectors
\item the two part characterization of vector spaces (closure under scalar multiplication and closure under addition)
\item vector spaces of the form $\mathbb{R}^S$ for finite sets $S$.
\item how choice of basis for a vector space facilitates notation of non-n-vectors as n-vectors. 
\item subspaces 
\end{enumerate}






%Suggest fields 
%Bio, statistics, 
%\\
%
%one simple eg and one detailed.\\
%
%List of ``The skills " was good. 
%\\

%Digression on models: Compromise between perfection of description and having the math structures that facilitate calculation. \\\



\newpage
\section*{Example}
\begin{enumerate}
\item[] 
\quad A door factory can buy supplies in two kinds of packages, $f$ and $g$. 
The package $f$ contains $3$ slabs of wood, $4$ fasteners, and $6$ brackets. 
The package $g$ contains $5$ fasteners, $3$ brackets, and $7$ slabs of wood. 

\begin{enumerate}
\item Model $f$ and $g$ as functions with codomain $\mathbb{R}$;
explicitly give the domain codomain, and rule of correspondence of these functions. 

\item Identify the range of $f$ and the range of $g$ and verify that they are subsets of the codomains for these functions. 

\item Interpret $\frac1\pi f $. Identify it's codomain of in terms of the codomain of $f$. 


\item 
Model the set of ways that supplies can be purchased as a vector space. 

\item Discuss the shortcomings of the model; what physical interpretation can be given to the properties i) closure under addition and ii) closure under scalar multiplication.

\item Identify your model vector space as a (proper) subspace of a vector space.

\item
Choose a basis for the vector larger vector space and use that basis to write i) a particular vector in the larger vector space that is not in the subspace and
ii) an arbitrary supply order as a n-vector.
\item
Choose a  basis for the subspace space and use that basis to write i) a particular and ii) an arbitrary supply order as a n-vector.
\end{enumerate}

{\bfseries Response:} 
\item[]
\begin{enumerate}

\item 

%The packages of supplies are elements of the vector space 
%\[ V:=\{  F: \{ slabs, fasters,brackets \} \to \mathbb{R}\}.\]
%In particular, 
The supply package $f$ can be modeled as the function that assigns to 
each type of supply item (slabs, fasters, and brackets) the number of items that come in that package. That is 
%\[ f: \{ \text{ slabs, fasters, brackets } \} \to \{ 3,4,6,7,5\} \]
%with rule of correspondence explicitly given by
%\[  f(\text{slabs})=3,f(\text{fasters})=4, f(\text{brackets})= 6.\]
%Similarly
%\[ g: \{ \text{ slabs, fasters, brackets } \} \to  \{ 3,4,6,7,5\} \]
%with rule of correspondence
%\begin{gather*}  g(slabs)=7,g(fasters)=5, g(brackets)= 3. \\
%\{3,4,6\} \subseteq  \{ 3,4,6,7,5\}\end{gather*}
%and
%\[ \{7,5,3\} \subseteq  \{ 3,4,6,7,5\}.\]
\[ f: \{ \text{ slabs, fasters, brackets } \} \to \mathbb{R}\]
with rule of correspondence explicitly given by
\[  f(\text{slabs})=3,f(\text{fasters})=4, f(\text{brackets})= 6.\]
Similarly
\[ g: \{ \text{ slabs, fasters, brackets } \} \to  \mathbb{R}\]
with rule of correspondence
\[  g(\text{slabs})=7,g(\text{fasters})=5, g(\text{brackets})= 3. \]

\item The range of $f$ is $\{ f(x)~|~x\in\dom(f)\}= \{3,4,6\} \}$
while the range of $g$ is $\{ g(a)~|~a\in \dom(g)\}= \{7,5,3\}\} $. These are both subsets of the codomain $\mathbb{R}$; 
\[ \{3,4,6\} \subseteq  \R\]
and
\[ \{7,5,3\} \subseteq  \R.\]

\item By the usual scalar multiplication of functions 
\[ \frac1\pi f: \{ \text{ slabs, fasters, brackets } \} \to \mathbb{R}\]
with rule of correspondence explicitly given by
\[  \frac1\pi f(\text{slabs})=\frac1\pi 3~,~~f(\text{fasters})=\frac1\pi 4~,~~ f(\text{brackets})= \frac1\pi 6.\] That is 
$\frac1\pi f$ has the same domain as $f$. This is the case because the codomain of $f$ is a set that is closed under scalar multiplication; if the codomain of $f$ had been chosen to be $\{1,2,3,4,5,6,7\}$ then in order to give $\frac1\pi f$ meaning the codomain of $\frac1\pi f$ would need to be different from that of $f$. 

The function $\frac1\pi f$ has no interpretation in terms of supply packages for the door shop; a fraction of items such as a hinge might in some cases have an interpretation (such as when two halves are held together my a pin) but this fraction is of a supply package is a purely mathematical entity with no obvious interpretation. The same is true for most multiples of $f$.


\item 
If a door factory can only buy supplies in two kinds of packages, $f$ and $g$ then a natural model of ways in which supplies can be purchased that builds off the functions $f$ and $g$ defined above is the set of sums of integer multiples of the supply packages:
\[P:= \{ mf+ng~\vert m,n\in \N \}. \]
This is natural since, presumably, only integer quantities of the sully packages can be purchased.
However, $P$ is not a vector space; 
a vector space is a set that is closed under scalar multiplication and under addition. 
The set $P$ is  closed under addition-- if $m_1f+n_1g,~m_2f+n_2g \in P$ then their sum  
\[
(m_1f+n_1g)+(m_2f+n_2g)=  
(m_1+m_2)f+(n_1+n_2)g  
\in P.\]
The set $P$ is not closed under scalar multiplication since $f$ is in $P$ but $\frac12 f$ is not. 

For this reason, the set 
\[V:= \{ af+bg~\vert ~a,b\in \mathbb{R}\}\]
is needed to model the ways supplies can be purchased as a vector space, even though most of the elements of this set have no interpretation.

%%%%%%%%%%%%%%%%%%%%%%%%%
\item 
i) The model $V$ has the property of closure under addition. This property can be roughly interpreted as follows: any two orders can be combined into one order. This is not always realistic as sometimes suppliers put restrictions on the size of orders. 
\\
ii) The model $V$ has the property of closure under scalar multiplication. 
This property can be interpreted as follows: any ordered can be doubled, tripled, halved, $\pi$-thed, etc. This is not realistic in that, for example, halving an order with an odd number of  fasteners has no obvious meaning. Nor do  multiples of orders by irrational numbers. 


\item 
The model vector space is a subspace of space of all functions with a particular domain and a codomain:
\[ W:=\{  F: \{ slabs, fasters,brackets \} \to \mathbb{R}\}.\]
In particular, 
%\begin{gather*} f: \{ \text{ slabs, fasters, brackets } \} \to \mathbb{R}\\
%f(\text{slabs})=3,f(\text{fasters})=4, f(\text{brackets})= 6\end{gather*}
%and
%\begin{gather*} g: \{ \text{ slabs, fasters, brackets } \} \to \mathbb{R}\\
%g(slabs)=7,g(fasters)=5, g(brackets)= 3 \end{gather*}
$f$ and $g$ and any sum of multiples of them are in $W$.
However, some elements of $W$, such as 
\begin{gather*}h:\{ slabs, fasters,brackets \} \to \mathbb{R}\\
 h(\text{slabs})=1,f(\text{fasters})=0, f(\text{brackets})= 0\end{gather*}
are not. This claim, that $h\notin V$, can be backed up with a brief calculation after a basis for $V$ or $W$ is chosen. 

\item 
A choice of basis for $W$ will induce n-vector notation for vectors in $V$ and $W$.
Let 
\[B:=( e_s,e_f,e_b)\] 
be a basis for $V$ defined by 
\begin{gather*}
e_s(x)=\left\{ \begin{array}{l} 1 \text{~if~} x=\text{slabs}\\ 0 \text{~else~} \end{array} \right.\, , ~
e_f(x)=\left\{ \begin{array}{l} 1 \text{~if~} x=\text{fasteners}\\ 0 \text{~else~} \end{array} \right.\, , \\
e_b(x)=\left\{ \begin{array}{l} 1 \text{~if~} x=\text{brackets }\\ 0 \text{~else~} \end{array} \right. .
\end{gather*}
i) The function $h\in W$ with the property $h\notin V$ can be written in this basis as follows.
\[h=1e_s+0e_f+0e_b \bv 1\\0\\0 \ev_B.\]
We can now verify that $h\notin V$. Since
\[f=3e_S+4e_f+6e_b=\bv 3\\4\\6 \ev~,~~g=7e_s+5e_f+3e_b = \bv 7\\5\\3 \ev_B\]
if $h$ is  a sum of multiples of $h$ and $g$ then there are number $a,b\in \mathbb{R}$ such that
\begin{gather*}af+bg=h \LR a\bv 3\\4\\6 \ev_B+b \bv 7\\5\\3 \ev_B= \bv 1\\0\\0 \ev_B\\
\LR \left[\ba 3&7 \\4&5\\6&3 \ev \bv a\\b \ev\right]_B= \bv 1\\0\\0 \ev_B\end{gather*}
This equation has no solution since 
\[
\left( \begin{array}{cc|c}
3&7&1\\4&5&0\\6&3&0
\end{array}\right)
\sim
\left( \begin{array}{cc|c}
12&28&4\\
12&15&0\\
0&-11&0
\end{array}\right)
\sim
\left( \begin{array}{cc|c}
12&28&4\\
0&-13&-4\\
0&1&0
\end{array}\right)
\sim
\left(\begin{array}{cc|c}
12&0&4\\
0&0&-4\\
0&1&0
\end{array}\right)
\]
encodes an inconsistent system of equations. \\

ii) An arbitrary supply package, meaning an arbitrary element of $V$, is of the form $af+bg$ with $a,b\in \mathbb{R}$. Since
\[f=3e_S+4e_f+6e_b=\bv 3\\4\\6 \ev~,~~g=7e_s+5e_f+3e_b = \bv 7\\5\\3 \ev_B\]
we can write 
\[
af+bg= a\bv 3\\4\\6 \ev_B+b \bv 7\\5\\3 \ev_B= \left[\ba 3&7 \\4&5\\6&3 \ev \bv a\\b \ev\right]_B.
\]




\item The ordered set $A=(f,g)$ forms a basis for $V$. 
\\
i) Thus $f=1f+0g=\bv 1\\0 \ev_A$.
\\
ii) With this choice of basis, an arbitrary element of $V$ can be written as  $af+bg=\bv a\\b \ev_A.$ 
\end{enumerate}
\end{enumerate}















\end{document}