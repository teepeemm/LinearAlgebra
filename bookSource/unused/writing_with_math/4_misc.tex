%%%%%%%%%%%%%%%%%% Latex %%%%%%%%%%%%%%%%%%%%%
\documentclass[12pt]{article}

\usepackage{amsmath}
\usepackage{shadow}
\usepackage{dingbat}
%\usepackage{wasysym}
%\usepackage{bbold}
\usepackage{amssymb}
\usepackage{slashed}
%\usepackage{graphics}
\usepackage{graphicx}
%\usepackage{pstricks,pst-node,pst-tree}


% equations
\def\TT{{\mathrm TT}}
\def\N{{\cal N}}
\def\GG{{\cal G\hspace{-2.3mm}G}}
\def\Gg{\mathbb G}
\def\E{{\cal E}}
\def\B{{\cal B}}
\def\lrd{\stackrel{\leftrightarrow}{\partial}}

\def\chib{\overline\chi}
\def\psib{\overline\psi\hspace{-2.6mm}\phantom{\psi}}
\def\tid{\hspace{.3mm}}
\def\ve{\varepsilon}
\def\D{{\cal D}}
\def\M{{\cal M}}
\def\J{{\cal J}}
%greek
\def\a{\alpha}
\def\b{\beta}
\def\de{\delta} 
\def\De{\Delta}
\def\eps{\epsilon}
\def\r{\rho}
\def\k{\kappa}
\def\l{\lambda}
\def\m{\mu}
\def\n{\nu}
\def\s{\sigma}
\def\t{\tau}
\def\z{\zeta}


%phantoms
\def\ph#1{\phantom{#1}}
\def\ag#1{\gamma({\scriptstyle #1})}
\def\Sl#1{\slashed{#1}}
%\def\Sl#1{#1\hspace{.1mm}\!\!\!\!/\,}
\def\sl#1{\slashed{#1}}
\def\wh#1{\widehat{#1}}
\def\wt#1{\widetilde{#1}}
\def\mt{{\tilde \mu}}
\def\nt{{\tilde \nu}}
\def\rt{{\tilde \rho}}
\def\st{{\tilde \sigma}}
\def\ol#1{\overline{#1}} 
\def\sss{\scriptscriptstyle}
\def\ts{\textstyle}
\def\d{{\bfseries d}}
\def\la{\langle}
\def\ra{\rangle}
\def\e{\epsilon}
\def\scirc{\!{\scriptstyle \circ}}
\def\psitb{\overline{\widetilde\psi}}
\def\vphi{\varphi}
\def\psid{\psi^\dagger}

%Sets
\def\F{{\cal F}}
\def\O{{\cal O}}
\def\Z{{\mathbb{Z}}}
\def\Q{{\mathbb{Q}}}
\def\A{{\mathbb{A}}}
\def\R{{\mathbb{R}}}
\def\C{{\mathbb{C}}}
\def\H{{\mathbb{H}}}

%\matricies
\def\ba{\left(\begin{array}{cc}}
\def\ea{\end{array}\right) }
\def\be{\begin{equation}}
\def\ee{\end{equation}}
\def\beq{\begin{equation}}
\def\eeq{\end{equation}}
\def\bea{\begin{align}}
\def\eea{\end{align}} 
\def\beqa{\begin{equation}\begin{array}{l}}
\def\eeqa{\end{array}\end{equation}}

%--------------------------------------------
% symbols
%warning \be is for equations now
\def\ga{\gamma} 
\def\G{{\it\Gamma}} 
\def\g{\gamma}
\def\veps{\varepsilon}  
\def\L{{\it\Lambda}}
\def\Pit{{\it\Pi}}
\def\Psit{{\it\Psi}}
\def\si{\sigma} \def\Si{{\it\Sigma}}
\def\th{\theta} \def\vth{\vartheta} \def\Th{\Theta}
\def\w{\omega} \def\W{\Omega} \def\hw{\hat{\omega}}
\def\vfi{\varphi}\def\vphi{\varphi}
\def\bra{\langle} \def\ket{\rangle}
\def\dd{{\mathrm d}}
\def\pa{\partial}
\def\vrho{\varrho}

\def\ie{{i.e., }}
\def\eg{{e.g.\ }}
\def\LR{\Leftrightarrow} 
\def\cl{\centerline}

\def\pa{\partial}
\def\rarr{\rightarrow}
\def\nn{\nonumber}

\def\psibar{\overline{\psi}}
\def\Gbar{\overline{G}}

\def\psidag{\psi^\dagger}
\def\Gdag{G^\dagger}

\def\N{{\bfseries N}\,}
\def\TR{{\bfseries tr}\,}
\def\G{{\bfseries g}\,}
\def\DIV{{\bfseries div}\,}
\def\GRAD{{\bfseries grad}\,}
\def\ord{{\bfseries ord}\,}

\def\Nt{{\cal N}}
\def\Tt{{\cal T}}
\def\c{{\hspace{.3mm}\bfseries c\hspace{.3mm}}}
\def\Real{\mathbb R}

\def\noi{\noindent}

\def\thup{\rightthumbsup}
\def\thdn{\rightthumbsdown}

%\def\checkmark{\tikz\fill[scale=0.4](0,.35) -- (.25,0) -- (1,.7) -- (.25,.15) -- cycle;}

%%%%%%%%%%%%%%%%%%%%%%%%%%%%%%%%%%%%%%%%%%%%

\begin{document}

\thispagestyle{empty}
~
\vspace{-3.9cm}
\begin{center}
{
\Large
 {\bfseries  Writing With Mathematics: Mood}  \\[10mm]
 {  
 \vspace{-.9cm} 
%MAT 003-03, SMC Fall, 2013\\
Cherney}\\[1mm]
%{\small Office hours: Tues \& Thurs 2-4pm, Galileo 103A\\dmc13@stmarys-ca.edu}
\vspace{3.5mm}
}

\end{center}





\section{Miscellaneous Common mistakes}

\subsection{Lies!}
e.g.  x+2=5=3.\\
e.g. 
\[3^{2x} -2(3^x) = -1 = (3^x)^2 -2(3^x) + 1 = 0 =\]
\[(3^x -1)^2 = 0 = 3^x = 1 = x = 0.\]

\noi e.g. 
\[3^{2x} -2(3^x) = -1\implies (3^x)^2 -2(3^x) + 1 = 0 \implies\]
\[(3^x -1)^2 = 0 \implies 3^x = 1 \implies x = 0.\]

\noi When going from one equation to an equivalent equation, the convention is to
just place each equation on a separate line.

\[32x -2(3x) = -1\]
\[(3^x)^2 -2(3^x) + 1 = 0\]
\[(3x -1)^2 = 0\]
\[3^x = 1\]
\[x = 0.\]

\subsection{By Definition}
If any logic is required, a result is not by definition. 

\subsection{Begging the question}
Why the sky blue? Because the sky is blue.\\
Not just for questions: "Show that the three lines form a triangle." "The three lines form a triangle."

\subsection{Pronoun ambiguity and vagueness}
To find the solutions to the equation $2x=4$ divide it by it and it is 2.

\subsection{Comparative terms without a pair to compare}
e.g. The value of $f$ at $a+1$ is bigger.\\

\noi The difference between the graph of a function and the domain of a function is the domain is the set of all objects for which a function assigns another object.

\subsection{\bfseries The students' mindset}
Giving sufficient context should not mean explaining to your reader that some jerk of an instructor told you you have to do something. You would not begin an essay for an english class with ``My instructions are to write 500 words about the book Of Mice and Men, so I'm writing these words." 
This kind of writing would be inappropriate if your employer asked you to write a report, and is 
Likewise, you should not write like this for your math homework.

Example:\\
\begin{quote}
\shabox{"We are given that three beans are in a jar. We are told that four more beans are added. We are asked to find the number of beans in the jar. The first step is to write $3+4$. Next add 3 and 4. Next write down 7. "}
\end{quote}



This sounds like what a tutor might say to a student that is failing arithmetic; it is a breakdown of concepts and process that reads like 
giving a man a fish (if focuses on what to write to get credit for this problem and for learning nothing.)


The symbol $+$ means that we have to add. \\
$\int_0^1$ means to anti-differentiate and evaluate between o and 1. \\
``The method of right sums asks..."\\




\subsection{Conflating vocabulary}
Order word matters and word choose counts! If you walk away from a math class unable to communicate about it you are getting very little from the class.  \\

e.g. 
Consider the curve $y=x^2$.\\
vs\\
Consider the solution set to the algebraic equation $y=x^2$.

\noi e.g. \\
The equation of the function...\\

\noi The solution to the function.\\

\noi The equation 3x+1...\\

\noi Set the equations equal to each other... \\

\noi If $f(x)=x+1$ then the function $f(1)=2$.

\noi If the domain of $f$ is $[0,3)$ then the function $f(4)$ is not defined.\\

\noi When one derives $x^2$ one obtains $2x$.

\subsection{Categorical error}

The data set is linear. \\

The graph of the equation.\\

The solution of the function. \\

The domain of the equation. \\


\subsection{Less than sophisticated wording}
%``Functions are equations are solutions are functions...."\\ In Conflating, above
``We are given ..."\\
``plug it in"\\
``Solve it out"\\
``We must"\\
``I would" \\
%``First do this, then do that"\\
``The answer is..."\\
Indicating canceling with slashes.\\
Chimera ``The value of f at $3= 5$."\\
Future subjunctive? ``1+x+1 would be x+2." \\
``A solution to a function is when you substitute..." A solution is a time? \\ 


\subsection{Idiosyncratic Notation }
$\frac{d}{dx}= 2x.$ (Has this been independently invented dozens of times? )\\
$ss= (HT),(TH)$.\\
Your syllabus says that one of the goals is for you to learn standard mathematical notations. 

\subsection{Inappropriate Mood}
Students often ask 

The declarative mood is for stating facts.\\
e.g.  If $f(x)=x^2$ for all numbers x then $f(2)=2^2=4$. \\
The imperative mood if for issuing commands.\\
e.g.  First write $2^2$ and then rewrite this as $2\cdot 2$. Last, multiply $2$ and $2$.\\
The subjunctive mood is for discussing hypothetical situations.\\
e.g. If $f(x)$ were $x^2$ then $f(2)$ would be 4.\\[.2cm]

Unless you intend to give your audience commands, do not use the imperative mood. The subjunctive mood is not needed for communicating facts about mathematics. 
\\


Use the declarative mood when writing about math.


\subsection{Process vs fact}
Next...\\
vs\\
by ..., ... .\\
By ... ... becomes ...

\subsection{It already IS a number}
``If we calculated $\sqrt{3.1}$ we would have the answer."\\
A solution to $x+y=2$ would be $(1,1)$.

\subsection{If there was no question, what do you mean by answer?}
e.g. Instructions: Calculate $1+2$.\\
Response: The answer is 3.

\subsection{Why use the formula?}
``In order to calculate the interest earned we use the formula $F=P(1+rt)$."
vs.\\
``Inflation is best modeled as compound interest compounded annually. The relationship between the Future value FV and present value PV of an account earning compounded interest at an annual interest rate  5\% is ..." 
%\subsection{Function do not have solutions}

\section*{What is it to be highly literate?}
Academics who study literacy in the ancient world have found that, in ancient times a person was considered literate if they could write their name even if they could not read a single word. Things have changed. These days, in our culture, if one can not read a novel then one is considered illiterate. 

Much, much more is expected of a college graduate;  a college degree should indicate that the degree holder has developed and demonstrated the ability to communicate about compacted, nuanced topics. To develop and communicate their own ideas. 

This kind of communication requires one to think hard about one's audience. The difficulty in verbalizing thoughts is not in putting your ideas into words such that you can read what you wrote, but in putting your ideas into words that other people can not misunderstand. 

Literacy is not just about the ability to read. It is the ability to read and write. If you can't read the text book then should you get a passing grade? 


The trifecta: Read, write, and speak about calculation.


\section{Want more examples? It is probably best to start writing}
I was once cleaning a house. Someone else's house. A person asked if they could help. 

"Sure," I said, "can you mop the floor of the kitchen?" 

"Ok. How do I do that?"

"The mop is right next to you."

"Do you have a bucket or something?"
"Where should I start?" "Where is the moping fluid?" "How do I not walk where I mopped?"

You have ceased to be helpful. You have taken more time than you saved. You have made work for me, not less. You are fired. 

It isn't about learning to do a job by watching a person do the whole job and then doing exactly what they did. Get started, use your own methods and ideas, and get as much done as you can. When you ask for help make it clear exactly where you need help. I'm not going to mop the floor for you. 




%%%voice?
%Using the definition of a function the value of f(1) would be 2. ?!?!?




%\section*{e.g.s}
%How should one measure accuracy of a model? \\
%One should measure accuracy by plugging in the numbers and calculating them. 

%Questions for me:\\
%Is there a voice with no person? "The equation Q=kt describes the relationship between the force Q of a spring attached to a block which is displaced from the ... in terms of a constant k which .."\\
%
%
%






\end{document}

