%%%%%%%%%%%%%%%%%% Latex %%%%%%%%%%%%%%%%%%%%%
\documentclass[12pt]{article}

\usepackage{amsmath}
\usepackage{shadow}
\usepackage{dingbat}
%\usepackage{wasysym}
%\usepackage{bbold}
\usepackage{amssymb}
\usepackage{slashed}
%\usepackage{graphics}
\usepackage{graphicx}
%\usepackage{pstricks,pst-node,pst-tree}


% equations
\def\TT{{\mathrm TT}}
\def\N{{\cal N}}
\def\GG{{\cal G\hspace{-2.3mm}G}}
\def\Gg{\mathbb G}
\def\E{{\cal E}}
\def\B{{\cal B}}
\def\lrd{\stackrel{\leftrightarrow}{\partial}}

\def\chib{\overline\chi}
\def\psib{\overline\psi\hspace{-2.6mm}\phantom{\psi}}
\def\tid{\hspace{.3mm}}
\def\ve{\varepsilon}
\def\D{{\cal D}}
\def\M{{\cal M}}
\def\J{{\cal J}}
%greek
\def\a{\alpha}
\def\b{\beta}
\def\de{\delta} 
\def\De{\Delta}
\def\eps{\epsilon}
\def\r{\rho}
\def\k{\kappa}
\def\l{\lambda}
\def\m{\mu}
\def\n{\nu}
\def\s{\sigma}
\def\t{\tau}
\def\z{\zeta}


%phantoms
\def\ph#1{\phantom{#1}}
\def\ag#1{\gamma({\scriptstyle #1})}
\def\Sl#1{\slashed{#1}}
%\def\Sl#1{#1\hspace{.1mm}\!\!\!\!/\,}
\def\sl#1{\slashed{#1}}
\def\wh#1{\widehat{#1}}
\def\wt#1{\widetilde{#1}}
\def\mt{{\tilde \mu}}
\def\nt{{\tilde \nu}}
\def\rt{{\tilde \rho}}
\def\st{{\tilde \sigma}}
\def\ol#1{\overline{#1}} 
\def\sss{\scriptscriptstyle}
\def\ts{\textstyle}
\def\d{{\bfseries d}}
\def\la{\langle}
\def\ra{\rangle}
\def\e{\epsilon}
\def\scirc{\!{\scriptstyle \circ}}
\def\psitb{\overline{\widetilde\psi}}
\def\vphi{\varphi}
\def\psid{\psi^\dagger}

%Sets
\def\F{{\cal F}}
\def\O{{\cal O}}
\def\Z{{\mathbb{Z}}}
\def\Q{{\mathbb{Q}}}
\def\A{{\mathbb{A}}}
\def\R{{\mathbb{R}}}
\def\C{{\mathbb{C}}}
\def\H{{\mathbb{H}}}

%\matricies
\def\ba{\left(\begin{array}{cc}}
\def\ea{\end{array}\right) }
\def\be{\begin{equation}}
\def\ee{\end{equation}}
\def\beq{\begin{equation}}
\def\eeq{\end{equation}}
\def\bea{\begin{align}}
\def\eea{\end{align}} 
\def\beqa{\begin{equation}\begin{array}{l}}
\def\eeqa{\end{array}\end{equation}}

%--------------------------------------------
% symbols
%warning \be is for equations now
\def\ga{\gamma} 
\def\G{{\it\Gamma}} 
\def\g{\gamma}
\def\veps{\varepsilon}  
\def\L{{\it\Lambda}}
\def\Pit{{\it\Pi}}
\def\Psit{{\it\Psi}}
\def\si{\sigma} \def\Si{{\it\Sigma}}
\def\th{\theta} \def\vth{\vartheta} \def\Th{\Theta}
\def\w{\omega} \def\W{\Omega} \def\hw{\hat{\omega}}
\def\vfi{\varphi}\def\vphi{\varphi}
\def\bra{\langle} \def\ket{\rangle}
\def\dd{{\mathrm d}}
\def\pa{\partial}
\def\vrho{\varrho}

\def\ie{{i.e., }}
\def\eg{{e.g.\ }}
\def\LR{\Leftrightarrow} 
\def\cl{\centerline}

\def\pa{\partial}
\def\rarr{\rightarrow}
\def\nn{\nonumber}

\def\psibar{\overline{\psi}}
\def\Gbar{\overline{G}}

\def\psidag{\psi^\dagger}
\def\Gdag{G^\dagger}

\def\N{{\bfseries N}\,}
\def\TR{{\bfseries tr}\,}
\def\G{{\bfseries g}\,}
\def\DIV{{\bfseries div}\,}
\def\GRAD{{\bfseries grad}\,}
\def\ord{{\bfseries ord}\,}

\def\Nt{{\cal N}}
\def\Tt{{\cal T}}
\def\c{{\hspace{.3mm}\bfseries c\hspace{.3mm}}}
\def\Real{\mathbb R}

\def\noi{\noindent}

\def\thup{\rightthumbsup}
\def\thdn{\rightthumbsdown}

%\def\checkmark{\tikz\fill[scale=0.4](0,.35) -- (.25,0) -- (1,.7) -- (.25,.15) -- cycle;}

%%%%%%%%%%%%%%%%%%%%%%%%%%%%%%%%%%%%%%%%%%%%

\begin{document}

\thispagestyle{empty}
~
\vspace{-3.9cm}
\begin{center}
{
\Large
 {\bfseries  Writing With Mathematics: \\
 Format, and Mood}  \\[10mm]
 {  
 \vspace{-.9cm} 
%MAT 003-03, SMC Fall, 2013\\
Cherney}\\[1mm]
%{\small Office hours: Tues \& Thurs 2-4pm, Galileo 103A\\dmc13@stmarys-ca.edu}
\vspace{3.5mm}
}

\end{center}



\section{Formatting }

The shorthand nature of mathematical notation allows formulation of extremely complicated statements. Various conventions for formatting have come about to help readers digest such statements.

\subsection{Center for focus}
Although equations can be clauses in larger sentences, for the sake of readability and emphasis, one ought to put important equations on separate lines, centered. The following is an example of such writing. 
\begin{quote}
\shabox{
The total revenue, $R$, made from selling widgets is given
by the equation\\
\phantom{The total revenue, $R$,  made } $R = pq,$\\
where $p$ is the price at which each widget is sold and $q$ is
the number of widgets sold. Based on past experience,
we know that when widgets are priced at $15$ each, $2000$
widgets will be sold. We also know that for every dollar
increase in price, 150 fewer widgets are sold. Hence, if
the price is increased by $x$ dollars, then the revenue is\\
\phantom{sold. Hence, if
the pri} $R = (15 +x)(2000 -150x)$\\[.2cm]
\phantom{sold. Hence, if
the pri}$= -150x^2 -250x + 30,000.$
}\thup
\end{quote}


\subsection{Carriage return for readability}

Not doing so  forces the readers eyes to move around the page in unfamiliar ways. The following is a slight offense.
\begin{quote}
\shabox{%
The system of equations \tagpdfsetup{table/tagging=presentation}%
\begin{tabular}{ccc}
$x+y=1$& is equivalent to & \\ 
$x-y=1$& the system    &  $x+y=1$\\
&of equations & $2x=2$
\end{tabular}
.}\thdn
\end{quote}

When one sees this kind of writing one sometimes worries that one is not reading things in the right order. Clear things up by respecting the carriage return conventions.
\begin{quote}
\shabox{ 
The system of equations 
\[x+y=1\]
\[x-y=1\]
is equivalent to the system  
\[x+y=1\]
\[2x=2.\] 
}\thup
\end{quote}


\subsection{Placement of symbols in carriage returns}
Note that, conventionally, equal signs, implies signs, etc go on the next line. The following is a violation of that convention.
\begin{center}
\shabox{ 
$\begin{array}{cc}
R =& (15 +x)(2000 -150x)=  \\[.2cm] 
&-150x^2 -250x + 30,000.
\end{array}
$} \thdn
\end{center}
The following is in agreement with that convention.
\begin{center}
\shabox{ 
$
\begin{array}{cc}
&2y=12x+24\\[.2cm] 
\implies&  y=6x+12.
\end{array}
$}
\thup
\end{center}


If a single algebraic expression is too long for a line, 
the operation connecting two lines goes on the latter line.
\begin{center}
\shabox{ 
$\begin{array}{cc}
(A+B)^3=\!\!\!\!\!\!\!&A^3 + AAB+ABA+BAA	\\[.2cm] 
&+ABB+BAB+BBA+BBB.
\end{array}
$} \thup
\end{center}





\section{ Mood}
Three moods dominate the english language. \\

\noi The {\bfseries imperative mood} if for issuing commands.\\[.2cm]
\shabox{First write $2^2$ and then rewrite this as $2\cdot 2$. Last, multiply $2$ and $2$.}\thdn \\
This mood is appropriate for communicating  algorithms (sequences of steps used to cary out tasks.) But be aware that  describing an algorithm and describing why an algorithm performs its intended task are very different things. For a math class, you need to describe why algorithms work; you need to state facts not instructions.\\

%Unless you intend to give your audience commands, do not use the imperative mood. \\

\noi The {\bfseries subjunctive mood} is for discussing hypothetical situations.\\[.2cm]
\shabox{If $f(x)$ were $x^2$ then $f(2)$ would be 4.}\thdn \\[.2cm]
Consideration of hypotheticals  is high level of thinking that requires considerable effort; there is no need to force your reader into this mode of thinking to communicating facts about mathematics.  
The phrases ``were" and ``would be" in above example should both be replaced with ``is" to put this statement in the declarative mood.\\

\noi The {\bfseries declarative mood} is for stating facts.\\[.2cm]
\shabox{If $f(x)=x^2$ for all numbers $x$ then $f(2)=2^2=4$. } \thup\\

Use the declarative mood when writing {\it about} mathematics; the logic side of mathematics deals with relationships between facts.




%%%%%%%%%%%%%%%%%%%

\subsection{Claim}
Clarify to your reader the reason for their reading before they read it by informing them that you are making a claim. % that some statement is true 
A claim is a statement; it is either true or false. 
You reader will read on to see how you argue that your claim is a true statement. 


%\begin{quote}
\shabox{I claim that for any natural number $k$, 
the sum of the first $k$ odd natural numbers is $k^2$. 
In particular, the sum of the first $2$ natural numbers is \\
$1+3=4=2^2$\\
and the sum of the fist $3$ odd natural numbers is \\
$1+3+5=9=3^2.$\\
In general, if the statement is true for some particular integer $k$ so that \\
$\sum\limits_{i=1}^k (2i-1)=k^2$\\
then \\
$\sum\limits_{i=1}^{k+1}(2i-1)=
\sum\limits_{i=1}^k (2i-1)+ [2(k+1)-1]=k^2+2k+1=(k+1)^2$\\
so the statement is true for the particular natural number $k+1$ also. Then, by the principle of induction, for any natural number $k$, 
the sum of the first $k$ odd natural numbers is $k^2$.
}\thup
%\end{quote}


\subsection{Algorithms vs Flows of Logic: Process vs fact}
The following is a common form of response from a novice to writing with math. 

\begin{quote}
\shabox{
Why is it that the sum of two odd numbers is even? \\[.3cm]
First, pick two odd numbers $m$ and $n$. \\
Next, find two numbers $a$ and $b$ such that $m=2a+1$ and $n=2b+1$.\\
Next, add $m$ and $n$.\\
Next, write the result $2a+2b+2=2(a+b+1)$. 
}\thdn
\end{quote}

This string of imperatives reads like a set of instructions given to a computer. If your audience is human, a flow of logic is more appropriate.

\begin{quote}
\shabox{
Why is it that the sum of two odd numbers is even? \\[.3cm]
If $m$ and $n$ are odd numbers then there exist integers $a$ and $b$ such that $m=2a+1$ and $n=2b+1$. Therefore 
\[m+n=2a+2b+2=2(a+b+1)\]
is a multiple of 2, and thus even. 
}\thup
\end{quote}


%by ..., ... .\\
%
%By ... ... becomes ...




%This sounds like what a tutor might say to a student that is failing arithmetic; it is a breakdown of concepts and process that reads like 
%giving a man a fish (if focuses on what to write to get credit for this problem and for learning nothing.)
%
%
%The symbol $+$ means that we have to add. \\
%$\int_0^1$ means to anti-differentiate and evaluate between o and 1. \\
%``The method of right sums asks..."\\



%\section{\bfseries The students' mindset}
%%Giving sufficient context should not mean explaining to your reader that some jerk of an instructor told you you have to do something. 
%The goal of this series of handouts is to develop your ability to write about technical matters in a way transferable to life beyond the classroom.  
%You have been trained elsewhere to not begin an essay for an english class with ``My instructions are to write 500 words about the book Of Mice and Men, so I'm writing these words." 
%This kind of writing would be inappropriate if your employer asked you to write a report, and likewise, you should not write like this for your math homework. 
%
%
%
%
%Example:\\
%\begin{quote}
%\shabox{"We are given that three beans are in a jar. We are told that four more beans are added. We are asked to find the number of beans in the jar. The first step is to write $3+4$. Next add 3 and 4. Next write down 7. "}
%\end{quote}
%








\section{Passive Voice}
%actions are carried out by the subject
%In the {\bfseries active voice} verbs are carried out by subjects.
%\begin{quote}
%\shabox{
%I added 3 to both sides of the equation $x-3=4$ to obtain $x=1$.
%}
%\begin{quote}
%
%In the passive voice no 
%\begin{quote}
%\shabox{
%a}
%\begin{quote}
%
Last century, there was  a movement toward eliminating the active voice from scientific writing. There has been pushback as the resultant writing was dry, boring,  pompous, and just plain bad. I do not care to participate in the ongoing debate. 


%\subsection{It already IS a number}
%``If we calculated $\sqrt{3.1}$ we would have the answer."\\
%A solution to $x+y=2$ would be $(1,1)$.

%\subsection{If there was no question, what do you mean by answer?}
%e.g. Instructions: Calculate $1+2$.\\
%Response: The answer is 3.


%\section{Want more examples? It is probably best to start writing}
%I was once cleaning a house. Someone else's house. A person asked if they could help. 
%
%"Sure," I said, "can you mop the floor of the kitchen?" 
%
%"Ok. How do I do that?"
%
%"The mop is right next to you."
%
%"Do you have a bucket or something?"
%"Where should I start?" "Where is the moping fluid?" "How do I not walk where I mopped?"
%
%You have ceased to be helpful. You have taken more time than you saved. You have made work for me, not less. You are fired. 
%
%It isn't about learning to do a job by watching a person do the whole job and then doing exactly what they did. Get started, use your own methods and ideas, and get as much done as you can. When you ask for help make it clear exactly where you need help. I'm not going to mop the floor for you. 




%%%voice?
%Using the definition of a function the value of f(1) would be 2. ?!?!?




%\section{e.g.s}
%How should one measure accuracy of a model? \\
%One should measure accuracy by plugging in the numbers and calculating them. 

%Questions for me:\\
%Is there a voice with no person? "The equation Q=kt describes the relationship between the force Q of a spring attached to a block which is displaced from the ... in terms of a constant k which .."\\
%
%
%








%\section{Comparative terms without a pair to compare}
%e.g. The value of $f$ at $a+1$ is bigger.\\
%
%\noi The difference between the graph of a function and the domain of a function is the domain is the set of all objects for which a function assigns another object.




\section{Less than sophisticated addictive phrases}
Here, I have collected some phrases that I frequently see in my students writing that strike me as unsophisticated. 
I do this so that you can look at your own writing and think over which phrases you might habitually use that are best left in elementary school before they end up on a job application. 
\\[1cm]

%``Functions are equations are solutions are functions...."\\ In Conflating, above
\noindent 
-``We are given ..."\\
``plug it in"\\
``Solve it out"\\
-``I would" \\
%``First do this, then do that"\\
``The answer is..."\\
-``3+4 looks like 7."\\
False urgency, or no other way mentality ``We must"\\
Indicating canceling with slashes.\\
Chimeras ``The value of f at $3= 5$."\\
Future subjunctive? ``1+x+1 would be x+2." \\
A solution is a time? ``A solution to an equation is when you substitute..."  \\ 



\section{Literacy }


%Academics who study literacy in the ancient world have found that, in ancient times a person was considered literate if they could write their name even if they could not read a single word. Things changed. 
%Last century, you were not literate if you did not read ``the great works" like Shakespere, Plato, etc..
%
%Things changed again. These days, in our culture, if one can not read a 
%novel then one is considered illiterate. 
%
%Much, much more is expected of a college graduate;  

A college degree should indicate that the degree holder has developed and demonstrated the ability to communicate about technical, complicated, nuanced topics and also to develop and communicate the holder's own ideas. 
This kind of communication requires one to think hard about one's audience. The difficulty in verbalizing thoughts is not in putting your ideas into words such that you can read what you wrote, but in putting your ideas into words that other people can not misunderstand. 

Literacy is not just about the ability to read. It is the ability to read and write. 
It is my opinion that college courses should require students to read, comprehend, write, and be comprehended. This is an important set of skills to have in California, 2014; 
our economy and job markets center more and more on technical skills. 
To participate, you must be able to read and write about technical material. 
Development of these skills takes tremendous amounts of practice, but I hope these handouts and homeworks have helped to provide you with a sense of direction in this process. 
\\


-Professor Cherney, \\

%2014



%If you can't read the text book then should you get a passing grade? 
%
%
%The trifecta: Read, write, and speak about calculation.



%\section{Idiosyncratic Notation }
%$\frac{d}{dx}= 2x.$ (Has this been independently invented dozens of times? )\\
%$ss= (HT),(TH)$.\\
%Your syllabus says that one of the goals is for you to learn standard mathematical notations. 
%

%\section{If there was no question, what do you mean by answer?}
%e.g. Instructions: Calculate $1+2$.\\
%Response: The answer is 3.


%\section{Why use the formula?}
%``In order to calculate the interest earned we use the formula $F=P(1+rt)$."
%vs.\\
%``Inflation is best modeled as compound interest compounded annually. The relationship between the Future value FV and present value PV of an account earning compounded interest at an annual interest rate  5\% is ..." 
%\section{Function do not have solutions}



%\section{Want more examples? It is probably best to start writing}
%I was once cleaning a house. Someone else's house. A person asked if they could help. 
%
%"Sure," I said, "can you mop the floor of the kitchen?" 
%
%"Ok. How do I do that?"
%
%"The mop is right next to you."
%
%"Do you have a bucket or something?"
%"Where should I start?" "Where is the moping fluid?" "How do I not walk where I mopped?"
%
%You have ceased to be helpful. You have taken more time than you saved. You have made work for me, not less. You are fired. 
%
%It isn't about learning to do a job by watching a person do the whole job and then doing exactly what they did. Get started, use your own methods and ideas, and get as much done as you can. When you ask for help make it clear exactly where you need help. I'm not going to mop the floor for you. 
%
%


%%%voice?
%Using the definition of a function the value of f(1) would be 2. ?!?!?




%\section{e.g.s}
%How should one measure accuracy of a model? \\
%One should measure accuracy by plugging in the numbers and calculating them. 

%Questions for me:\\
%Is there a voice with no person? "The equation Q=kt describes the relationship between the force Q of a spring attached to a block which is displaced from the ... in terms of a constant k which .."\\
%
%
%






\end{document}

