%%%%%%%%%%%%%%%%%% Latex %%%%%%%%%%%%%%%%%%%%%
\documentclass[12pt]{article}

\usepackage{amsmath}
\usepackage{shadow}
\usepackage{dingbat}
%\usepackage{bbold}
\usepackage{amssymb}
\usepackage{slashed}
%\usepackage{graphics}
\usepackage{graphicx}
%\usepackage{pstricks,pst-node,pst-tree}

% equations
\def\TT{{\mathrm TT}}
\def\N{{\cal N}}
\def\GG{{\cal G\hspace{-2.3mm}G}}
\def\Gg{\mathbb G}
\def\E{{\cal E}}
\def\B{{\cal B}}
\def\lrd{\stackrel{\leftrightarrow}{\partial}}

\def\chib{\overline\chi}
\def\psib{\overline\psi\hspace{-2.6mm}\phantom{\psi}}
\def\tid{\hspace{.3mm}}
\def\ve{\varepsilon}
\def\D{{\cal D}}
\def\M{{\cal M}}
\def\J{{\cal J}}
%greek
\def\a{\alpha}
\def\b{\beta}
\def\de{\delta} 
\def\De{\Delta}
\def\eps{\epsilon}
\def\r{\rho}
\def\k{\kappa}
\def\l{\lambda}
\def\m{\mu}
\def\n{\nu}
\def\s{\sigma}
\def\t{\tau}
\def\z{\zeta}


%%phantoms
%\def\ph#1{\phantom{#1}}
%\def\ag#1{\gamma({\scriptstyle #1})}
%\def\Sl#1{\slashed{#1}}
%\def\Sl#1{#1\hspace{.1mm}\!\!\!\!/\,}
%\def\sl#1{\slashed{#1}}
%\def\wh#1{\widehat{#1}}
%\def\wt#1{\widetilde{#1}}
%\def\mt{{\tilde \mu}}
%\def\nt{{\tilde \nu}}
%\def\rt{{\tilde \rho}}
%\def\st{{\tilde \sigma}}
%\def\ol#1{\overline{#1}} 
%\def\sss{\scriptscriptstyle}
%\def\ts{\textstyle}
\def\d{{\bfseries d}}
\def\la{\langle}
\def\ra{\rangle}
\def\e{\epsilon}
\def\scirc{\!{\scriptstyle \circ}}
\def\psitb{\overline{\widetilde\psi}}
\def\vphi{\varphi}
\def\psid{\psi^\dagger}

%Sets
\def\F{{\cal F}}
\def\O{{\cal O}}
\def\Z{{\mathbb{Z}}}
\def\Q{{\mathbb{Q}}}
\def\A{{\mathbb{A}}}
\def\R{{\mathbb{R}}}
\def\C{{\mathbb{C}}}
\def\H{{\mathbb{H}}}

%\matricies
\def\ba{\left(\begin{array}{cc}}
\def\ea{\end{array}\right) }
\def\be{\begin{equation}}
\def\ee{\end{equation}}
\def\beq{\begin{equation}}
\def\eeq{\end{equation}}
\def\bea{\begin{align}}
\def\eea{\end{align}} 
\def\beqa{\begin{equation}\begin{array}{l}}
\def\eeqa{\end{array}\end{equation}}

%--------------------------------------------
% symbols
%warning \be is for equations now
\def\ga{\gamma} 
\def\G{{\it\Gamma}} 
\def\g{\gamma}
\def\veps{\varepsilon}  
\def\L{{\it\Lambda}}
\def\Pit{{\it\Pi}}
\def\Psit{{\it\Psi}}
\def\si{\sigma} \def\Si{{\it\Sigma}}
\def\th{\theta} \def\vth{\vartheta} \def\Th{\Theta}
\def\w{\omega} \def\W{\Omega} \def\hw{\hat{\omega}}
\def\vfi{\varphi}\def\vphi{\varphi}
\def\bra{\langle} \def\ket{\rangle}
\def\dd{{\mathrm d}}
\def\pa{\partial}
\def\vrho{\varrho}

\def\ie{{i.e., }}
\def\eg{{e.g.\ }}
\def\LR{\Leftrightarrow} 
\def\cl{\centerline}

\def\pa{\partial}
\def\rarr{\rightarrow}
\def\nn{\nonumber}

\def\psibar{\overline{\psi}}
\def\Gbar{\overline{G}}

\def\psidag{\psi^\dagger}
\def\Gdag{G^\dagger}

\def\N{{\bfseries N}\,}
\def\TR{{\bfseries tr}\,}
\def\G{{\bfseries g}\,}
\def\DIV{{\bfseries div}\,}
\def\GRAD{{\bfseries grad}\,}
\def\ord{{\bfseries ord}\,}

\def\Nt{{\cal N}}
\def\Tt{{\cal T}}
\def\c{{\hspace{.3mm}\bfseries c\hspace{.3mm}}}
\def\Real{\mathbb R}

\def\thup{\rightthumbsup}
\def\thdn{\rightthumbsdown}


\def\noi{\noindent}
%%%%%%%%%%%%%%%%%%%%%%%%%%%%%%%%%%%%%%%%%%%%

\begin{document}

\thispagestyle{empty}
~
\vspace{-3.9cm}
\begin{center}
{
\Large
 {\bfseries  Writing With Mathematics\\
 Part 1: Write everything in complete sentences.}  \\[10mm]
 {  
 \vspace{-.9cm} 
%MAT 003-03, SMC Fall, 2013\\
Cherney}\\[1mm]
%{\small Office hours: Tues \& Thurs 2-4pm, Galileo 103A\\dmc13@stmarys-ca.edu}
\vspace{3.5mm}
}

\end{center}

\setcounter{footnote}{0}
\setcounter{page}{1}
%%%%%%%%%%%%%%%%%%%%%%%%%%%%%%%%%%%%%%%%%%%%%%%%%%%%%%%%%%%%%%%%%

%Your grade in this course will be, in part, determined by your ability to write about mathematics. 
%alternative: One of the main goals of this course is to develop your ability to write about mathematics/write using mathematics. 
%alt: 
Part of learning mathematics is learning to read and write with mathematics.  
Aspects of this course will build and exercise high level mathematical literacy. 
%this course is practicing writing about mathematics and with mathematics. 
This handout is intended to convey the importance of practicing such writing,
%practicing writing about and with math, 
what writing with mathematics looks like, what it should not look like, and what common mistakes are made while writing about math. 

%If this document looks too long and technical to you, take that as an indicator that you do not have sufficient mathematical literacy. 

This handout is the first of a set. The rest of the set will cover 
2) gramatical structure,
3) formatting, voice, and mood.
It will be helpful to reread previous handouts and examine their ideas in light of later handouts.


\section*{Why write and not just calculate?} 
%Many students believe that they are good at math, but they cannot write about it.  
%as long as it does not involve writing about 

A typical high school education in math %from the past 20 years 
focused on preparation for standardized tests such as STAR, CAT, SAT, NAEP etc. These tests tend to be multiple choice simply because multiple choice exams are easy to grade via scanning technology. 
%The unfortunate side effect of this mode of training for testing via scaning is that 
As a result students are not trained to communicate about mathematics with people, 
but rather to communicate with Scantron machines. 
It has been a matter of considerable debate if this training leads to any desirable skill set. It is not a matter of debate that practicing communication with humans leads to improved communication with humans. 
%It is not clear that this skill is 
%This training leads to a skill which is not transferable.

\subsection*{For Beyond This Class} 
The ability to calculate using certain algorithms is ephemeral and of little importance; 
you will forget how to perform long division just as you will forget how to `sound out' the words ``See Jane Run". 
You will almost never perform long division 
and you will almost never sound out ``See Jane Run" again. 
You learned long division as a step toward mathematical literacy, 
just as you learned how to `sound out' simple words as a step in your literacy. 

Your verbal literacy and mathematical literacy are not independent of one another; mathematics needs to become a part of your everyday language if you wish to participate in the world. 
From pitching a business plan to understanding a coupon, 
from learning science to describing the way the world works to others, 
mathematics is a tool for communication of precise ideas.
Communication skills, in contrast to calculation skills, develop cumulatively over the course of a life and are of prime importance. 
To communicate or develop precise ideas you must be an effective communicator. 
If you can calculate and cannot communicate then you are not of use; the modern world is full of things that can compute better than you. They are called computers. On the other hand, if you can communicate well 
%because you used to be able to calculate  
then you can easily relearn to any algorithms you have forgotten 
AND develop new means of calculation with others. 
This is what the world wants of you, not a vain attempt to learn to compute better than a computer.

So, imagine your classwork stapled to your future job applications (or business plan pitches etc..) as an example to your prospective employers of your ability to communicate. 
Ask yourself what would get you hired; 
would it be pages of calculations that only you can read (effectively scratch paper) or would it be well written, well thought out, clearly delineated steps of deductive logic?

\subsection*{For This Class}
%When new to a subject it is hard to use the jargon accurately. 
%Blind use of formulas is rampant (in the era of standardized testing) and often leads to mistakes. 
%Requiring 
As you carefully write about a topic you often discover gaps in your understanding, and that you can fill in these gaps on your own. The act of filling in these gaps will build self-confidence in your understanding. That confidence will lead to better performance on exams. 
You should practice the way you want to perform. 
This holds for sports, music...  and academics.

%Careful writing has a way of bringing out misunderstandings. 
%You, dear student, need to be able to write questions relevant to your course. 
%If you can't, then you didn't learn the domain of applicability of the course's material. 
%\\

\section*{What should my goals be while writing?}
These three imperatives should guide your writing: 

\begin{enumerate} 
\item Communicate while you Calculate;
\item Give sufficient context;
\item Be clear, concise, and accurate.
\end{enumerate} 


\subsection{Communicate While You Calculate}
%When preparing for standardized tests, there is no need to practice communication; practice with calculations via scratch-work is sufficient for deciding which bubble to fill. Such work may use idiosyncratic notation, be incomplete is various regards (such as missing symbols held in the writer's mind), and is typically unreadable to even the author after a few days. 
%If one is going to communicate using mathematical ideas one must eventually learn the difference between writing scratch-work and writing to communicate. 
%
%
%Transitioning from math classes that only require scratchwork and do not require written communication is difficult. 
%Students often do not know where to begin.  
%They often word their frustration as ``I can do the math, I just can't write about it."
%What they mean is they can calculate but they cannot communicate. 

The phrase to keep in mind while starting the transition from calculating without communicating to communicating while calculating is\\

\shabox{ \Large{{\bfseries ``Write everything in complete sentences."} }} \\

\noi The only exceptions to this rule being pictures and diagrams, which will be discussed later.

Your primary goal should be to show (via complete sentences) a string of logic that leads to an answer/response. 
Simply reporting a numerical result of a calculation is unacceptable. 

An example of an unacceptable writing follows.
\begin{quote}
\shabox{$f(x)=x^2, g(x)=x+1, f(g(2))=9.$} \thdn
\end{quote}
This writing is not in sentences and does not show a chain of logic. 
Presumably someone who wrote this was attempting communicate the contents of the following line.\\

%\begin{quote}
\noi \shabox{
If $f(x)=x^2$ and $g(x)=x+1$     then $f(g(2))=f(2+1)=f(3)= 3^2=9.$}\thup\\
%\end{quote}

\noi This writing is in sentences and shows how certain statements imply other statements. That is, it shows a flow of logic. 

If one is new to writing about mathematics one may uncertain about which details to explicitly state. For example, a student might worry that the above is insufficient because it does not include the line ``since $2+1=3$". This uncertainty can be alleviated by practice and by knowing your audience; some readers would be annoyed at the inclusion of the arithmetic statement, others might need the reminder. For the sake of classwork, pretend your audience is another student that knows as much as you did a few weeks ago. \\

To reiterate, 
when you answer a question or implement instructions 
 your goal is not to obtain a number and put it in a box. 
 Your goal is to put together a string of deductive logic that leads from from some conditions to some conclusions. 
 
A good saying is 

\begin{quote}
 \shabox{
{\bfseries ``The answer is the whole argument." } }\thup
\end{quote}


%Many students have been trained to perform scratch calculations, and you may do so, but not in any place where another person will read your work.

\subsection {\bfseries Give sufficient context} 
Your secondary goal should be to give your reader enough information to understand what you are saying without needing to look up the question/instructions you are responding to. You should not even implant the idea of looking at another source.


An example of writing with insufficient context is the following line.
\begin{quote}
\shabox{The number of wolves after 3 years is $\frac{1,000}{3+2}=200$ so there aren't enough to go hunting.}\thdn
\end{quote}
It is not clear why we are reading the words ``$\frac{1,000}{3+2}=$" nor in what sense there are not enough wolves to go hunting. An improvement upon this writing is the following. 
\begin{quote}
\shabox{If in order for wolf hunting licenses to be issued for a certain forest the pollution of wolves must exceed 250, and the number of wolves $N$ as a function of time $t$ in years since the licensing began is given by  $N(t)=\frac{1000}{t+2}$, then after 3 years the number of wolves $N(3)=\frac{1,000}{3+2}=200$ is below the threshold for issuing licenses. }\thup
\end{quote}
Even when responding to questions/problems that are not ``word problems" you should give sufficient context, as the following line fails to do.
\begin{quote}
\shabox{If you multiply $f$ and $g$ then you get $x+1$.} \thdn
\end{quote}
No one besides the writer of this sentence knows why it might be true. Contrast this with the following sentence.
\begin{quote}
\shabox{The product of the functions $x+2$ and $\frac{x-1}{x+2}$ is the function $x-1$.}\thup 
\end{quote}


Part of establishing context is introducing any relevant data, symbols, and formulas, etc. You need to do so without explicit or implicit mention of the question/instructions you are responding to.
Students tend to use the phrase ``the given" to establish context in terms of a question or instructions they are given. 


\begin{quote}
\shabox{Using the given $f(x)=x^2$ we can find $f(2)=2^2 $. }  \thdn
\end{quote}



A conditional statement can remove the necessity of informing your reader that you have been given some information.

\begin{quote}
\shabox{If $f(x)=x^2$ then $f(2)=2^2=4$.} \thup
\end{quote}

For a more in depth example, lets say you are given the following instructions.

\begin{quote}
Calculate the quantities of vanilla ice cream and mocha ice cream you can make if 
each gallon of vanilla ice cream requires 2 cups of sugar and 2 cups of milk,  
each pint of mocha ice cream requires 3 cups of sugar and 2 cups of milk, 
and you have 60 cups of sugar and 45 cups of milk. 
\end{quote}
%%%write here, jerk 

Statements like ``Let x be  vanilla and y be mocha" are insufficient to give context of an answer to a question or implementation of instructions.
Giving sufficient context means writing as though your audience has not seen the question or instructions you are responding to. \\

\shabox{If each gallon of vanilla ice cream requires 2 cups of sugar and each gallon of mocha ice cream requires 3 cups of sugar and all of a 60 cup sugar supply is to be used to make ice cream, then
\[2x+3y=60\] where x and y are the numbers of gallons of vanilla  and mocha ice cream made, respectively. If further 
each gallon of vanilla ice cream requires 2 cups of milk and each gallon of mocha ice cream require 2 cups of milk and all of a 30 cup sugar supply is to be used to make ice cream, then
\[2x+2y=45.\]
Subtracting the second of these equations from the first yields 
\[ y=15.\]
Substituting 15 for $y$ in the second equation yields
\[2x+2(15)=45 \] 
or 
\[x=15/2 .\] 
Thus, 7.5 and 15 gallons of vanila and mocha ice cream, respectively, can be made with 30 cups of sugar and 35 cups of milk. 
}\thup

%\begin{quote}
%\shabox{If $x=1$ and $y=2$, since the question is asking for the sum of x and y, $x+y=2$. }  %student's mindset? 
%\end{quote}



\subsection{\bfseries Be clear, concise, and accurate.} 
Your tertiary goal should be to choose words carefully to construct a small number of sentences, each of reasonable length, with no mistakes in logic or calculation.\\

\noi The following is not clear.\\
% but is typical of written work presented by high school graduates.\\

\shabox{
\noi The amount of money. 
\[F=P(1+rt)\]
Its .05 and $P$ is 500 so when t is 2 it is
\[500(1+(.05)  2)=550.\]}\thdn \\

The first line is not a sentence; if your writing is to be readable put everything in sentences. The meaning of the symbols F,P,R, and t are not specified, nor is any context given. A vast improvement on the above presentation is given below.\\

\shabox{
\noi 
The  value $F$ of an investment earning simple interest at an annual interest rate $r$ after $t$ years in terms of its initial $P$ when $t$ is zero is given by
\[F=P(1+rt).\]
If the interest rate is $.05$ and initial value is 500 then after 2 years the value of the investment is
\[500(1+(.05)  2)=550.\]}\thup \\

For clarity, it pays to proof read. But proof reading one's own writing is next to impossible just after writing. Put some time in between your writing and your proof reading. 


The following is not concise.\\

\shabox{
\noi Since the janitorial costs in 2011 were $\$10^6$ and the janitorial costs in 2012  were $\$3 \times 10^6$ and since a linear is to be used to extrapolate the janitorial costs in 2014, and since the point-point form for a linear function whose graph includes the points $(x_1,y_1)$ and $(x_1,y_1)$ is 
$\frac{y_2-y_1}{x_2-x_1}(x-x_1) +y_1$ we can calculate the extrapolation. To calculate the extrapolation we calculate 
\[(3-1)/({2012-2011}) (x-2011) + 1.\]
This calculation gives 
\[{2} (x-2011) + 1.\]
The calculation of the extrapolation then looks like 
\[{2} (2014-2011) + 1 =5 .\]
The janitorial costs will be approximately $5\times 10^6$ in 2014. 
}
\thdn \\

Repetition of various phrases rendered that example difficult to read. Also, the phrase `looks like' is noncommittal and inappropriate for a calculation.
%Other inappropriate noncommittal phrases are `or something like that',  
The following is an improved version.\\

\shabox{
\noi If the janitorial costs of a business in the years 2011 and 2012 were $\$10^6$ and $\$3 \times 10^6$ in 2011 and 2012, respectively, then a linear model $L$ based on this information may be used to extrapolate the costs in 2014. The point-point form of a linear function whose graph includes the points 
$(x_1,y_1)$ and $(x_1,y_1)$ is 
$\frac{y_2-y_1}{x_2-x_1}(x-x_1) +y_1$, so identifying the years and costs (in millions of dollars) as $x$ and $y$ values, respectively, yields the linear model
\[L(x)=(3-1)/({2012-2011}) (x-2011) + 1= {2} (x-2011) + 1.\]
Since $L(2014)={2} (2014-2011) + 1 =5$ 
 the  model predicts that the janitorial costs for 2014 will be $\$5\times10^6$.
} \thup\\


In homework, as in the rest of life, work may be repetitive. 
When this is the case a writer need not pass on that repetition to his reader. The following is not concise.
\begin{quote}
\shabox{
13) If $f(x)=x^2$ then $f(1)=1^2=1$.\\
14) If $f(x)=x+2$ then $f(1)=1+2=3$.\\
15) If $f(x)=x/2$ then $f(1)=1/2$.
}\thdn 
\end{quote}
Compare this collection of three sentences to the following single sentence.
\begin{quote}
\shabox{
13-15) \\[.2cm]
If $f(x)=x^2,~g(x)=x+2$ and $h(x)=x/2$ \\then $f(1)=1^2=1,~g(1)=1+2=3$ and $h(1)=1/2$.}\thup
\end{quote}

The following is not accurate.\\

\shabox{
\noi If the the function $x+1$ is composed with the function $x^2$ the result derived the result is \\[.2cm]
$\frac{d}{dx} (x^2+1)^2=2(x^2+1) \frac{d}{dx}(x^2+1)
=2(x^2+1) 2x$.
}\thdn\\

Not only is the composition of the functions supposed to be $(x+1)^2$, the verb ``derived" is in error; the word ``differentiated" would have been accurate. Accurate use of technical terms is necessary to communicate in technical fields. Inaccurate use of these terms leads to distraction of the reader at minimum, and insurmountable confusion for your reader at worst.







\section*{Whose example do I follow?}
Examples in textbooks usually follow the three imperatives above very well. However, textbooks often {\it additionally} give diagrammatic presentations. You should be aware of the distinction between diagrammatic  and verbal communication.

Lecture examples are, unfortunately, highly constrained by time and space (available on the blackboard or projector) and so are usually very poor examples of writing with the three imperatives above in mind. Keep in mind that examples in lecture are supplemented by vocalizations, gesticulations, and emotions which allow for an important kind of communication that is not possible in writing. 

Lastly, be aware that heavy reliance on examples will not lead to skill. 
%Writing is hard, and emulating the writing of another person is extremely hard.
Think of learning to play an instrument or play a sport. 
There is no replacement for practice in developing skill.
The best way to bring about improvement in your writing is to stay humble, realize that all writing can be improved upon (including mine and yours), and reflect often on others' feedback and on how you can improve. \\

\shabox{\Large{\bfseries Do not fear productive struggle!}}








\end{document}

