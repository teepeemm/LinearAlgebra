%%%%%%%%%%%%%%%%%% Latex %%%%%%%%%%%%%%%%%%%%%
\documentclass[12pt]{article}

\usepackage{amsmath}
\usepackage{shadow}
%\usepackage{bbold}
\usepackage{amssymb}
\usepackage{lmodern}
%\usepackage{slashed}
%\usepackage{graphics}
\usepackage{graphicx}
%\usepackage{pstricks,pst-node,pst-tree}

%Sets
\def\A{{\mathbb{A}}}
\def\C{{\mathbb{C}}}
\def\F{{\cal F}}
\def\H{{\mathbb{H}}}
\def\N{{\mathbb{N}}}
\def\O{{\cal O}}
\def\Q{{\mathbb{Q}}}
\def\R{{\mathbb{R}}}
\def\Real{\mathbb R}
%\def\R{{\cal R}}
\def\Z{{\mathbb{Z}}}

%greek
\def\a{\alpha}
\def\b{\beta}
\def\de{\delta} 
\def\De{\Delta}
\def\ga{\gamma} 
\def\g{\gamma}
\def\si{\sigma} 
\def\Si{{\Sigma}}
\def\e{\epsilon}
\def\eps{\epsilon}
\def\La{\Lambda}
\def\r{\rho}
\def\Th{\Theta}
\def\th{\theta} 
\def\k{\kappa}
\def\l{\lambda}
\def\m{\mu}
\def\n{\nu}
\def\s{\sigma}
\def\t{\tau}
\def\z{\zeta}

%equation 
\def\be{\begin{equation}}
\def\ee{\end{equation}}
\def\bea{\begin{align}}
\def\eea{\end{align}} 
\def\nn{\nonumber}

%quantum operators
\def\d{{\bfseries d}}
\def\D{{\bfseries {\cal D}}}
\def\Del{\bm \delta}
\def\Delb{\bar{\bm \delta}}
\def\DIV{{\bfseries div}}
\def\G{{\bfseries g}}
\def\GRAD{{\bfseries grad}}
%\def\N{{\bfseries N}}
\def\ord{{\bfseries ord}}
\def\Pa{\bm \partial}
\def\Partial{\bm \partial}
\def\Pab{\bar{\bm \partial}}
\def\TR{{\bfseries tr}}

%misc
\def\la{\langle}
\def\ra{\rangle}
\def\scirc{\!{\scriptstyle \circ}}
\def\pa{\partial}
\def\bra{\langle} \def\ket{\rangle}
\def\ie{{i.e., }}
\def\eg{{e.g.\ }}
\def\cl{\centerline}
\def\ni{\noindent}
\def\noi{\noindent}
\def\ba{\left(\begin{array}{cc}}
\def\ea{\end{array}\right) }


\newcommand{\comment}[1]{{\bfseries [#1]}}
\newcommand{\ul}[1]{\underline{#1}}
\newcommand{\bm}[1]{\mbox{\boldmath $ #1 $}}

%%%%%%%%%%%%%%%%%%%%%%%%%%%%%%%%%%%%%%%%%%%%%%%%%%%%%%%%%%%%%%%%%%%%%


%\renewcommand{\baselinestretch}{1.8}

\begin{document}
 
\thispagestyle{empty}
\phantom{a}
\vspace{-3.5cm}

\begin{center}
{
\LARGE {Terminology from Elementary Algebra }  \\[10mm]
{\sc \small Cherney%$^\diamondsuit$
}
}

%{\sc Abstract}
%\\
\end{center}

\setcounter{footnote}{0}
\setcounter{page}{1}
%%%%%%%%%%%%%%%%%%%%%%%%%%%%%%%%%%%%%%%%%%%%%%%%%%%%%%%%%%%%%%%%%
%

\section{Arithmetic}
\ni {\bfseries Definition}:
An {\bfseries arithmetic expression} is a sequence of numbers and symbols for operations (together with parentheses to disambiguate orders of operation).\\

\ni e.g.) $7\cdot13\div(4+8)-3$\\

\ni {\bfseries Definition}:
An {\bfseries arithmetic equation} is a statement equating two arithmetic expressions.\\

\ni e.g.)  $8\cdot3\div2=4\cdot3 $\\

\ni N.B. Arithmetic equations (like all statements) have truth value. The example above is a true arithmetic statement. The statement $3=1$ is a false arithmetic statement. \\

\section{Algebra}
 \ni {\bfseries Definition}:
An {\bfseries algebraic expression} is a sequence of numbers, symbols of operations, and additional symbols (usually letters) referred to as variables.\\

\ni e.g.s) $2,~a\cdot4\div3,~x^2+\frac23,~x^2y^3,~4\cdot\spadesuit\cdot \heartsuit$\\

 \ni {\bfseries Definition}:
An {\bfseries algebraic equation} is a statement equating two algebraic expressions. 
\\

\ni e.g.) $4x+2=14x$\\

\ni N.B. Algebraic equations do not have truth value.\\

 \ni {\bfseries Definition}:
 A {\bfseries solution to an algebraic equation} is a(n ordered collection of) number(s) that when substituted for the variable(s) yields a true arithmetic statement.\\
 
 \ni N.B. The definition above without parentheses is for algebraic equations on one variable, with parentheses for multiple variables. \\
 
\ni  e.g.s) \\
I) $7$ is a solution to $3x=21$ because $3\cdot7=21$ is a true arithmetic statement.\\
 II) $4$ is not a solution to $3x=21$ because $3\cdot4=21$ is a false arithmetic statement.\\
 III) $(2,4)$ is a solution to $x^2y=16$ because $2^2\cdot4=16$ is true.\\
 IV) $(0,1)$ is not a solution to $y=x^2$ because $1=0^2$ is false. \\ 
 V) $(1,3,5)$ is  a solution to $x^2=(z-y)/2$ because $1^2=(5-3)/2$. \\ 

%\newpage
%~
%\vspace{-3.3cm}
%
 \ni {\bfseries Definition}:
{\bfseries To solve} an algebraic equation is to follow a procedure that leads to solutions of the algebraic equation.\\



\ni {\bfseries Recall}: Algebra (from the old arabic ``al-jabr" meaning 'consolidate' or 'rejoin') can (sometimes) be used to solve algebraic equations in one variable. It is convenient to think of algebraic steps as instances of moving things from one side of an equation to another, according to various rules.\\

\ni e.g.) The process below is what it looks like to solve $3x^2+2=8$ by first moving the addend 2, then the multiplicand 3, then the power 2.
\bea \nn
3x^2+2=8\\ \nn
3x^2=6\\ \nn
x^2=2 \\ \nn
x=\pm 2^{1/2}
\eea
N.B. This example was written in the style of scratchwork.\\

  \ni {\bfseries Definition}: 
  The {\bfseries solution set} of an algebraic equation is the set of all solutions to the equation.\\
  
\ni e.g.) The solution set of $y=x$ is the set of all pairs of the form $(a,a)$ with $a$ a number. The pictorial representation of this set is below.\\
\includegraphics[scale=0.2]{yx}\\

\ni {\bfseries The main point of this handout:} Many people misspeak and call the solution set of an algebraic equation ``the graph of the equation". 
Equations do not have graphs; graphs are properties of functions. Similarly, functions do not have solution sets.  
We will discuss what a graph is below.\\


\ni {\bfseries Definition}: Two algebraic equations are {\bfseries equivalent} if their solution sets coincide.\\

\ni e.g.) $x+1=y+1$ is equivalent to $y=x$. \\

\ni N.B.The shorthand $\Leftrightarrow$ is used for the words ``is equivalent to'' so that we can say concisely $x+1=y+1\Leftrightarrow y=x$. 
%The convention when solving via algebra is that gong to a new line means \LR. 
\\

\ni {\bfseries Definition}: An algebraic equation is {\bfseries trivial} (or {\bfseries tautological}) if substitution of any numbers for it's variables yields a true arithmetic statement. \\

\ni e.g.) I) $4x=2x+2x$ is trivial.  II) $2x=4$ is not trivial.\\

\ni Note that all trivial algebraic equations are equivalent to $0=0$, and this is the most efficient way to check for triviality.

%\newpage
%~
%\vspace{-3.3cm}

\section{Functions %\\vs \\Algebraic Equations 
}

\ni {\bfseries The main point of this handout:} Functions are not equations. 
Algebraic equations are not functions. In the last part of this handout I'll try to explain two reasons students conflate these concepts. \\

%\ni {\bfseries Definition}: A {\bfseries function} is an assignment of each element of a set (called the domain) to an element of another set (called the target, or codomain). \\

\ni {\bfseries Definition}: 
A {\bfseries function} is comprised of three things: 
1) a set called the domain; 
2) a set called the codomain; 
3) a rule of correspondence that assigns to every element of the domain exactly one element of the codomain.\\


%\noi This is the definition of a function is the sense that it delineates the necessary and sufficient conditions for an object to be called a function.  If someone asks you to ``define a function" it is ambiguous is you are being asked to state the definition above or to give an example. 

%Do not get confused between\\ 
%a) the definition of a function \\
%b) the definition of a particular function \\
%(wow, this is hard!)
\noi e.g) Consider the function with domain the set of books in Cherney's office, codomain the positive integers, and rule of correspondence that assigns to each book in Cherney's office the number of pages in that book.\\ 

\noi N.B. It was not necessary to use an equal sign to define that function. Nor was it necessary to use a symbol like $f$ or $g$. However, if I wish to refer to the function defined  in the example above, I do not have means of doing so quickly. It might pay to give the function a nick name for later use, as is done in the following example. \\


\noi {e.g. of giving a function a short name:} \\
Let $f$ be the function that assigns to each current employee of 
UCD his biological mother. The domain of this function is the set of current employee of UCD while the codomain is not explicitly specified but could be, for e.g., the set of human females. While it is true that $f(\text{David~Cherney})=\text{Susan~Cherney}$ this example does not define the function; the complete sentence with which this example begins defines all of the function with the exception of the codomain. \\

\noi {\bfseries Note}: One can define a function in terms of an algebraic expression. Not all functions can be defined this way.\\

\noi {\bfseries e.g.}    Let $f$ be the function with domain the set of all real numbers, codomain the set of all real numbers, and rule of correspondence that assigns to $x$ the number $x^2$ for all $x$ in the real numbers. (So $f(4)=16$.)\\
{\bfseries Non-e.g.} Let $g$ be the function with domain the set of all real numbers, codomain the set of all real numbers, and rule of correspondence that assigns to $x$  the smallest prime number greater than $x$  for all $x$ in the real numbers. (So $f(4)=5$.)
\\

\ni N.B. \\
1) Both of these functions  were defined without use of an equal sign. Functions are not equations.\\
2) If you can find an algebraic formula for this second function you will immediately become one of the  most famous mathematicians of all time. \\

\ni {\bfseries Reason \#1  for conflating functions and equations:} \\
The implicit domain convention.\\

Because full specification of a function is wordy and because functions are often taught to young people (8th grade in Common Core), algebraic definitions of functions (such as the first example above) are frequently abbreviated to statements like 
`Let $f$ be the function such that $f(x)=x^2$.'  This wording assumes the reader knows that the domain is $\R$ and that the codomain is $\R$; of the three parts of a function (domain, codomain, and rule of correspondence) only a terse form of the rule of correspondence is stated. Even more laconically, authors often define functions with statements like `Let $f(x)=x^2$.' \\

\ni {\bfseries Convention:} \\
If the rule of correspondence of a function is algebraically specified without specification of the domain or codomain then 
the intended codomain is $\R$ and
the intended domain is the largest set of (ordered sets of) numbers that can be substituted into the expression to yield real numbers. 
This is called the {\bfseries natural domain} of the algebraic expression.
%If the rule of correspondence of a function is algebraically specified without specification of the domain or codomain \\
%then \\
%the intended domain is the largest set of (ordered sets of) numbers that can be substituted into the expression to yield real numbers; the codomain will not contain any objects that are not real numbers. This is called the {\bfseries natural domain} of the algebraic expression.\\

\ni e.g.s \\
1) The natural domain of $x^2$ is $\R$. \\
2) The natural domain of 
$\sqrt{1-x^2}$ is
\begin{gather*}\{ x\in \mathbb{R}| \sqrt{1-x^2} \in \R\} 
= 
\{ x\in \mathbb{R}| 1-x^2\geq 0 \} \\= 
\{ x\in \mathbb{R}| 1\geq x^2 \} 
= 
\{ x\in \mathbb{R}| 1 -1\leq1 x \leq 1 \} = 
[-1,1].\end{gather*}
3) The natural domain of $\ln (x^2+y^2-1)$ is 
\[
\{ (x,y) \in \mathbb{R}^2 | \ln (x^2+y^2-1) \in \mathbb{R}\} 
= 
\{ (x,y) \in \mathbb{R}^2 | x^2+y^2-1>0 \}.
\]

\noi {Note:} When using an algebraic expression to concisely refer to a function
there is almost no reason to given that function another name. For example, when discussing the function $x^2$ (with its conventional domain and codomain left tacit) there is almost no need to name that function $f$ as very little space is saved. Of course the algebraically defined function $\ln\left( \frac{x^2-3x+2}{\sqrt{x-5}+2}\right)$ (with its conventional domain and codomain left tacit) is painful to write several times, and it may be nice to name this function something more concise like $f$, or some other one symbol name.\\

\ni {\bfseries Note:}
There is an extreme emphasis on algebraically defined functions in most mathematics education, and as a result of seeing nearly exclusively algebraically defined functions specified in this almost maximally laconic notation students are led to believe that functions are equations. \\

\ni But functions are not equations.\\


\ni {\bfseries Reason \#2 for conflating  functions and equations:}\\
A certain relationship between graphs of functions and solutions sets of algebraic equations is often taken for granted.\\

\noi {\bfseries Definition}: The {\bfseries graph of a function} $f$ is the set of ordered pairs of the form $(x,f(x))$ with $x$ in the domain of $f$.\\

\ni e.g.s\\
1) The graph of the function $x^2-1$ is $\{ (x,x^2-1)\in \R~ |~ x\in \R\}$.\\
2) The graph of the function $\sqrt(x+y)$ is $\{ (x,y,\sqrt{x+y}) ~\vline~ x,y \in \R\}$\\

\noi {\bfseries Coincidence}: If $f$ is a function defined through an algebraic expression $f(x)$ then the graph of $f$ is identical to the solution set of the algebraic equation $y=f(x)$. \\

\noi e.g.s \\
1) The  graph of the function %$f$ 
algebraically defined by %$f(x)=
$2x$ is 
\[\{(a,2a)| a\in \mathbb{R}\}\] 
while the solution set of the algebraic equation $y=2x$ is  \[\{(a,2a)| a\in \mathbb{R}\}.\]
2) The graph of the function $x^2+y$ is 
\[ \{ (x,y, x^2+y) ~|~ x,y,\in \mathbb{R}\}\] 
while the solution set to the algebraic equation $z=x^2+y$ is 
\[ \{ (x,y, x^2+y) ~|~ x,y,\in \mathbb{R}\}.\]


Dear math 22A Students,

You Should have all received and email 


%Why is it important to keep track of these differences? Am I just being pedantic? 





%
%\newpage
%~
%\vspace{-3.3cm}
%
%\section{Linear Functions \\vs Linear Algebraic Expressions\\ vs Linear Equations}
%
%First, functions:\\
%
%
%\ni {\bfseries Definition}: A function $f$ is {\bfseries homogeneous of degree 1}
%(over the real numbers) if $f(cx)=cf(x)$ 
%for  any real number $c$ and any $x$ in the domain of $f$ such that $cx$ is also in the domain of $f$.\\
%
%\noi e.g.) I) The function $f$ defined algebraically by $f(x)=3x$ for all real numbers $x$ is homogeneous of degree 1 since $f(cx)=3(cx)=c(3x)=cf(x)$ for any numbers $x$ and $c$.\\
%II) The function $g$ defined by 
%\\
%
%\ni {\bfseries Definition}: A function $f$ is {\bfseries linear} (over the real numbers) if the following two conditions are satisfied for all $x,y$\\
%$f(cx)=cf(x)$ for every number $c$.\\
%
%
%\ni {\bfseries Definition}: A an algebraic expression $f(x)$ of one variable $x$ is {\bfseries linear} (over the real numbers) if the following two conditions are satisfied for all $x,y$\\
%$f(cx)=cf(x)$ for every number $c$.\\
%
%\ni e.g. I) The algebraic expression $3x$ is linear because $3cx=c3x$.\\
%II)  $3x+2$ is not linear because $3cx+2\neq c(3x+2)$.\\
%
%
%\ni Note: Students often find this confusing because somewhere along the way they learned that the graph of a function of the form $mx+b$ is a line, and 
%


 





















\end{document}

