


\begin{enumerate}
\item \label{numerouno} Check that $\left\{\colvec{x\\y} \middle|\,  x,y \in \mathbb{R} \right\} = \mathbb{R}^2$ (with the usual addition and scalar multiplication) satisfies all of the parts in the definition of a vector space.

\vspace{5mm}
%%%%%%%%%%%%%%%%%

\item 
\begin{enumerate}
\item Check that the complex numbers $\mathbb{C}= \left\{x+iy \mid i^2=-1, x,y\in \mathbb{R} \right\}$, satisfy all of the parts in the definition of a vector space over  ${\mathbb C}$.
Make sure you state carefully what your rules for vector addition and scalar multiplication are. \\
\item What would happen if you used ${\mathbb R}$ as the base field (try comparing to problem~\ref{numerouno}).
\end{enumerate}
\vspace{5mm}
\phantomnewpage

\item \begin{enumerate}
\item Consider the set of convergent sequences, with the same addition and scalar multiplication that we defined for the space of sequences: 
\[V = \left\{f \mid f \colon \mathbb{N} \rightarrow \mathbb{R}, \lim_{n \rightarrow \infty} f(n) \in \mathbb{R} \right\}\subset {\mathbb R}^{\mathbb N}\, .\]

Is this still a vector space?  Explain why or why not.

\item Now consider the set of divergent sequences, with the same addition and scalar multiplication as before:
\[V = \left\{f \mid f \colon \mathbb{N} \rightarrow \mathbb{R}, \lim_{n \rightarrow \infty} f (n)\text{ does not exist or is }\pm \infty \right\}\subset {\mathbb R}^{\mathbb N}\, .\]

Is this a vector space? Explain why or why not.
\end{enumerate}

%\item Let $V= \{\colvec{x\\y} : x,y \in \mathbb{R} \} = \mathbb{R}^2$.  

%Propose as many rules for addition and scalar multiplication as you can that satisfy some of the vector space conditions while breaking some others.

\vspace{5mm}
\phantomnewpage

\item Consider the set of $2\times 4$ matrices:
\[ V = \left\{ 
\begin{pmatrix}
a & b & c & d \\
e & f & g & h 
\end{pmatrix}
\middle| a,b,c,d,e,f,g,h \in \mathbb{C} \right\}
\]

Propose definitions for addition and scalar multiplication in $V$.  Identify the zero vector in $V$, and check that every matrix in $V$ has an additive inverse.  

\vspace{5mm}
\phantomnewpage

\item \label{problem_polynomials} Let $P_3^{\mathbb{R}}$ be the set of polynomials with real coefficients of degree three or less.
	\begin{enumerate}
	\item Propose a definition of addition and scalar multiplication to make $P_3^{\mathbb{R}}$ a vector space.

	\item Identify the zero vector, and find the additive inverse for the vector $-3-2x+x^2$.

	\item Show that $P_3^{\mathbb{R}}$ is not a vector space over $\mathbb{C}$.  Propose a small change to the definition of $P_3^{\mathbb{R}}$ to make it a vector space over $\mathbb{C}$. (Hint: Every little symbol in the the instructions for par (c) is important!)
\end{enumerate}

\Videoscriptlink{vector_spaces_hint.mp4}{Hint}{scripts_vector_spaces_hint}

\item Let $V=\{x\in {\mathbb R}|x>0\}=:{\mathbb R}_+$. For $x,y\in V$ and $\lambda\in {\mathbb R}$, define
\[
x\oplus y = xy\, ,\qquad \lambda \otimes x = x^\lambda\, .
\]
Show that $(V,\oplus,\otimes,{\mathbb R})$ is a vector space.

\item The component in the $i$th row and $j$th column of a matrix can be labeled $m^i_{j}$. In this sense a matrix is a function of a pair of integers. For what set $S$ is the set of  $2\times2$ matrices the same as the set $\mathbb{R}^S$? Generalize to other size matrices. 

\item Show that any function in $\mathbb{R}^{\{*,\star,\# \}}$ can be written as a sum of multiples of the functions $e_*,e_\star,e_\#$ defined by 
\[
   e_* (k)= \left\{\!\!
     \begin{array}{ll}
       1\, , &  k=*\\
       0\, , &  k=\star \\
       0\, , & k=\# 
     \end{array}
   \right. 
 ,~
   e_\star (k)=  \left\{\!\!
     \begin{array}{ll}
       0\, , &  k=*\\
       1\, , &  k=\star \\
       0\, , & k=\# 
     \end{array}
   \right. 
  ,~
   e_\# (k)=  \left\{\!\!
     \begin{array}{ll}
       0\, , &  k=*\\
       0\, , &  k=\star \\
       1\, , & k=\# 
     \end{array}
   \right.. \]

\item Let $V$ be a vector space and $S$ any set. Show that the set $V^S$ of all functions  $S \to V$ is a vector space.
{\itshape Hint:} first decide upon a rule for adding functions whose outputs are vectors.

\end{enumerate}

\phantomnewpage