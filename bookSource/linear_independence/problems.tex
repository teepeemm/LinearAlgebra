


\begin{enumerate}
\item \label{bitsprob}Let $B^n$ be the space of $n\times 1$ bit-valued matrices ({\itshape i.e.}, column vectors) over the field \({\mathbb Z}_2\).
Remember that this means that the coefficients in any linear combination can be only \(0\) or \(1\), with rules for adding and multiplying coefficients given \hyperref[Z2]{here}.  
\begin{enumerate}
\item How many different vectors are there in $B^n$?
\item  Find a collection $S$ of vectors that span $B^3$ and are linearly independent.  In other words, find a basis of $B^3$.
\item Write each other vector in $B^3$ as a linear combination of the vectors in the set $S$ that you chose.
\item Would it be possible to span $B^3$ with only two vectors?
\end{enumerate}

\Videoscriptlink{linear_independence_hint.mp4}{Hint}{linear_independence_hint}

\phantomnewpage

\item \label{stdbasis} Let $e_i$ be the vector in $\mathbb{R}^n$ with a $1$ in the $i$th position and $0$'s in every other position.  Let $v$ be an arbitrary vector in $\mathbb{R}^n$.
\begin{enumerate}
\item Show that the collection $\{e_1, \ldots, e_n \}$ is linearly independent.
\item Demonstrate that $v=\sum_{i=1}^n (v\dotprod e_i)e_i$.
\item The $\spa \{e_1, \ldots, e_n \}$ is the same as what vector space?
\end{enumerate}

\phantomnewpage

\item Consider the ordered set of vectors from $\mathbb{R}^3$
\[
\left( \colvec{1\\2\\3 } , \colvec{2\\4\\6}, \colvec{1\\0\\1} , \colvec{1\\4\\5} \right) 
\]
\begin{enumerate}
\item Determine if the set is linearly independent by using the vectors as the columns of a matrix $M$ and finding $\rref(M)$.
\item If possible, write each vector as a linear combination of the preceding ones.
\item Remove the vectors which can be expressed as linear combinations of the preceding vectors to form a linearly independent ordered set. (Every vector in your set should be from the given set.)
\end{enumerate}

\item
Gaussian elimination is a useful tool to figure out whether a set of vectors spans a vector space and if they are linearly independent.
Consider a matrix $M$ made from an ordered set of column vectors $(v_1,v_2,\ldots,v_m)\subset {\mathbb R}^n$ and the three cases listed below:
\begin{enumerate}
\item $\rref(M)$ is the identity matrix.
\item $\rref(M)$ has a row of zeros.
\item Neither case (a)  or (b) apply.
\end{enumerate}
First give an explicit example for each case,
state whether the column vectors you use are linearly independent or spanning in each case.
 Then, in general, determine whether $(v_1,v_2,\ldots,v_m)$ are linearly independent and/or spanning ${\mathbb R}^n$
 in each of the three cases. If they are linearly dependent, does $\rref(M)$ tell you which vectors could be 
 removed to yield an independent set of vectors?
\end{enumerate}

