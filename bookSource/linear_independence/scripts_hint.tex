
\subsection*{Hint for Review Problem~\ref{bitsprob}}

%%%Insert this to get the typewriter font so it looks like a real movie script
{\ttfamily
\fontdimen2\font=0.4em
\fontdimen3\font=0.2em
\fontdimen4\font=0.1em
\fontdimen7\font=0.1em
\hyphenchar\font=`\-


\hypertarget{linear_independence_hint}{Lets first remember how} $\mathbb{Z}_2$ works. The only two elements are 1 and 0. Which means when you add $1+1$ you get $0$. It also means when you have a vector $\vec{v} \in B^n$ and you want to multiply it by a scalar, your only choices are 1 and 0. This is kind of neat because it means that the possibilities are finite, so we can look at an entire vector space.

Now lets think about $B^3$ there is choice you have to make for each coordinate, you can either put a 1 or a 0, there are three places where you have to make a decision between two things. This means that you have $2^3= 8$ possibilities for vectors in $B^3$.

When you want to think about finding a set $S$ that will span $B^3$ and is linearly independent, you want to think about how many vectors you need. You will need you have enough so that you can make every vector in $B^3$ using linear combinations of elements in $S$ but you don't want too many so that some of them are linear combinations of each other. I suggest trying something really simple perhaps something that looks like the columns of the identity matrix

For part (c) you have to show that you can write every one of the elements as a linear combination of the elements in $S$, this will check to make sure $S$ actually spans $B^3$. 

For part (d) if you have two vectors that you think will span the space, you can prove that they do by repeating what you did in part (c), check that every vector can be written using only copies of of these two vectors. If you don't think it will work you should show why, perhaps using an argument that counts the number of possible vectors in the span of two vectors.




} % Closing bracket for font

%\newpage
