
\subsection*{Worked Example}

%%%Insert this to get the typewriter font so it looks like a real movie script
{\ttfamily
\fontdimen2\font=0.4em
\fontdimen3\font=0.2em
\fontdimen4\font=0.1em
\fontdimen7\font=0.1em
\hyphenchar\font=`\-


\hypertarget{linear_independence_example}{This video} gives some more details behind the example for the following four vectors in ${\mathbb R}^3$
Consider the following vectors in \(\mathbb{R}^3\):
\[
v_1=\colvec{4\\-1\\3}, \qquad
v_2=\colvec{-3\\7\\4}, \qquad
v_3=\colvec{5\\12\\17}, \qquad
v_4=\colvec{-1\\1\\0}.
\]
The example asks whether they are linearly independent, and the answer is immediate: NO, four vectors can never be linearly independent in ${\mathbb R}^3$. This vector space is simply not big enough for that, but you need to understand the notion of the dimension of a vector space to see why.
So we think the vectors $v_1$, $v_2$, $v_3$ and $v_4$ are linearly dependent, which means we need to show that there is a solution to 
\[
\alpha_1 v_1 + \alpha_2 v_2 + \alpha_3 v_3 + \alpha_4 v_4 = 0
\]
for the numbers $\alpha_1$, $\alpha_2$, $\alpha_3$ and $\alpha_4$  not all vanishing. 

To find this solution we need to set up a linear system. Writing out the above linear combination gives 
\[
\begin{array}{cccccc}
4\alpha_1&-3\alpha_2&+5\alpha_3&-\alpha_4 &=&0\, ,\\
-\alpha_1&+7\alpha_2&+12\alpha_3&+\alpha_4 &=&0\, ,\\
3\alpha_1&+4\alpha_2&+17\alpha_3& &=&0\, .\\
\end{array}
\]
This can be easily handled using an augmented matrix whose columns are just the vectors we started with
\[
\left(
\begin{array}{cccc|c}
4&-3&5&-1 &0\, ,\\
-1&7&12&1 &0\, ,\\
3&4&17& 0&0\, .\\
\end{array}\right)\, .
\]
Since there are only zeros on the right hand column, we can drop it. Now we perform row operations to achieve RREF
\[
\begin{pmatrix}
4&-3&5&-1 \\
-1&7&12&1 \\
3&4&17& 0\\
\end{pmatrix}\sim
\begin{pmatrix}
1 & 0 & \frac{71}{25}& -\frac 4{25}\\[1mm]
0&1&\frac{53}{25}&\frac 3{25}\\[1mm]
0&0&0&0
\end{pmatrix}\, .
\]
This says that $\alpha_3$ and $\alpha_4$ are not pivot variable so are arbitrary, we set them to $\mu$ and $\nu$, respectively.
Thus
\[
\alpha_1=\Big(-\frac{71}{25}\, \mu+\frac4{25}\, \nu\Big)\, ,\qquad \alpha_2=\Big(-\frac{53}{25}\, \mu-\frac{3}{25}\, \nu\Big)\, ,\qquad
\alpha_3=\mu\, ,\qquad \alpha_4= \nu\, .
\]
Thus we have found a relationship among our four vectors
\[
\Big(-\frac{71}{25}\, \mu+\frac4{25}\, \nu\Big)\, v_1+\Big(-\frac{53}{25}\, \mu-\frac{3}{25}\, \nu\Big)\, v_2
+\mu\,  v_3+ \mu_4\,  v_4=0\, .
\]
In fact this is not just one relation, but infinitely many, for any choice of $\mu,\nu$. 
The relationship quoted in the notes is just one of those choices.

Finally, since the vectors $v_1$, $v_2$, $v_3$ and $v_4$ are linearly dependent, we can try to eliminate some of them.
The pattern here is to keep the vectors that correspond to columns with pivots. For example, setting $\mu=-1$ (say) and $\nu=0$ 
in the above allows us to solve for $v_3$ while $\mu=0$ and $\nu=-1$ (say) gives $v_4$, explicitly we get
\[
v_3=\frac{71}{25}\,  v_1 + \frac{53}{25}\, v_2\, ,\qquad 
v_4=-\frac{4}{25}\, v_3 + \frac 3{25} \, v_4\, .
\]
This eliminates $v_3$ and $v_4$ and leaves a pair of linearly independent vectors $v_1$ and $v_2$.




} % Closing bracket for font

%\newpage
