



\begin{enumerate}

\item Find formulas for the inverses of the following matrices, when they are not singular:
\begin{enumerate}
\item $\begin{pmatrix}
1 & a & b \\
0 & 1 & c \\
0 & 0 & 1 \\
\end{pmatrix}$
\item $\begin{pmatrix}
a & b & c \\
0 & d & e \\
0 & 0 & f \\
\end{pmatrix}$
\end{enumerate}
When are these matrices singular?

\phantomnewpage

\item Write down all $2\times 2$ bit matrices and decide which of them are singular.  For those which are not singular, pair them with their inverse.

\phantomnewpage

\item \label{problem_unique_solution} Let $M$ be a square matrix.  Explain why the following statements are equivalent:
\begin{enumerate}
\item $MX=V$ has a \emph{unique} solution for every column vector $V$.
\item $M$ is non-singular.
\end{enumerate}
Hint: In general for problems like this, think about the key words:

First, suppose that there is some column vector \(V\) such that the equation \(MX=V\) has two distinct solutions. Show that \(M\) must be singular; that is, show that \(M\) can have no inverse.

Next, suppose that there is some column vector \(V\) such that the equation \(MX=V\) has no solutions. Show that \(M\) must be singular.

Finally, suppose that \(M\) is non-singular. Show that no matter what the column vector \(V\) is, there is a unique solution to \(MX=V.\)
%Show that $(i)\Rightarrow (ii)$ and $(ii)\Rightarrow (i)$.)

\Videoscriptlink{inverse_matrix_unique_solution.mp4}{Hint}{scripts_inverse_matrix_unique_soln}

\phantomnewpage

\item \label{leftright}{\em Left and Right Inverses:} So far we have only talked about inverses of square matrices. This problem will explore the notion of
a left and right inverse for a matrix that is not square. Let
$$
A=\begin{pmatrix}0 & 1 & 1 \\ 1&1&0\end{pmatrix}
$$
\begin{enumerate}
\item Compute: 
\begin{enumerate}
\item $A A^T$,
\item $\big(A A^T\big)^{-1}$,
\item $B:=A^T \big(A A^T\big)^{-1}$
\end{enumerate}

\item Show that the matrix $B$ above is a {\it right inverse} for $A$, {\it i.e.}, verify that
$$
AB=I\, .
$$
\item Is $BA$ defined? (Why or why not?)

\item Let $A$ be an $n\times m$ matrix with $n>m$. Suggest a formula for a left inverse $C$
such that
$$
CA=I
$$
{\it Hint: you may assume that $A^TA$ has an inverse.}

\item Test your proposal for a left inverse for the simple example
$$
A=\begin{pmatrix}1\\2\end{pmatrix}\, ,
$$
\item True or false: Left and right inverses are unique. If false give a counterexample.
\end{enumerate}

\Videoscriptlink{inverse_matrix_hint.mp4}{Hint}{inverse_matrix_hint}

\item Show that if the range (remember that the range of a function is the set of all its outputs, not the codomain) of a $3\times3$ matrix $M$ (viewed as a function ${\mathbb R}^3\to {\mathbb R}^3$) is a plane then one of the columns is a sum of multiples of the other columns. Show that this relationship is preserved under EROs. Show, further, that the solutions to $Mx=0$ describe this relationship between the columns. 





\item If $M$ and $N$ are square matrices of the same size such that  $M^{-1}$ exists and $N^{-1}$ does not exist, does $(MN)^{-1}$ exist? 

\item If $M$ is a square matrix which is not invertible, is $e^{M}$ invertible? 

\item Elementary Column Operations  (ECOs) can be defined in the same 3 types as EROs. Describe the 3 kinds of ECOs. Show that if maximal elimination using ECOs is performed on a square matrix and a column of zeros is obtained then that matrix is not invertible. 
















\end{enumerate}
\phantomnewpage
