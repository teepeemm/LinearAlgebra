
\chapter{\diagSymMatTitle}\label{symmetricmatrices}

Symmetric matrices have many applications.  For example, if we consider the shortest distance between pairs of important cities, we might get a table like the following.
\[
\begin{array}{c|ccc}
 & \text{Davis} & \text{Seattle} 
& \text{San Francisco} \\ \hline
\text{Davis} & 0 & 2000 & 80 \\
\text{Seattle} & 2000 & 0 & 2010 \\
\text{San Francisco} & 80 & 2010 & 0
\end{array}
\]
Encoded as a matrix, we obtain
\[
M=\begin{pmatrix}
\mc{0} & \mc{2000} & \mc{80} \\
\mc{2000} & \mc{0} & \mc{2010} \\
\mc{80} & \mc{2010} & \mc{0}
\end{pmatrix}=M^T.
\]

\begin{definition}
A matrix $M$ is {\bf symmetric}\index{Symmetric matrix} if  $M^T=M.$
\end{definition}

One very nice property of symmetric matrices is that they always have real eigenvalues.  Review exercise~\ref{prob_real_eigenvalues} guides you through the general proof, but below is an example for $2\times 2$ matrices.

\begin{example}
For a general symmetric $2\times 2$ matrix, we have:
\begin{eqnarray*}
P_\lambda \begin{pmatrix} a & b \\ b& d \end{pmatrix}
 &=&
\det\begin{pmatrix}\lambda-a&\mc{-b}\\\mc{-b}&\lambda-d \end{pmatrix}\\[1mm]
&=& (\lambda-a)(\lambda-d)-b^2 \\[2mm]
&=& \lambda^2-(a+d)\lambda-b^2+ad\\[1mm]
\Rightarrow \lambda &=& \frac{a+d}{2}\pm \sqrt{b^2+\left(\frac{a-d}{2}\right)^2}.
\end{eqnarray*}
Notice that the discriminant $4b^2+(a-d)^2$ is always positive, so that the eigenvalues must be real.
\end{example}

Now, suppose a symmetric matrix $M$ has two distinct eigenvalues $\lambda \neq \mu$ and eigenvectors $x$ and $y$;
\[
Mx=\lambda x, \qquad My=\mu y.
\] 
Consider the dot product $x\dotprod y = x^Ty = y^Tx$ and calculate
\begin{eqnarray*}
x^TM y &=& x^T\mu y = \mu x\dotprod y, \text{ and }\\[3mm]
x^TM y &=& (y^TMx)^T \text{ (by transposing a $1\times 1$ matrix)}\\[1mm]
       &=& (y^T\lambda x)^T \\
       &=& (\lambda x\dotprod y)^T \\
             &=& \lambda x\dotprod y.
\end{eqnarray*}
Subtracting these two results tells us that:
\begin{eqnarray*}
0 &=& x^TMy-x^TMy=(\mu-\lambda)\,x\dotprod y.
\end{eqnarray*}
Since $\mu$ and $\lambda$ were assumed to be distinct eigenvalues, $\lambda-\mu$ is non-zero, and so $x\dotprod y=0$.  We have proved the following theorem.

\begin{theorem}
Eigenvectors of a symmetric matrix with distinct eigenvalues are orthogonal.
\end{theorem}

%\begin{center}\href{\webworkurl ReadingHomework23/1/}{Reading homework: problem \ref{symmetricmatrices}.1}\end{center}
\Reading{DiagonalizingSymmetricMatrices}{1}

\begin{example}
The matrix $M=\begin{pmatrix}2&1\\1&2\end{pmatrix}$
has eigenvalues determined by
\[
\det(M-\lambda I)=(2-\lambda)^2-1=0.
\] 
So the eigenvalues of $M$ are $3$ and $1$, and the associated eigenvectors turn out to be 
$\colvec{1\\1}$ and $\colvec{1\\-1}$.  It is easily seen that these eigenvectors are \hyperref[orthogonal]{orthogonal}; 
\[
\colvec{1\\1} \dotprod \colvec{1\\-1}=0.
\]
\end{example}

In \hyperlink{basisorthog}{chapter~\ref{orthonormalbases}} we saw that the matrix $P$ built from any orthonormal basis  $(v_1,\ldots, v_n )$
for ${\mathbb R}^n$ as its columns,
\[
P=\rowvec{v_1 & \cdots & v_n}\, ,
\]
was an orthogonal matrix. This means that 
\[
P^{-1}=P^T, \text{ or } PP^T=I=P^TP.
\]
Moreover, given any (unit) vector $x_1$, one can always find vectors $x_2, \ldots, x_n$ such that $(x_1,\ldots, x_n)$ is an orthonormal basis.  (Such a basis can be obtained using the~\hyperref[GramSchmidt]{Gram-Schmidt procedure}.)

Now suppose $M$ is a symmetric $n\times n$ matrix and $\lambda_1$ is an eigenvalue with eigenvector $x_1$ (this is always the case because every matrix has at least one eigenvalue--see Review Problem~\ref{atleastone}).  
Let $P$ be the square matrix of orthonormal column vectors 
\[
P=\rowvec{x_1 & x_2 & \cdots & x_n},
\]
While $x_1$ is an eigenvector for $M$, the others are not necessarily eigenvectors for $M$.  
Then
\[
MP=\rowvec{\lambda_1 x_1 & Mx_2 & \cdots & Mx_n}.
\]
But $P$ is an orthogonal matrix, so $P^{-1}=P^T$.  Then:
\begin{eqnarray*}
P^{-1}=P^T &=& \ccolvec{x_1^T\\ \vdots \\ x_n^T} \\[1mm]
\Rightarrow P^TMP &=& \begin{pmatrix}
  x_1^T\lambda_1x_1  & * & \cdots & *\\
  x_2^T\lambda_1x_1  & * & \cdots & *\\
  \mc\vdots             &   & & \mc\vdots\\
   x_n^T\lambda_1x_1 & * & \cdots & *\\
  \end{pmatrix}\\[2mm]
&=& \begin{pmatrix}
  \lambda_1  & * & \cdots & *\\
  \mc 0  & * & \cdots & *\\
 \mc \vdots             & *  & & \mc\vdots\\
  \mc 0 & * & \cdots & *\\
  \end{pmatrix}\\[2mm]
&=& \begin{pmatrix}
  \lambda_1  & 0 & \cdots & 0\\
  \mc 0          & & & \\
  \mc\vdots     & & \hat{M} & \\
  \mc0          & & & \\
  \end{pmatrix}\, .\\
\end{eqnarray*}
The last equality follows since $P^TMP$ is symmetric.  The asterisks in the matrix are where ``stuff'' happens; this extra information is denoted by $\hat{M}$ in the final expression.  We know nothing about $\hat{M}$ except that it is an $(n-1)\times (n-1)$ matrix and that it is symmetric.  But then, by finding an (unit) eigenvector for $\hat{M}$, we could repeat this procedure successively.  The end result would be a diagonal matrix with eigenvalues of $M$ on the diagonal. Again, we have proved a theorem: %we also need that every matrix has an eigenvector.

\begin{theorem}
Every symmetric matrix is similar to a diagonal matrix of its eigenvalues.  In other words,
\[
M=M^T \Leftrightarrow M=PDP^T
\]
where $P$ is an orthogonal matrix and $D$ is a diagonal matrix whose entries are the eigenvalues of $M$.
\end{theorem}

%\begin{center}\href{\webworkurl ReadingHomework23/2/}{Reading homework: problem \ref{symmetricmatrices}.2}
\Reading{DiagonalizingSymmetricMatrices}{2}
%\end{center}

To diagonalize a real symmetric matrix, begin by building an orthogonal matrix from an orthonormal basis of eigenvectors, as in the example below. 

\begin{example}
The symmetric matrix 
$$M=\begin{pmatrix}2&1\\1&2\end{pmatrix}\,  ,$$ has eigenvalues $3$ and $1$ with eigenvectors $\colvec{1\\1}$ and $\colvec{1\\-1}$ respectively.  After normalizing these eigenvectors, we  build the orthogonal matrix:
\[
P = \begin{pmatrix}
\frac{1}{\sqrt{2}} & \frac{1}{\sqrt{2}} \\[2mm]
\frac{1}{\sqrt{2}} & \frac{-1}{\sqrt{2}}
\end{pmatrix}\, .
\]
Notice that $P^TP=I$.  Then:
\[
MP = \begin{pmatrix}
\frac{3}{\sqrt{2}} & \frac{1}{\sqrt{2}} \\[2mm]
\frac{3}{\sqrt{2}} & \frac{-1}{\sqrt{2}}
\end{pmatrix} = 
\begin{pmatrix}
\frac{1}{\sqrt{2}} & \frac{1}{\sqrt{2}} \\[2mm]
\frac{1}{\sqrt{2}} & \frac{-1}{\sqrt{2}}
\end{pmatrix} \begin{pmatrix}
3 & 0 \\[2mm]
0 & 1
\end{pmatrix}.
\]
In short, $MP=PD$, so $D=P^TMP$.  Then $D$ is the diagonalized form of $M$ and $P$ the associated change-of-basis matrix from the standard basis to the basis of eigenvectors.
\end{example}

\Videoscriptlink{diagonalizing_symmetric_matrices_3by3_example.mp4}{ $3\times 3$ Example}{scripts_diagonalizing_symmetric_matrices_3by3_example}












%\section*{References}
%Hefferon, Chapter Three, Section V: Change of Basis
%\\
%Beezer, Chapter E, Section PEE, Subsection EHM
%\\
%Beezer, Chapter E, Section SD, Subsection D
%\\
%Wikipedia:
%\begin{itemize}
%\item \href{http://en.wikipedia.org/wiki/Symmetric_matrix}{Symmetric Matrix}
%\item \href{http://en.wikipedia.org/wiki/Diagonalizable_matrix}{Diagonalizable Matrix}
%\item \href{http://en.wikipedia.org/wiki/Similar_matrix}{Similar Matrix}
%\end{itemize}

\section{Review Problems}

{\bf Webwork:} 
\begin{tabular}{|c|c|}
\hline
Reading Problems & 
 \hwrref{DiagonalizingSymmetricMatrices}{1}, 
 \hwrref{DiagonalizingSymmetricMatrices}{2}, 
 \\
Diagonalizing a symmetric matrix &  \hwref{DiagonalizingSymmetricMatrices}{3}, \hwref{DiagonalizingSymmetricMatrices}{4}\\
   \hline
\end{tabular}








\begin{enumerate}
\item \label{det33} Let $M=\begin{pmatrix}
m^1_1 & m^1_2 & m^1_3\\
m^2_1 & m^2_2 & m^2_3\\
m^3_1 & m^3_2 & m^3_3\\
\end{pmatrix}$.  Use row operations to put $M$ into \emph{row echelon form}.  For simplicity, assume that $m_1^1\neq 0 \neq m^1_1m^2_2-m^2_1m^1_2$.

Prove that $M$ is non-singular if and only if:
\[
m^1_1m^2_2m^3_3 
- m^1_1m^2_3m^3_2 
+ m^1_2m^2_3m^3_1 
- m^1_2m^2_1m^3_3 
+ m^1_3m^2_1m^3_2
- m^1_3m^2_2m^3_1
\neq 0
\]

\phantomnewpage

\item 
\begin{enumerate}
\item What does the matrix $E^1_2=\begin{pmatrix}
0 & 1 \\
1 & 0
\end{pmatrix}$ do to $M=\begin{pmatrix}
a & b \\
d & c
\end{pmatrix}$ under left multiplication?  What about right multiplication?
\item Find elementary matrices $R^1(\lambda)$ and $R^2(\lambda)$ that respectively multiply rows $1$ and $2$ of $M$ by $\lambda$ but otherwise leave $M$ the same under left multiplication.
\item Find a matrix $S^1_2(\lambda)$ that adds a multiple $\lambda$ of row $2$ to row $1$ under left multiplication.
\end{enumerate}

\phantomnewpage

\item Let $M$ be a matrix and $S^i_jM$ the same matrix with rows \(i\) and \(j\) switched.  Explain every line of the 
\hyperlink{rowswap}{series of equations} proving that $\det M = -\det (S^i_jM)$.

\phantomnewpage

%\item \label{prob_inversion_number} This problem is a ``hands-on'' look at why \hyperlink{permutation_parity}{the property} describing the parity of permutations is true.
%
%\hypertarget{inversion_number}{The \emph{inversion number}}\index{Permutation!Inversion number} of a permutation $\sigma$ is the number of pairs $i<j$ such that $\sigma(i)>\sigma(j)$; it's the number of ``numbers that appear left of smaller numbers'' in the permutation.  For example, for the permutation $\rho = [4,2,3,1]$, the inversion number is $5$. The number $4$ comes before $2,3,$ and $1$, and $2$ and $3$ both come before $1$.
%
%Given a permutation $\sigma$, we can make a new permutation $\tau_{i,j} \sigma$ by exchanging the $i$th and $j$th entries of $\sigma$.
%
%\begin{enumerate}
%\item What is the inversion number of the permutation \(\mu=[1,2,4,3]\) that exchanges 4 and 3 and leaves everything else alone? Is it an even or an odd permutation?
%
%\item What is the inversion number of the permutation \(\rho=[4,2,3,1]\) that exchanges 1 and 4 and leaves everything else alone? Is it an even or an odd permutation?
%
%\item What is the inversion number of the permutation \(\tau_{1,3} \mu\)? Compare the parity\footnote{The \emph{parity} of an integer refers to whether the integer is even or odd. Here the parity of a permutation $\mu$ refers to the parity of its inversion number.} of \(\mu\) to the parity of \(\tau_{1,3} \mu.\)
%
%\item What is the inversion number of the permutation \(\tau_{2,4} \rho\)? Compare the parity of \(\rho\) to the parity of \(\tau_{2,4} \rho.\)
%
%\item What is the inversion number of the permutation \(\tau_{3,4} \rho\)? Compare the parity of \(\rho\) to the parity of \(\tau_{3,4} \rho.\)
%\end{enumerate}
%
%\videoscriptlink{elementary_matrices_determinant_hint.mp4}{Problem~\ref{prob_inversion_number} hints}{scripts_elementary_matrices_determinants_hint}

\phantomnewpage

%\item \label{problem_permutation} (Extra credit) Here we will examine a (very) small set of the general properties about permutations and their applications. In particular, we will show that one way to compute the sign of a permutation is by finding the \hyperlink{inversion_number}{inversion number} $N$ of $\sigma$ and we have
%\[
%\sgn(\sigma) = (-1)^N.
%\]
%
%For this problem, let $\mu = [1,2,4,3]$.
%
%\begin{enumerate}
%\item Show that every permutation $\sigma$ can be sorted by only taking simple (adjacent) transpositions\index{Permutation!Simple transposition} $s_i$ where $s_i$ interchanges the numbers in position $i$ and $i+1$ of a permutation $\sigma$ (in our other notation $s_i = \tau_{i,i+1}$). For example $s_2 \mu = [1, 4, 2, 3]$, and to sort $\mu$ we have $s_3 \mu = [1, 2, 3, 4]$.
%
%\item \label{prob_part_relations} We can compose simple transpositions together to represent a permutation (note that the sequence of compositions is not unique), and these are associative, we have an identity (the trivial permutation where the list is in order or we do nothing on our list), and we have an inverse since it is clear that $s_i s_i \sigma = \sigma$. Thus permutations of $[n]$ under composition are an example of a \hyperref[groups]{group}. However note that not all simple transpositions commute with each other since
%\begin{align*}
%s_1 s_2 [1, 2, 3] & = s_1 [1, 3, 2] = [3, 1, 2]
%\\ s_2 s_1 [1, 2, 3] & = s_2 [2, 1, 3] = [2, 3, 1]
%\end{align*}
%(you will prove here when simple transpositions commute). When we consider our initial permutation to be the trivial permutation $e = [1, 2, \dotsc, n]$, we do not write it; for example $s_i \equiv s_i e$ and $\mu = s_3 \equiv s_3 e$. This is analogous to not writing 1 when multiplying. Show that $s_i s_i = e$ (in shorthand $s_i^2 = e$), $s_{i+1} s_i s_{i+1} = s_i s_{i+1} s_i$ for all $i$, and $s_i$ and $s_j$ commute for all $|i - j| \geq 2$.
%
%\item Show that every way of expressing $\sigma$ can be obtained from using the relations proved in part~\ref{prob_part_relations}. In other words, show that for any expression $w$ of simple transpositions representing the trivial permutation $e$, using the proved relations.
%
%\emph{Hint: Use induction on $n$. For the induction step, follow the path of the $(n+1)$-th strand by looking at $s_n s_{n-1} \cdots s_k s_{k\pm1} \cdots s_n$ and argue why you can write this as a subexpression for any expression of $e$. Consider using diagrams of these paths to help.}
%
%\item The simple transpositions \hyperlink{action}{acts on} an $n$-dimensional vector space $V$ by $s_i v = E^i_{i+1} v$ (where $E^i_j$ is \hyperlink{elem_matrix_row_swap}{an elementary matrix}) for all vectors $v \in V$. Therefore we can just represent a permutation $\sigma$ as the matrix $M_{\sigma}$\footnote{Often people will just use $\sigma$ for the matrix when the context is clear.}, and we have $\det(M_{s_i}) = \det(E^i_{i+1}) = -1$. Thus prove that $\det(M_{\sigma}) = (-1)^N$ where $N$ is a number of simple transpositions needed to represent $\sigma$ as a permutation. You can assume that $M_{s_i s_j} = M_{s_i} M_{s_j}$ (it is not hard to prove) and that $\det(A B) = \det(A) \det(B)$ \hyperref[detmultiplicative]{from Chapter~\ref*{elementarydeterminantsII}}.
%
%\emph{Hint: You to make sure $\det(M_{\sigma})$ is well-defined since there are infinite ways to represent $\sigma$ as simple transpositions.}
%
%\item Show that $s_{i+1} s_i s_{i+1} = \tau_{i, i+2}$, and so give one way of writing $\tau_{i, j}$ in terms of simple transpositions? Is $\tau_{i,j}$ an even or an odd permutation? What is $\det(M_{\tau_{i,j}})$? What is the inversion number of $\tau_{i,j}$?
%
%\item The minimal number of simple transpositions needed to express $\sigma$ is called the \emph{length}\index{Permutation!Length} of $\sigma$; for example the length of $\mu$ is 1 since $\mu = s_3$. Show that the length of $\sigma$ is equal to the inversion number of $\sigma$.
%
%\emph{Hint: Find an procedure which gives you a new permutation $\sigma^{\prime}$ where $\sigma = s_i \sigma^{\prime}$ for some $i$ and the inversion number for $\sigma^{\prime}$ is 1 less than the inversion number for $\sigma$.}
%
%\item Show that $(-1)^N = \sgn(\sigma) = \det(M_{\sigma})$, where $\sigma$ is a permutation with $N$ inversions. Note that this immediately implies that $\sgn(\sigma \rho) = \sgn(\sigma) \sgn(\rho)$ for any permutations $\sigma$ and $\rho$.
%\end{enumerate}

\item Let $M'$ be the matrix obtained from $M$ by swapping two columns $i$ and $j$. Show that $\det M'=-\det M $.

\item The scalar triple product of three vectors $u,v,w$ from $\Re^3$ is $u\cdot(v\times w)$. Show that this product is the same as the determinant of the matrix whose columns are $u,v,w$ (in that order). What happens to the scalar triple product when the factors are permuted? 

\item Show that if $M$ is a $3\times 3$ matrix whose third row is a sum of multiples of the other rows ($R_3=aR_2+bR_1$) then $\det M=0$. Show that the same is true if one of the columns is a sum of multiples of the others. 

\end{enumerate}

\phantomnewpage

%\newpage
