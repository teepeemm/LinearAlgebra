
\subsection*{Hint for Review Problem~\ref{prob_real_eigenvalues}}

%%%Insert this to get the typewriter font so it looks like a real movie script
{\ttfamily
\fontdimen2\font=0.4em
\fontdimen3\font=0.2em
\fontdimen4\font=0.1em
\fontdimen7\font=0.1em
\hyphenchar\font=`\-


\hypertarget{scripts_diagonalizing_symmetric_matrices_hint}{For part~(a), we can consider} any complex number $z$ as being a vector in $\mathbb{R}^2$ where complex conjugation corresponds to the matrix $\begin{pmatrix} 1 & 0 \\ 0 & -1 \end{pmatrix}$. Can you describe $z \bar{z}$ in terms of $\norm{z}$? For part~(b), think about what values $a \in \mathbb{R}$ can take if $a = -a$? Part~(c), just compute it and look back at part~(a).

For part~(d), note that $x^{\dagger} x$ is just a number, so we can divide by it. Parts~(e) and~(f) follow right from definitions. For part~(g), first notice that every row vector is the (unique) transpose of a column vector, and also think about why $(A A^T)^T = A A^T$ for any matrix $A$. Additionally you should see that $\overline{x^T} = x^{\dagger}$ and mention this. Finally for part~(h), show that
\[
\frac{x^{\dagger} M x}{x^{\dagger} x} = \overline{\left(\frac{x^{\dagger} M x}{x^{\dagger} x}\right)^T}
\]
and reduce each side separately to get $\lambda = \overline{\lambda}$.

} % Closing bracket for font

%\newpage
