


\begin{enumerate}
\item\label{prob_real_eigenvalues} (On Reality of Eigenvalues) \begin{enumerate}
\item\label{stab}
Suppose $z=x+iy$ where $x,y \in \mathbb{R}, i=\sqrt{-1}$, and $\overline{z}=x-iy$. Compute $z\overline{z}$ and $\overline{z}z$ in terms of \(x\) and \(y\). What kind of numbers are \(z \overline{z}\) and \(\overline{z}z\)? (The complex number \(\overline{z}\) is called the \emph{complex conjugate} of \(z\)).

\item Suppose that \(\lambda=x+iy\) is a complex number with \(x,y \in \mathbb{R}\), and that $\lambda=\overline{\lambda}$. Does this determine the value of \(x\) or \(y\)? What kind of number must \(\lambda\) be?

\item Let $x=\colvec{z^1\\ \vdots \\ z^n}\in \C^n$.  Let $x^\dagger=\rowvec{\overline{z^1}& \cdots & \overline{z^n}} \in \mathbb{C}^n$ (a $1 \times n$ complex matrix or a row vector).  Compute $x^\dagger x$.  Using the result of part~\ref{stab}, what can you say about the number $x^\dagger x$? ({\itshape E.g.,} is it real, imaginary, positive, negative, etc.)

\item Suppose $M=M^T$ is an $n\times n$ symmetric matrix with real entries.  Let $\lambda$ be an eigenvalue of $M$ with eigenvector $x$, so $Mx=\lambda x$. Compute:
\[
\frac{x^\dagger Mx}{x^\dagger x}
\]

\item Suppose $\Lambda$ is a $1\times 1$ matrix.  What is $\Lambda^T$?

\item What is the size of the matrix $x^\dagger Mx$?

\item For any matrix (or vector) $N$, we can compute $\overline{N}$ by applying complex conjugation to each entry of $N$.
Compute $\overline{(x^\dagger)^T}$.  
Then compute $\overline{(x^\dagger M x)^T}$.
Note that for matrices $\overline{AB + C} = \overline{A} \overline{B} + \overline{C}$.

\item Show that $\lambda=\overline{\lambda}$.  Using the result of a previous part of this problem, what does this say about $\lambda$?
\end{enumerate}

\Videoscriptlink{diagonalizing_symmetric_matrices_hint.mp4}{Hint}{scripts_diagonalizing_symmetric_matrices_hint}

\phantomnewpage

\item Let \[x_1=\colvec{a\\ b \\ c}\, ,\] where $a^2+b^2+c^2=1$.  Find vectors $x_2$ and $x_3$ such that $\{x_1,x_2,x_3\}$ is an orthonormal basis for $\mathbb{R}^3$. What can you say about the matrix $P$ whose columns are the vectors $x_1$, $x_2$ and $x_3$ that you found?

\item~\label{atleastone} Let $V\ni v\neq0$ be a vector space, $\dimension V=n$ and $L:V\stackrel{\textrm{linear}}{-\!\!-\!\!\!\longrightarrow}V$.
\begin{enumerate}
\item Explain why the list of vectors $(v,Lv,L^2v,\ldots,L^n v)$ is linearly dependent.
\item Explain why there exist scalars $\alpha_i$ not all zero such that
\[
\alpha_0 v + \alpha_1 L v+\alpha_2 L^2 v+\cdots + \alpha_n L^n v=0\, .
\]
\item Let $m$ be the largest integer such that $\alpha_m\neq0$ and \[p(z)=\alpha_0+ \alpha_1 z + \alpha_2 z^2+\cdots + \alpha_m z^n \, .\]
Explain why the polynomial $p(z)$ can be written as
\[
p(z)=\alpha_m (z-\lambda_1)(z-\lambda_2)\ldots(z-\lambda_{m})\, .
\]
[Note that some of the roots $\lambda_i$ could be complex.]
\item Why does the following equation hold
\[
(L-\lambda_1)(L-\lambda_2)\ldots(L-\lambda_{m})
 v=0\, ?
\]
\item Explain why one of the numbers $\lambda_i$ ($1\leq i\leq m$) must be an eigenvalue of $L$.
\end{enumerate}



\phantomnewpage

\item (Dimensions of Eigenspaces) \begin{enumerate}
\item Let \[A=
\begin{pmatrix}
4 & 0 & 0 \\
0 & 2 & -2 \\
0 & -2 & 2 \\
\end{pmatrix}\, .\]
Find all eigenvalues of \(A.\)
\item Find a basis for each eigenspace of \(A.\)  What is the sum of the dimensions of the eigenspaces of \(A\)?
\item Based on your answer to the previous part, guess a formula for the sum of the dimensions of the eigenspaces of a real \(n \times n\) symmetric matrix. Explain why your formula must work for any real \(n \times n\) symmetric matrix.
\end{enumerate}




\item If $M$ is not square then it can not be symmetric. However, $MM^T$ and $M^TM$ are symmetric, and therefore diagonalizable. 
\begin{enumerate}
\item Is it the case that all of the eigenvalues of $MM^T$ must also be eigenvalues of $M^TM$? 
\item Given an eigenvector of $MM^T$ how can you obtain an eigenvector of $M^TM$?
\item Let \[M=\begin{pmatrix}1&2\\3&3\\2&1\end{pmatrix}\, .\]
Compute an orthonormal basis of eigenvectors for both $MM^T$ and $M^TM$. If any of the eigenvalues for these two matrices agree,
choose an order for them and use it to help order your orthonormal bases. Finally, change the input and output bases
for the matrix~$M$ to these ordered orthonormal bases. Comment on what you find. ({\itshape Hint:} 
The result is called \emph{the Singular Value Decomposition Theorem}\index{Singular values}.) 
\end{enumerate}












\end{enumerate}

\phantomnewpage