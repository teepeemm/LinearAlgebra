

\begin{enumerate}


\item State whether the  following augmented matrices are in RREF and compute their solution sets.
\begin{gather*}\allowdisplaybreaks
\begin{amatrix}{5} 
1 & 0 & 0 & 0 & 3 & 1\\ 
0 & 1& 0 & 0 & 1 & 2\\ 
0 & 0 & 1 & 0 & 1 & 3\\ 
0 & 0 & 0 & 1 & 2 & 0
\end{amatrix}\, ,
\\
\begin{amatrix}{6} 
1 & 1 & 0 & 1 & 0 & 1&0\\ 
0 & 0 & 1 & 2 & 0 & 2&0\\ 
0 & 0 & 0 & 0 & 1 & 3& 0\\ 
0 & 0 & 0 & 0 & 0 & 0&0
\end{amatrix}\, ,
\\
\begin{amatrix}{7} 
1 & 1 & 0 & 1 & 0 & 1 &0 &1\\ 
0 & 0 & 1 & 2 & 0 & 2& 0&-1\\ 
0 & 0 & 0 & 0 & 1 & 3 & 0&1\\ 
0 & 0 & 0 & 0 & 0 & 2 & 0&-2\\
0 &0&   0&  0& 0  & 0 & 1 &1
\end{amatrix}\, .
\end{gather*}

%%%%%%%%%%%%%%%%%%%%%%%%%%%%%%

\item Solve the following linear system:
\[
\begin{linsys}{7}
               2x_1 &  +\ 5x_2 &  -\ 8x_3  & +\ 2 x_4 &+\ 2x_5 &=&0  \\[1mm]
               6x_1 &  +\ 2x_2 &  -10x_3  & +\ 6 x_4 &+\ 8x_5 &=&6  \\[1mm]
               3x_1 &  +\ 6x_2 &  +\ 2x_3  & +\ 3 x_4 &+\ 5x_5 &=&6  \\[1mm]
                3x_1 &  +\ 1x_2 &  -\ 5x_3  & +\ 3 x_4 &+\ 4x_5 &=&3  \\[1mm]
                6x_1 &  +\ 7x_2 &  -\ 3x_3  & +\ 6 x_4 &+\ 9x_5 &=&9 
      \end{linsys}
\]
 Be sure to set your work out carefully with equivalence signs $\sim$ between
each step, labeled by the row operations you performed.

%%%%%%%%%%%%%%%%%%%%%%%%%%
%\item Set notation for solution sets \\
%Consider the system of equations associated with the augmented matrix
%\[\begin{amatrix}{4} 
%1 &  0 & 1 & -1 & 1\\ 
% 0 & 1 & -1& 1  & 1\\
% 0 &0   & 0  & 0   &0 \\
%\end{amatrix}\]
%Here $x_3$ and $x_4$ are non-pivot variables.  
%The solutions are then of the form $x_3=\mu_1$, $x_4=\mu_2$, $x_2=1 + \mu_1 - \mu_2$, $x_1=1-\mu_1+\mu_2$.
%%
%The preferred way to write a solution set is with set notation\index{Solution set!set notation}.  %Let $S$ be the set of solutions to the system.  Then:
%
%\[S = \left\{\colvec{x_1\\x_2\\x_3\\x_4} = \colvec{1\\1\\ 0\\0 } + \mu_1 \colvec{-1\\1\\1\\0 }  + \mu_2  \colvec{1\\-1\\ 0 \\1 } : \mu_1,\mu_2\in  {\mathbb R} \right\} \]
%
%%%%%%%%%%%%%%%%%
\item %\label{colremove}
Check that the following two matrices are row-equivalent:
\[
\begin{amatrix}{3}
1 & 4 & 7 & 10 \\
2 & 9 & 6 & 0 \\
\end{amatrix}
\mbox{ and }
\begin{amatrix}{3}
0 & -1 & 8 & 20 \\
4 & 18 & 12 & 0 \\
\end{amatrix}\, .
\]
Now remove the third column from each matrix, and show that the resulting two matrices (shown below) are row-equivalent:
\[
\begin{amatrix}{2}
1 & 4 & 10 \\
2 & 9 & 0 \\
\end{amatrix}
\mbox{ and }
\begin{amatrix}{2}
0 & -1 & 20 \\
4 & 18 & 0 \\
\end{amatrix}\, .
\]
Now remove the fourth column from each of the original two matrices, and show that the resulting two matrices, viewed as augmented matrices (shown below) are row-equivalent:
\[
\begin{amatrix}{2}
1 & 4 & 7 \\
2 & 9 & 6 \\
\end{amatrix}
\mbox{ and }
\begin{amatrix}{2}
0 & -1 & 8 \\
4 & 18 & 12 \\
\end{amatrix}\, .
\]
Explain why row-equivalence is never affected by removing columns.

%%%%%%%%%%%%%%%%%%%%%%%%%%%%%%%%%

\item Check that the system of equations corresponding to the augmented matrix
\[\begin{amatrix}{2}
1 & 4 & 10 \\
3 & 13 & 9 \\
4 & 17 & 20 \\
\end{amatrix}\]
has no solutions. If you remove one of the rows of this matrix, does the new matrix have any solutions? In general, can row equivalence be affected by removing rows? Explain why or why not.

%%%%%%%%%%%%%%%%%%%%%%%%%%%%%%%%%

\item \label{Ihavenoslons} 
Explain why the linear system has no solutions:
\[\begin{amatrix}{3}
1 & 0 & 3 & 1 \\
0 & 1 & 2 & 4 \\
0 & 0 & 0 & 6 \\
\end{amatrix}\]

For which values of $k$ does the system below have a solution?

    \begin{equation*}
      \begin{array}{rrrrrrr}
            x   &  -& 3y   & & &=  & 6  \\
            x   &  & &+& 3 z   &=  & -\ 3 \\
            2x  &  +& ky &  +&\!\! (3-k)z   &=  &1  
      \end{array}
    \end{equation*}

\Videoscriptlink{elementary_row_operations_hint.mp4}{Hint}{video_elementary_row_operations_hint}

%%%%%%%%%%%%%%%%%%%%%%%

\item Show that the RREF of a matrix is unique. (Hint: Consider what happens if the same augmented matrix had two different RREFs.
Try to see what happens if you removed columns from these two RREF augmented matrices.)

%%%%%%%%%%%%%%%%%%%%%%%%%%%%%%%%%

\item %(Gaussian Elimination) 
Another method for solving linear systems is to use row operations to bring the augmented matrix to Row Echelon Form\index{Row echelon form} (REF as opposed to RREF).  In REF, the pivots are not necessarily set to one, and we only require that all entries left of the pivots are zero, not necessarily entries above a pivot.  Provide a counterexample to show that row echelon form is not unique.

Once a system is in row echelon form, it can be solved by ``back substitution.''  Write the following row echelon matrix as a system of equations, then solve the system using back-substitution.
\[\begin{amatrix}{3}
2 & 3 & 1 & 6 \\
0 & 1 & 1 & 2 \\
0 & 0 & 3 & 3 \\
\end{amatrix}\]

%%%%%%%%%%%%%%%%%%%%%%%%%%%%%%%%%%%
\item Show that this pair of augmented matrices are row equivalent, assuming $ad-bc \neq 0$:
\label{inverserowops}
\[
\begin{amatrix}{2} 
a & b & e  \\[2mm] 
c & d & f  \\ 
\end{amatrix}\sim
\begin{amatrix}{2} 
1 & 0 & \frac{de-bf}{ad-bc}  \\[2mm]\ 
0 & 1 & \frac{af-ce}{ad-bc}  
\end{amatrix}
\]

\phantomnewpage

% lets Remove this -David
\item Consider the augmented matrix: \[ \begin{amatrix}{2} 
2 & \!-1 & 3  \\ 
\!-6 & 3 & 1  
\end{amatrix}\, . \]

Give a \emph{geometric} reason why the associated system of equations has no solution. (Hint, plot the three vectors
given by the columns of this augmented matrix in the plane.) Given a general augmented matrix \[
\begin{amatrix}{2} 
a & b & e  \\ 
c & d & f  \\ 
\end{amatrix}\, ,\] can you find a condition on the numbers $a,b,c$ and $d$ that corresponds to  the geometric condition you found?


\item \label{ge5} 
A relation $\sim$ on a set of objects $U$ is an \emph{equivalence relation} if the following three properties are satisfied:
\begin{itemize}
\item Reflexive:  For any $x\in U$, we have $x\sim x$.
\item Symmetric:  For any $x,y \in U$, if $x\sim y$ then $y\sim x$.
\item Transitive: For any $x,y$ and $z \in U$, if $x\sim y$ and $y\sim z$ then $x\sim z$.
\end{itemize}
Show that row equivalence of matrices is an example of an {equivalence relation}.  

(For a  discussion of equivalence relations, see \href{\webworkurl Homework0-Background/4/}{Homework 0, Problem 4})


\Videoscriptlink{gaussian_elimination_hints.mp4}{Hint}{script_gaussian_elimination_hints}

\item Equivalence of augmented matrices does not come from equality of their solution sets. Rather, we define two matrices to be equivalent if one can be obtained from the other by elementary row operations. 

Find a pair of augmented matrices that are not row equivalent but do have the same solution set.
%For example, the two augmented  matrices 
%\[
%\begin{amatrix}{4} 
%1 & 1 & 0 & 0 &4\\ 
%0 & 0 & 1 & 0 & 3\\ 
%0 & 0 & 0 & 1 & 2\\ 
%0 & 0 & 0 & 0 &1\\
%\end{amatrix}
%~,~~
%\begin{amatrix}{4} 
%1 & 2 & 0 & 0 &4\\ 
%0 & 0 & 1 & 0 & 3\\ 
%0 & 0 & 0 & 1 & 2\\ 
%0 & 0 & 0 & 0 & 1\\
%\end{amatrix}
%\]
%have identical solutions sets; since the last equation is $0=1$ in both, the solution set is empty in both. 
%However, the first can not be obtained from the other by elementary row operations on the second.. 

\end{enumerate}

%what we need here is "solve the following system of linear equations, 
%maybe a word problem.



