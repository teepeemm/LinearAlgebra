%\subsection*{\gaussElimTitle: Augmented Matrix Notation}
\subsection*{Augmented Matrix Notation}

%%%Insert this to get the typewriter font so it looks like a real movie script
{\ttfamily
\fontdimen2\font=0.4em
\fontdimen3\font=0.2em
\fontdimen4\font=0.1em
\fontdimen7\font=0.1em
\hyphenchar\font=`\-


%%%%put a hypertarget around the opening bit of text
\hypertarget{script_gaussian_elimination_more}{Why is the augmented  matrix} 
$$ \left( \begin{array}{cc | c}
1 & 1 & 27 \\
2 & -1 & 0  
\end{array} \right)\, ,
$$
equivalent to the system of equations
\begin{eqnarray*}
 x+y &=& 27\\
 2x - y &=& 0\, ?
\end{eqnarray*}
Well the augmented matrix is just a new notation for the matrix equation
\begin{equation*}
    \begin{pmatrix}
      1             &1  \\
      2             &-1
    \end{pmatrix}
  \colvec{x \\ y}
  =
  \colvec{27 \\ 0}
\end{equation*}
and if you review your matrix multiplication remember that 
\begin{equation*}
    \begin{pmatrix}
      1             &1  \\
      2             &-1
    \end{pmatrix}
  \colvec{x \\ y}
  =
  \colvec{x+ y \\ 2x - y}
\end{equation*}
This means that

\begin{equation*}
  \colvec{x+ y \\ 2x - y}
  =
  \colvec{27 \\ 0}\, ,
\end{equation*}
which is our original equation.

%%%%don't forget to close the bracket so the stuff after your file doesn't look like a movie!
}

