
\subsection*{Elementary Matrices}

%%%Insert this to get the typewriter font so it looks like a real movie script
{\ttfamily
\fontdimen2\font=0.4em
\fontdimen3\font=0.2em
\fontdimen4\font=0.1em
\fontdimen7\font=0.1em
\hyphenchar\font=`\-


\hypertarget{scripts_elementary_matrices_explanation}{This video will explain some} 
of the ideas behind elementary matrices. First think back to linear systems, for example
$n$ equations in $n$ unknowns:
\[
\left\{
\begin{array}{ccc}
a^1_1 x^1 + a^1_2 x^2+\cdots +a^1_n x^n &=&v^1\\[1mm]
a^2_1 x^1 + a^2_2 x^2+\cdots +a^2_n x^n &=&v^2\\[1mm]
\vdots &&\\[2mm]
a^n_1 x^1 + a^n_2 x^2+\cdots +a^n_n x^n &=&v^n\, .
\end{array}\right .
\]
We know it is helpful to store the above information with matrices and vectors
\[
M:=\begin{pmatrix}
a^1_1&a^1_2&\cdots& a^1_n\\
a^2_1&a^2_2&\cdots& a^2_n\\
\vdots&\vdots&&\vdots\\
a^n_1&a^n_2&\cdots& a^n_n\\
\end{pmatrix}\, ,\qquad
X:=\begin{pmatrix}x^1\\x^2\\\vdots\\ x^n\end{pmatrix}\, ,\qquad
V:=\begin{pmatrix}v^1\\v^2\\\vdots\\v^n\end{pmatrix}\, .
\]
Here we will focus on the case the $M$ is square because we are interested in its inverse $M^{-1}$ (if it exists) and its determinant (whose job it will be to determine the existence of $M^{-1}$).

We know at least three ways of handling this linear system problem:
\begin{enumerate}
\item As an augmented matrix
\[
\left(
\begin{array}{c|c}
M & V
\end{array}
\right)\, .
\]
Here our plan would be to perform row operations until the system looks like
\[
\left(
\begin{array}{c|c}
I & M^{-1}V
\end{array}
\right)\, ,
\]
(assuming that $M^{-1}$ exists).
\item As a matrix equation
\[
MX=V\, ,
\]
which we would solve by finding $M^{-1}$ (again, if it exists), so that
\[
X=M^{-1}V\, .
\]
\item As a linear transformation
\[L:{\mathbb R}^n\longrightarrow {\mathbb R}^n\]
via
\[
{\mathbb R}^n\ni X \longmapsto MX \in {\mathbb R}^n\, .
\]
In this case we have to study the equation $L(X)=V$ because $V\in {\mathbb R}^n$.
\end{enumerate}
Lets focus on the first two methods. In particular we want to think about how the augmented matrix  method can give information about finding $M^{-1}$. In particular, how it can be used for handling determinants. 

The main idea is that the row operations changed the augmented matrices, but we also know how to change a matrix $M$ by multiplying it by some other matrix $E$, so that $M\to EM$.
In particular can we find ``elementary matrices'' the perform row operations?

Once we find these elementary matrices is {\itshape very important} to  ask how they affect the determinant, but you can think about that for your own self  right now. 

Lets tabulate our names for the matrices that perform the various row operations:
\begin{center}\tagpdfsetup{table/header-columns={},table/header-rows={1}}
\begin{tabular}{cc}
Row operation & Elementary Matrix\smallskip\\\hline\\
$R_i\leftrightarrow R_j$ & $E^i_j$\\\smallskip
$R_i\to \lambda R_i$ & $R^i(\lambda)$\\\smallskip
$R_i\to R_i+\lambda R_j$ & $S^i_j(\lambda)$
\end{tabular}
\end{center}

To finish off the video, here is how all these elementary matrices work for a $2\times 2$ example. Lets take 
\[
M=\begin{pmatrix}a&b\\c&d\end{pmatrix}\, .
\]
A good thing to think about is what happens to $\det M = ad-bc$ under the operations below.
\begin{itemize}
\item Row swap:
\[
E^1_2=\begin{pmatrix}0&1\\1&0\end{pmatrix}\, ,\qquad
E^1_2 M = \begin{pmatrix}0&1\\1&0\end{pmatrix}\begin{pmatrix}a&b\\c&d\end{pmatrix}
=\begin{pmatrix}c&d\\a&b\end{pmatrix}\, .
\]
\item Scalar multiplying:
\[
R^1(\lambda)=\begin{pmatrix}\lambda&0\\0&1\end{pmatrix}\, ,\qquad
E^1_2 M = \begin{pmatrix}\lambda&0\\0&1\end{pmatrix}\begin{pmatrix}a&b\\c&d\end{pmatrix}
=\begin{pmatrix}\lambda a&\lambda b\\c&d\end{pmatrix}\, .
\]
\item Row sum:
\[
S^1_2(\lambda)=\begin{pmatrix}1&\lambda\\0&1\end{pmatrix}\, ,\quad
S^1_2(\lambda) M = \begin{pmatrix}1&\lambda\\0&1\end{pmatrix}\begin{pmatrix}a&b\\c&d\end{pmatrix}
=\begin{pmatrix}a+\lambda c&b+\lambda d\\c&d\end{pmatrix}\, .
\]
\end{itemize}


} % Closing bracket for font

%\newpage
