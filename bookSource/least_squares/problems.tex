
\begin{enumerate}
\item \label{kernalsolspaceprob} Let $L:U\rightarrow V$ be a linear transformation.  Suppose $v\in L(U)$ and you have found a vector $u_{\rm ps}$ that obeys $L(u_{\rm ps})=v$.

Explain why you need to compute $\ker L$ to describe the solution 
set %space 
of the linear system $L(u)=v$.

\Videoscriptlink{least_squares_hint.mp4}{Hint}{scripts_least_squares_hint}
\phantomnewpage

\item \label{problem:ls_prob2} Suppose that $M$ is an $m\times n$ matrix with trivial kernel.  Show that for any vectors $u$ and $v$ in $\Re^m$:
\begin{itemize}
\item $u^TM^TMv = v^TM^TMu$.
\item $v^TM^TMv \geq 0$. In case you are concerned (you don't need to be) and for future reference, the notation $v \geq 0$ means each component $v^i \geq 0$.
\item If $v^TM^TMv=0$, then $v=0$.
\end{itemize}
(Hint: Think about the dot product in $\Re^n$.)

\Videoscriptlink{least_squares_hint2.mp4}{Hint}{scripts_least_squares_hint2}


\phantomnewpage

\item Rewrite the Gram-Schmidt algorithm in terms of projection matrices. 

\item Show that if $v_1,\dots, v_k$ are linearly independent that  the matrix 
$M=(v_1 \cdots v_k)$ is not necessarily invertible but  the matrix $M^TM$ is invertible. 

\item Write out the singular value decomposition of a $3\times 1$, a $3\times 2$, and a $3\times 3$ symmetric matrix. 
Choose matrices that have no vanishing components but still make your computations simple. Explain your particular choice of matrices.

\item Find the best polynomial approximation to a solution to the differential equation $\frac{d}{dx} f=x+x^2$ by considering the derivative to have domain and codomain $\spa\left\{ 1,x,x^2  \right\}$. \\
(Hint: Begin by defining bases for the domain and codomain.)
%
%{\bfseries Instructor's Response:}\\
%In the basis $B:=( 1,x,x^2)$
%\[  \frac{d}{dx} \bv a\\b \\c \ev_B= \frac{d}{dx} a+bx+cx^2=b+2cx+0x^2 =\bv b\\2c\\0\ev_B \]
%\[=  
%\left[    
%\baa 
%0&1&0 \\
%0&0&2\\
%0&0&0
%\eaa
%\bv a\\b\\c \ev
%\right]_B.
%\]
%
%The solutions to $ \frac{d}{dx} f=x+x^2$ are then encoded by the solutions to 
%\[\baa 
%0&1&0 \\
%0&0&2\\
%0&0&0
%\eaa
%\bv a\\b\\c \ev
%=\bv 0\\1\\1 \ev\]
%This clearly has no solutions, and so we multiply both sides of the equation by 
%$M^T:= 
%\baa 
%0&0&0 \\
%1&0&0\\
%0&2&0
%\eaa
%$ to obtain 
%\[
%\baa 
%0&0&0 \\
%1&0&0\\
%0&2&0
%\eaa
%\baa 
%0&1&0 \\
%0&0&2\\
%0&0&0
%\eaa
%\bv a\\b\\c \ev
%=\baa 
%0&0&0 \\
%1&0&0\\
%0&2&0
%\eaa
%\bv 0\\1\\1 \ev\]
%\[
%\LR\baa 
%0&0&0 \\
%0&1&0\\
%0&0&4
%\eaa\bv a\\b\\c \ev
%= \bv 0\\0\\2\ev 
%\]
%with solution set $ \left\{   \bv 0\\0\\ \frac12 \ev + c\bv 1\\0\\0 \ev~|~c\in \R  \right\} $ encoding the approximate solution set 
%\[\left\{ \frac12x^2 +c ~|~c\in \R \right\}\] 
%to the differential equation $\frac{d}{dx}f=x+x^2$.



\end{enumerate}

