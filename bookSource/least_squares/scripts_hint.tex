
\subsection*{\leastSquaresTitle: Hint for Review Problem~\ref{kernalsolspaceprob}}

%%%Insert this to get the typewriter font so it looks like a real movie script
{\ttfamily
\fontdimen2\font=0.4em
\fontdimen3\font=0.2em
\fontdimen4\font=0.1em
\fontdimen7\font=0.1em
\hyphenchar\font=`\-


\hypertarget{scripts_least_squares_hint}{Lets work through this} problem. 
 Let $L:U\rightarrow V$ be a linear transformation.  Suppose $v\in L(U)$ and you have found a vector $u_{\rm ps}$ that obeys $L(u_{\rm ps})=v$.

Explain why you need to compute $\ker L$ to describe the solution space of the linear system $L(u)=v$.

Remember the property of linearity that comes along with any linear transformation: $L(ax+by) = aL(x) + bL(y)$ for scalars $a$ and~$b$. This allows us to break apart and recombine terms inside the transformation.

Now suppose we have a solution $x$ where $L(x)=v$. If we have an vector $y \in \ker{L}$ then we know $L(y)=0$. If we add the equations together $L(x)+  L(y) = L(x+ y)=v +0$ we get another solution for free. Now we have two solutions, is that all?



} % Closing braket for font

%\newpage
