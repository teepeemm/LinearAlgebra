%%%%%%%%%%%%%%%%%% Latex %%%%%%%%%%%%%%%%%%%%%
\documentclass[12pt]{article}

\usepackage{amsmath}
\usepackage{shadow}
%\usepackage{bbold}
\usepackage{amssymb}
\usepackage{slashed}
%\usepackage{graphics}
\usepackage{graphicx}
%\usepackage{pstricks,pst-node,pst-tree}


%Sets
\def\A{{\mathbb{A}}}
\def\C{{\mathbb{C}}}
\def\F{{\cal F}}
\def\H{{\mathbb{H}}}
\def\O{{\cal O}}
\def\N{{\mathbb{N}}}
\def\Q{{\mathbb{Q}}}
\def\R{{\mathbb{R}}}
\def\Real{\mathbb R}
%\def\R{{\cal R}}
\def\Z{{\mathbb{Z}}}

%greek
\def\a{\alpha}
\def\b{\beta}
\def\de{\delta} 
\def\De{\Delta}
\def\ga{\gamma} 
\def\g{\gamma}
\def\si{\sigma} 
\def\Si{{\Sigma}}
\def\e{\epsilon}
\def\eps{\epsilon}
\def\La{\Lambda}
\def\r{\rho}
\def\Th{\Theta}
\def\th{\theta} 
\def\k{\kappa}
\def\l{\lambda}
\def\m{\mu}
\def\n{\nu}
\def\s{\sigma}
\def\t{\tau}
\def\z{\zeta}

%equation 
\def\be{\begin{equation}}
\def\ee{\end{equation}}
\def\bea{\begin{eqnarray}}
\def\eea{\end{eqnarray}} 
\def\nn{\nonumber}

%matricies
\def\ba{\left(\begin{array}{cc}}
\def\ea{\end{array}\right) }
\def\baa{\left(\begin{array}{ccc}}
\def\eaa{\end{array}\right) }
\def\baaa{\left(\begin{array}{cccc}}
\def\eaaa{\end{array}\right) }

%vectors
\def\v{\vec}
\def\bv{\left(\begin{array}{c}}
\def\ev{\end{array}\right) }


%quantum operators
%\def\d{{\bf d}}
\def\D{{\bf {\cal D}}}
\def\Del{\bm \delta}
\def\Delb{\bar{\bm \delta}}
\def\DIV{{\bf div}}
\def\G{{\bf g}}
\def\GRAD{{\bf grad}}
%\def\N{{\bf N}}
\def\ord{{\bf ord}}
\def\Pa{\bm \partial}
\def\Partial{\bm \partial}
\def\Pab{\bar{\bm \partial}}
\def\TR{{\bf tr}}
\def\dsl{\slashed{\partial}} 
\def\psl{\slashed{p}} 

%misc
\def\la{\langle}
\def\ra{\rangle}
\def\scirc{\!{\scriptstyle \circ}}
\def\pa{\partial}
\def\bra{\langle} \def\ket{\rangle}
\def\ie{{i.e., }}
\def\eg{{e.g.\ }}
\def\cl{\centerline}
\def\noi{\noindent}
\def\f{\frac}
\def\LR{\Leftrightarrow}
\def\d{\frac{d}{dx}}
\def\span{ {\rm span}}
\def\ran{ {\rm ran} \, }

\def\ni{\noindent}

%\newcommand{\comment}[1]{{\bf [#1]}}
%\newcommand{\ul}[1]{\underline{#1}}
%\newcommand{\bm}[1]{\mbox{\boldmath $ #1 $}}

%%%%%%%%%%%%%%%%%%%%%%%%%%%%%%%%%%%

\begin{document}

\thispagestyle{empty}
~
\vspace{-2cm}

\begin{center}
\vspace{-1.5cm}
{\Large{\bf  
Written Exercise part III:\\
Approximate Solutions
}
 }  \\[9mm]
% {\sc \small 22A, Summer 2014, \\
% Prof. Cherney}
\end{center}

Recognize when approximation methods are appropriate. \\

Option for approximating a matrix instead? \\



~
\vspace{-2cm}
\section*{Example}
To measure the strength of the gravitational field at the surface of mars, a mars rover picks up a rock, throws it, and video records the rock flying through the air. A team of volunteers watches the video and estimates the height of the rock in each frame of the video. 
%The frames are numbered $1,2,3,...$ and there values are refered to as t values
\[
\begin{array}{|l|c|c|c|c|c|}
\hline
{\rm Frame~ number}& 1&2&3&4&5\\%[.5mm]
\hline
{\rm Estimated~height}& 2&4&5&3&1\\
\hline
\end{array}
\]
Estimate the acceleration due to gravity at the surface of mars based on this data.
Describe ways that more data can be obtained to yield a better estimation.\\


\noindent {\bf Response:}\\
If the acceleration of the rock due to gravity is constant $a$ then the function $h:[0,5]\to \R$ that gives the height of the rock (in the units were used to describe the height of the rock) as a function of time (in units of time equal to the time between video frames) satisfies 
\[ \frac{d^2}{dt^2} h(t)=a \implies h(t)=\frac12 a t^2+bt+c {\rm ~for~some ~}b,c\in \R.\]
If $h$ is to fit the data exactly then $(a,b,c)$ should be a solution to the system of equations 
\bea \nn
\begin{array}{cc}
\left.
\begin{array}{cc}
\frac12a 1^2+b1+c &= 2 \\
\frac12a 2^2+b2+c &= 4 \\
\frac12a 3^2+b3+c &= 5 \\
\frac12a 4^2+b4+c &= 3 \\
\frac12a 5^2+b5+c &= 1 
\end{array}
\right\}\LR
& 
\baa
\frac12 &1&1\\
2&2&1\\
\frac92 &3&1\\
8&4&1\\
\frac{25}{2} &5&1
\eaa
\bv a\\b\\c \ev = \bv 2\\4\\5\\3\\1\ev
\end{array}.
\eea
This system has no solution, as shown by row reduction of the associated augmented matrix; 
\begin{gather*}
\left( \begin{array}{rrr|c}
\frac12 &1&1&2\\
2&2&1&4\\
\frac92 &3&1&5\\
8&4&1&3\\
\frac{25}{2} &5&1&1
\end{array} \right)
\sim 
\left( \begin{array}{rrr|r}
1 &2&2&4\\
0      &-2&-3&-4\\
0&-6&-8&-13\\
0&4&1&3\\
0&-20&-24&-49
\end{array} \right)
%
\sim 
\left( \begin{array}{rrr|r}
1 &0&-1&0\\
0      &2&3&4\\
0&0&1&-1\\
0&0&-5&-5\\
0&0&6&-9
\end{array} \right)
\\
\sim 
\left( \begin{array}{rrr|r}
1 &0&0&-1\\
0      &2&0&7\\
0&0&1&-1\\
0&0&0&-10\\
0&0&0&-3
\end{array} \right)
\sim 
\left( \begin{array}{rrr|r}
1 &0&0&-1\\
0      &2&0&7\\
0&0&1&-1\\
0&0&0&1\\
0&0&0&0
\end{array} \right).
\end{gather*}
The system encoded by the augmented matrix $(M|V)$ does not have a solution if $V$ can not be expressed as a linear combination of the columns of $M$ since this means that $V$ is not in the range of $M$. 
Out of all the systems of equations $(M|U)$ that do have solutions the system with $U$ closest to $V$ is the system $(M|V_r)$ where $V_r$ is the projection of 
$V$ onto ${\rm ran~} M$. The matrix that projects a vector onto ${\rm ran~} M$ is 
$M(M^TM)^-1 M $
\begin{gather*}= 
\left( \begin{array}{rrr}
1 &0&0\\
0      &2&0\\
0&0&1\\
0&0&0\\
0&0&0
\end{array} \right)
\left(
\left( \begin{array}{rrrrr}
1 &0&0&0&0\\
0      &2&0&0&0\\
0&0&1&0&0\\
\end{array} \right)
\left( \begin{array}{rrr}
1 &0&0\\
0      &2&0\\
0&0&1\\
0&0&0\\
0&0&0
\end{array} \right)
\right)^{-1}
\left( \begin{array}{rrrrr}
1 &0&0&0&0\\
0      &2&0&0&0\\
0&0&1&0&0\\
\end{array} \right)
\\= 
\left( \begin{array}{rrr}
1 &0&0\\
0      &2&0\\
0&0&1\\
0&0&0\\
0&0&0
\end{array} \right)
\left( \begin{array}{rrr}
1 &0&0\\
0      &4&0\\
0&0&1\\
\end{array} \right)^{-1}
\left( \begin{array}{rrrrr}
1 &0&0&0&0\\
0      &2&0&0&0\\
0&0&1&0&0\\
\end{array} \right)
\\= 
\left( \begin{array}{rrr}
1 &0&0\\
0      &2&0\\
0&0&1\\
0&0&0\\
0&0&0
\end{array} \right)
\left( \begin{array}{rrr}
1 &0&0\\
0      &\frac14&0\\
0&0&1\\
\end{array} \right)
\left( \begin{array}{rrrrr}
1 &0&0&0&0\\
0      &2&0&0&0\\
0&0&1&0&0\\
\end{array} \right)
\\= 
\left( \begin{array}{rrr}
1 &0&0\\
0      &\frac12&0\\
0&0&1\\
0&0&0\\
0&0&0
\end{array} \right)
\left( \begin{array}{rrrrr}
1 &0&0&0&0\\
0      &2&0&0&0\\
0&0&1&0&0\\
\end{array} \right)
\\= 
\left( \begin{array}{rrrrr}
1 &0&0&0&0\\
0      &1&0&0&0\\
0&0&1&0&0\\
0&0&0&0&0\\
0&0&0&0&0
\end{array} \right).
\end{gather*}
Applying this matrix to $V$ gives 
\[
V_r:=\left( \begin{array}{rrrrr}
1 &0&0&0&0\\
0      &1&0&0&0\\
0&0&1&0&0\\
0&0&0&0&0\\
0&0&0&0&0
\end{array} \right) 
\bv -1\\7\\-1\\1\\0 \ev 
=
\bv -1\\7\\-1\\0\\0 \ev.
\]

The equation $Mx=V_r$ then has augmented matrix 

\[
\left( \begin{array}{rrr|r}
1 &0&0&-1\\
0      &2&0&7\\
0&0&1&-1\\
0&0&0&0\\
0&0&0&0
\end{array} \right)
\]
and unique solution 
\[\bv -1\\ \frac72 \\-1 \ev. \]
The quadratic function that best approximates the data set is then 
\begin{gather*} h:[0,5] \to \R\\
h(t)=\frac12 (-1) t^2 +\frac72 t +(-1).\end{gather*}

In particular, the data indicate that acceleration at the surface of mars $a$ is approximately $-1$ (in whatever units were used to described the height in the data set per the the time between video frames squared.) Mode data would yield a better approximation for this value. That data may come in the form of more video frames per second, data for another rock throw, throwing a rock harder so that its hang time is longer, or setting up a different experiment such as spherical balls rolling down inclined planes as Galileo used to measure earth's gravity. 




\end{document}