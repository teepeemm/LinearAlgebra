
\subsection{\solutionSetsTitle: Hint}

{\ttfamily
\fontdimen2\font=0.4em
\fontdimen3\font=0.2em
\fontdimen4\font=0.1em
\fontdimen7\font=0.1em
\hyphenchar\font=`\-

\hypertarget{scripts_solution_sets_for_systems_of_linear_equations_hint}{For the first part of} \hyperref[matvect]{this problem}, the key is to consider the vector as a $n \times 1$ matrix. For the second part, all you need to show is that
\[
M(\alpha \cdot X + \beta \cdot Y) = \alpha \cdot (MX) + \beta \cdot (MY)
\]
where $\alpha, \beta \in \mathbb{R}$ (or whatever field we are using) and
\[
Y = \begin{pmatrix}y^1 \\ y^2 \\ \vdots \\ y^k\end{pmatrix}.
\]
Note that this will be somewhat tedious, and many people use summation notation or Einstein's summation convention with the added notation of $M_j$ denoting the $j$-th row of the matrix. For example, for any $j$ we have
\[
(MX)_j = \sum_{i=1}^k a_i^j x^i = a_i^j x^i.
\]

You can see \hyperlink{solution_sets_for_systems_of_linear_equations_concrete_example}{a concrete example} after the definition of the linearity property.

} % Closing brace for the font

\newpage
