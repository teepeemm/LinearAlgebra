\chapter{\solutionSetsTitle}

Algebra problems can have multiple solutions. For example $x(x-1)=0$ has  two solutions: $0$ and $1$. By contrast, equations of the form $Ax=b$ with $A$ a linear operator have have the following property.\\



If $A$ is a linear operator and $b$ is a known then $Ax=b$ has either
\begin{enumerate}
\item One solution
\item  No solutions
\item Infinitely many solutions
\end{enumerate}


\section{The Geometry of Solution Sets: Hyperplanes}
Consider the following algebra problems and their solutions

\begin{enumerate}
\item $6x=12$, one solution: $2$
\item $0x=12$, no solution
\item $0x=0$, one solution for each number: $x$
\end{enumerate}
In each case the linear operator is a $1\times 1$ matrix. In the first case, the linear operator is invertible. 
In the other two cases it is not. 
In the first case, the solution set is a point on the number line, in the third case the solution set is the whole number line.

Lets examine similar situations with larger matrices.
\begin{enumerate}
\item
$\begin{pmatrix}
6	&0 	\\
0 	&2 	
\end{pmatrix} 
\begin{pmatrix}
 x \\ 
y 
\end{pmatrix} 
=
\begin{pmatrix}
12 \\ 
6
\end{pmatrix}$, one solution: 
$\begin{pmatrix}
2 \\ 
3
\end{pmatrix}$
%\\linear operator is invertible

\item 
$\begin{pmatrix}
1	&3 	\\
0 	&0 	
\end{pmatrix} 
\begin{pmatrix}
 x \\ 
y 
\end{pmatrix} 
=
\begin{pmatrix}
4 \\ 
1 
\end{pmatrix}$, no solutions
%not in the range of the linear operator

\item 
$\begin{pmatrix}
1	&3 	\\
0 	&0 	
\end{pmatrix} 
\begin{pmatrix}
 x \\ 
y 
\end{pmatrix} 
=
\begin{pmatrix}
4 \\ 
0
\end{pmatrix} $, one solution for each number $y$: 
$\begin{pmatrix}
4-3y \\ 
y
\end{pmatrix} $

\item 
$\begin{pmatrix}
0	&0 	\\
0 	&0 	
\end{pmatrix} 
\begin{pmatrix}
 x \\ 
y 
\end{pmatrix} 
=
\begin{pmatrix}
0 \\ 
0
\end{pmatrix} $, one solution for each pair of numbers $x,y$:
$\begin{pmatrix}
x\\ 
y
\end{pmatrix} $
\end{enumerate}
Again, in the first case the linear operator is invertible while in the other cases it is not. When the operator is not invertible the solution set can be empty, a line in the plane or the plane itself.


For a system of equations with $r$ equations and $k$ veriables, one can have a number of different outcomes.  For example, consider the case of $r$ equations in three variables.  Each of these equations is the equation of a plane in three-dimensional space.  To find solutions to the system of equations, we look for the common intersection of the planes (if an intersection exists).  Here we have five different possibilities:

\begin{enumerate}
\item \textbf{Unique Solution.}  The planes have a unique point of intersection.

\item \textbf{No solutions.}  Some of the equations are contradictory, so no solutions exist.

\item \textbf{Line.}  The planes intersect in a common line; any point on that line then gives a solution to the system of equations.

\item \textbf{Plane.}  Perhaps you only had one equation to begin with, or else all of the equations coincide geometrically.  In this case, you have a plane of solutions, with two free parameters.

\videoscriptlink{solution_sets_for_systems_of_linear_equations_planes.mp4}{Planes}{solution_sets_for_systems_of_linear_equations_planes}

\item \textbf{All of $\mathbb{R}^3$.}  If you start with no information, then any point in $\mathbb{R}^3$ is a solution.  There are three free parameters.
\end{enumerate}

In general, for systems of equations with $k$ unknowns, there are $k+2$ possible outcomes, corresponding to the possible numbers (i.e. $0,1,2,\dots,k$) of free parameters in the solutions set plus the possibility of no solutions.  These types of ``solution sets''\index{Solution set} are ``hyperplanes''\index{Hyperplane}, generalizations of planes the behave like planes in $\mathbb{R}^3$ in many ways.

\videoscriptlink{solution_sets_for_systems_of_linear_equations_overview.mp4}{Pictures and Explanation}{solution_sets_for_systems_of_linear_equations_overview}

\vspace{3mm}

\reading{4}{1}
%\begin{center}\href{\webworkurl ReadingHomework4/1/}{Reading homework: problem 4.1}\end{center}



\section{Particular Solution $+$ Homogeneous solutions }

In the \hyperlink{standard approach}{standard approach}, variables corresponding to columns that do not contain a pivot (after going to reduced row echelon form) are \emph{free}.  
We called them non-pivot variables. 
They index elements of the solutions set by acting as coefficients of vectors.
%In this way the number of non-pivot columns determines (in part) the size of the solution set.  
%We can denote them with dummy variables $\mu_1, \mu_2, \ldots$. 

\begin{example} (Non-pivot columns determine terms of the solutions)
\[\begin{pmatrix}
1 &  0 & 1 & -1 \\ 
 0 & 1 & -1& 1  \\
 0 &0   & 0  & 0 \\
\end{pmatrix}
\colvec{x_1\\x_2\\x_3\\x_4} 
=
\colvec{1\\1\\0} 
\Leftrightarrow
\left\{
\begin{array}{lcr}
	1x_1 +0x_2+ 1x_3 - 1x_4 & = 1 \\
	0x_1 +1x_2 - 1x_3 + 1x_4 & = 1 \\
	0x_1 +0x_2 + 0x_3 + 0x_4 & = 0 
\end{array}
     \right.
\]
Following the standard approach, express the pivot variables in terms of the non-pivot variables and add ``freebee equations". Here $x_3$ and $x_4$ are non-pivot variables.  
\begin{eqnarray*}
\left.
\begin{array}{rcl}
	x_1 & = &1 -x_3+x_4 \\
	x_2 & = &1 +x_3-x_4 \\
	x_3 & = &\phantom{1+~\,}x_3\\
	x_4 & =&\phantom{1+x_3+~}x_4         
\end{array}
     \right\}
     \Leftrightarrow
\colvec{x_1\\x_2\\x_3\\x_4} 
= \colvec{1\\1\\0\\0} + x_3\colvec{-1\\1\\1\\0} + x_4\colvec{1\\-1\\0\\1}
\end{eqnarray*}
The preferred way to write a solution set is with set notation\index{Solution set!set notation}.  \[S = \left\{\colvec{x_1\\x_2\\x_3\\x_4} = \colvec{1\\1\\ 0\\0 } + \mu_1 \colvec{-1\\1\\1\\0 }  + \mu_2  \colvec{1\\-1\\ 0 \\1 } : \mu_1,\mu_2\in  {\mathbb R} \right\} \]
Notice that the first two components of the second two terms come from the non-pivot columns
Another way to write the solution set is
\[S= \left\{  X_0 + \mu_1 Y_1 + \mu_2 Y_2   : \mu_1,\mu_2 \in  {\mathbb R}   \right\} \]
where 
\[X_0= \colvec{1\\1\\0 \\0 }, Y_1=\colvec{-1\\1\\1\\0 } , Y_2= \colvec{1\\-1\\0 \\1 }
\]
\end{example}
Here $X_0$ is called a particular solution while $Y_1$ and $Y_2$ are called homogeneous solutions. 



\section{Linearity and these parts}
%
%\begin{definition}   A function $f$ is \emph{linear}\index{Linear!function} if 
%for any vectors $X,Y$  in the domain of $f$, and any scalars $\alpha,\beta$ 
%\[f(\alpha X + \beta Y) = \alpha f(X) + \beta f(Y) \,.\]
%\end{definition}

%
%
%\begin{example}
%\hypertarget{solution_sets_for_systems_of_linear_equations_concrete_example}{Consider our example system above with} 
%\[
%M=    \begin{pmatrix}
%      1  & 0  & 1 & -1  \\
%       0  & 1 & -1 & 1  \\
%        0 &0   & 0  & 0    \\
%    \end{pmatrix} \, ,\quad
%X= \colvec{x_1\\x_2\\x_3\\x_4} \mbox{ and } Y=\colvec{y_1\\y_2\\y^3 \\y^4 }\, ,
%\]
%and take for the function of vectors
%\[
%f(X)=MX\, .
%\]
%Now let us check the linearity property for $f$. 
%The property needs to hold for {\itshape any} scalars $\alpha$ and $\beta$, so for simplicity
%let us concentrate first on the case $\alpha=\beta=1$. This means that we need to
%compare the following two calculations:
%\begin{enumerate}
%\item First add $X+Y$, then compute $f(X+Y)$.
%\item First compute $f(X)$ and $f(Y)$, then compute the sum $f(X)+f(Y)$.
%\end{enumerate}
%The second computation is slightly easier:
%\[
%f(X) = MX 
%    =\colvec{x_1+x_3-x_4\\x_2-x_3+x_4\\0}\mbox{ and }
%f(Y) = MY   
%    =\colvec{y_1+y_3-y_4\\y_2-y_3+y_4\\0}\, ,
%\]
%(using our result above). Adding these gives
%\[
%f(X)+f(Y)=\colvec{x_1+x_3-x_4+y_1+y_3-y_4\\[1mm]x_2-x_3+x_4+y_2-y_3+y_4\\[1mm]0}\, .
%\]
%Next we perform the first computation beginning with:
%\[
%X+Y=\colvec{x_1 + y_1\\x_2+y_2\\ x_3+y_3\\ x_4+y_4}\, ,
%\]
%from which we calculate
%\[
%f(X+Y)=\colvec{x_1+y_2+x_3+y_3-(x_4+y_4)\\[1mm] x_2+y_2-(x_3+y_3)+x_4+y_4\\[1mm]0}\, .
%\]
%Distributing the minus signs and remembering that the order of adding numbers like $x_1,x_2,\ldots$ 
%does not matter, we see that the two computations give exactly the same answer.
%
%Of course, you should complain that we took a special choice of $\alpha$ and $\beta$.
%Actually, to take care of this we only need to check that $f(\alpha X)=\alpha f(X)$.
%It is your job to explain this in  \hyperref[linear]{Review Question~\ref*{linear}}
%\end{example}
%
%\noindent
%Later we will show that matrix multiplication is always linear.  Then we will know that:
With the previous example in mind, lets say that the matrix equation $MX=V$ has  solution set  $\{ X_0 + \mu_1 Y_1 + \mu_2 Y_2):\mu_1,\mu_2 \in {\mathbb R} \}$.
Recall from \hyperlink{{Matrices are linear operators}}{earlier} that matrices are linear.
%\[M(\alpha X + \beta Y) = \alpha MX + \beta MY\]
%
%Then 
%
%the two equations 
Thus 
\[M( X_0 + \mu_1 Y_1 + \mu_2 Y_2)  = MX_0 + \mu_1MY_1 + \mu_2MY_2 =V\]
for \emph{any} $\mu_1, \mu_2 \in \mathbb{R}$. 
Choosing $\mu_1=\mu_2=0$, we obtain 
\[MX_0=V\, .\]  
This is why $X_0$ is an example of a  \emph{particular solution}\index{Particular solution!an example}.

%Given a particular solution to the system, we can then deduce that $\mu_1MY_1 + \mu_2MY_2 = 0$.  
Setting $\mu_1=1, \mu_2=0$, and using the particular solution $MX_0=V$, we obtain 
\[MY_1=0\, .\] 
Likewise, setting $\mu_1=0, \mu_2=1$, we obtain \[MY_2=0\, .\]
Here $Y_1$ and $Y_2$ are examples of what are called  \emph{homogeneous} solutions\index{Homogeneous solution!an example} to the system.
They {\itshape do not} solve the original equation $MX=V$, but instead its associated 
{\itshape homogeneous  equation}\index{homogeneous equation} $M Y =0$.

One of the fundamental lessons of linear algebra: the  solution set to $Ax=b$ with $A$ a linear operator consists of a particular solution plus homogeneous solutions.

\begin{center}
\shabox{
general solution $=$ particular solution $+$ homogeneous solutions.}
\end{center}

\begin{example}
Consider the matrix equation of the previous example. It has  solution set
\[S = \left\{\colvec{x_1\\x_2\\x_3\\x_4} = \colvec{1\\1\\0 \\0 } + \mu_1 \colvec{-1\\1\\1\\0 } + \mu_2 \colvec{1\\-1\\ 0\\1 } \right\} \]
Then $MX_0=V$ says that $\colvec{x_1\\x_2\\x_3\\x_4} = 
\colvec{1\\1\\0 \\ 0}$ solves the original matrix equation, which is certainly true, but this is not the only solution.

$MY_1=0$ says that $\colvec{x_1\\x_2\\x_3\\x_4} = \colvec{-1\\1\\1\\ 0}
$ solves the homogeneous equation.

\vspace{2mm}

$MY_2=0$ says that $\colvec{x_1\\x_2\\x_3\\x_4} = 
\colvec{1\\-1\\0 \\1}$ solves the homogeneous equation.

\vspace{2mm}

\noindent
Notice how adding any multiple of a homogeneous solution to the particular solution yields another particular solution.
\end{example}

%\begin{definition}
%Let $M$ a matrix and $V$ a vector.  Given the linear system $MX=V$, we call $X_0$ a \emph{particular solution}\index{Particular solution} if $MX_0=V$.  We call $Y$ a \emph{homogeneous solution} if $MY=0$.  
%The linear system 
%\[MX=0\] is called the (associated) \emph{homogeneous system}\index{Homogeneous system}.
%\end{definition}
%
%If $X_0$ is a particular solution, then the general solution\index{General solution} to the system is\footnote{The notation \(S=\{X_0+Y : MY=0\}\) is read, ``\(S\) is the set of all \(X_0+Y\) such that \(MY=0,\)'' and means exactly that. Sometimes a pipe \(|\) is used instead of a colon.}:
%
%\[S= \{X_0+Y : MY=0 \} \]

\reading{4}{2}
%\begin{center}\href{\webworkurl ReadingHomework4/2/}{Reading homework: problem 4.2}\end{center}

%\section*{References}
%
%Hefferon, Chapter One, Section I.2
%\\
%Beezer, Chapter SLE, Section TSS
%\\
%Wikipedia, \href{http://en.wikipedia.org/wiki/System_of_linear_equations}{Systems of Linear Equations}


%\section{The size of solution sets vs size of homogeneous solution set}


\section{Review Problems}





\begin{enumerate}
\item \label{det33} Let $M=\begin{pmatrix}
m^1_1 & m^1_2 & m^1_3\\
m^2_1 & m^2_2 & m^2_3\\
m^3_1 & m^3_2 & m^3_3\\
\end{pmatrix}$.  Use row operations to put $M$ into \emph{row echelon form}.  For simplicity, assume that $m_1^1\neq 0 \neq m^1_1m^2_2-m^2_1m^1_2$.

Prove that $M$ is non-singular if and only if:
\[
m^1_1m^2_2m^3_3 
- m^1_1m^2_3m^3_2 
+ m^1_2m^2_3m^3_1 
- m^1_2m^2_1m^3_3 
+ m^1_3m^2_1m^3_2
- m^1_3m^2_2m^3_1
\neq 0
\]

\phantomnewpage

\item 
\begin{enumerate}
\item What does the matrix $E^1_2=\begin{pmatrix}
0 & 1 \\
1 & 0
\end{pmatrix}$ do to $M=\begin{pmatrix}
a & b \\
d & c
\end{pmatrix}$ under left multiplication?  What about right multiplication?
\item Find elementary matrices $R^1(\lambda)$ and $R^2(\lambda)$ that respectively multiply rows $1$ and $2$ of $M$ by $\lambda$ but otherwise leave $M$ the same under left multiplication.
\item Find a matrix $S^1_2(\lambda)$ that adds a multiple $\lambda$ of row $2$ to row $1$ under left multiplication.
\end{enumerate}

\phantomnewpage

\item Let $M$ be a matrix and $S^i_jM$ the same matrix with rows \(i\) and \(j\) switched.  Explain every line of the 
\hyperlink{rowswap}{series of equations} proving that $\det M = -\det (S^i_jM)$.

\phantomnewpage

%\item \label{prob_inversion_number} This problem is a ``hands-on'' look at why \hyperlink{permutation_parity}{the property} describing the parity of permutations is true.
%
%\hypertarget{inversion_number}{The \emph{inversion number}}\index{Permutation!Inversion number} of a permutation $\sigma$ is the number of pairs $i<j$ such that $\sigma(i)>\sigma(j)$; it's the number of ``numbers that appear left of smaller numbers'' in the permutation.  For example, for the permutation $\rho = [4,2,3,1]$, the inversion number is $5$. The number $4$ comes before $2,3,$ and $1$, and $2$ and $3$ both come before $1$.
%
%Given a permutation $\sigma$, we can make a new permutation $\tau_{i,j} \sigma$ by exchanging the $i$th and $j$th entries of $\sigma$.
%
%\begin{enumerate}
%\item What is the inversion number of the permutation \(\mu=[1,2,4,3]\) that exchanges 4 and 3 and leaves everything else alone? Is it an even or an odd permutation?
%
%\item What is the inversion number of the permutation \(\rho=[4,2,3,1]\) that exchanges 1 and 4 and leaves everything else alone? Is it an even or an odd permutation?
%
%\item What is the inversion number of the permutation \(\tau_{1,3} \mu\)? Compare the parity\footnote{The \emph{parity} of an integer refers to whether the integer is even or odd. Here the parity of a permutation $\mu$ refers to the parity of its inversion number.} of \(\mu\) to the parity of \(\tau_{1,3} \mu.\)
%
%\item What is the inversion number of the permutation \(\tau_{2,4} \rho\)? Compare the parity of \(\rho\) to the parity of \(\tau_{2,4} \rho.\)
%
%\item What is the inversion number of the permutation \(\tau_{3,4} \rho\)? Compare the parity of \(\rho\) to the parity of \(\tau_{3,4} \rho.\)
%\end{enumerate}
%
%\videoscriptlink{elementary_matrices_determinant_hint.mp4}{Problem~\ref{prob_inversion_number} hints}{scripts_elementary_matrices_determinants_hint}

\phantomnewpage

%\item \label{problem_permutation} (Extra credit) Here we will examine a (very) small set of the general properties about permutations and their applications. In particular, we will show that one way to compute the sign of a permutation is by finding the \hyperlink{inversion_number}{inversion number} $N$ of $\sigma$ and we have
%\[
%\sgn(\sigma) = (-1)^N.
%\]
%
%For this problem, let $\mu = [1,2,4,3]$.
%
%\begin{enumerate}
%\item Show that every permutation $\sigma$ can be sorted by only taking simple (adjacent) transpositions\index{Permutation!Simple transposition} $s_i$ where $s_i$ interchanges the numbers in position $i$ and $i+1$ of a permutation $\sigma$ (in our other notation $s_i = \tau_{i,i+1}$). For example $s_2 \mu = [1, 4, 2, 3]$, and to sort $\mu$ we have $s_3 \mu = [1, 2, 3, 4]$.
%
%\item \label{prob_part_relations} We can compose simple transpositions together to represent a permutation (note that the sequence of compositions is not unique), and these are associative, we have an identity (the trivial permutation where the list is in order or we do nothing on our list), and we have an inverse since it is clear that $s_i s_i \sigma = \sigma$. Thus permutations of $[n]$ under composition are an example of a \hyperref[groups]{group}. However note that not all simple transpositions commute with each other since
%\begin{align*}
%s_1 s_2 [1, 2, 3] & = s_1 [1, 3, 2] = [3, 1, 2]
%\\ s_2 s_1 [1, 2, 3] & = s_2 [2, 1, 3] = [2, 3, 1]
%\end{align*}
%(you will prove here when simple transpositions commute). When we consider our initial permutation to be the trivial permutation $e = [1, 2, \dotsc, n]$, we do not write it; for example $s_i \equiv s_i e$ and $\mu = s_3 \equiv s_3 e$. This is analogous to not writing 1 when multiplying. Show that $s_i s_i = e$ (in shorthand $s_i^2 = e$), $s_{i+1} s_i s_{i+1} = s_i s_{i+1} s_i$ for all $i$, and $s_i$ and $s_j$ commute for all $|i - j| \geq 2$.
%
%\item Show that every way of expressing $\sigma$ can be obtained from using the relations proved in part~\ref{prob_part_relations}. In other words, show that for any expression $w$ of simple transpositions representing the trivial permutation $e$, using the proved relations.
%
%\emph{Hint: Use induction on $n$. For the induction step, follow the path of the $(n+1)$-th strand by looking at $s_n s_{n-1} \cdots s_k s_{k\pm1} \cdots s_n$ and argue why you can write this as a subexpression for any expression of $e$. Consider using diagrams of these paths to help.}
%
%\item The simple transpositions \hyperlink{action}{acts on} an $n$-dimensional vector space $V$ by $s_i v = E^i_{i+1} v$ (where $E^i_j$ is \hyperlink{elem_matrix_row_swap}{an elementary matrix}) for all vectors $v \in V$. Therefore we can just represent a permutation $\sigma$ as the matrix $M_{\sigma}$\footnote{Often people will just use $\sigma$ for the matrix when the context is clear.}, and we have $\det(M_{s_i}) = \det(E^i_{i+1}) = -1$. Thus prove that $\det(M_{\sigma}) = (-1)^N$ where $N$ is a number of simple transpositions needed to represent $\sigma$ as a permutation. You can assume that $M_{s_i s_j} = M_{s_i} M_{s_j}$ (it is not hard to prove) and that $\det(A B) = \det(A) \det(B)$ \hyperref[detmultiplicative]{from Chapter~\ref*{elementarydeterminantsII}}.
%
%\emph{Hint: You to make sure $\det(M_{\sigma})$ is well-defined since there are infinite ways to represent $\sigma$ as simple transpositions.}
%
%\item Show that $s_{i+1} s_i s_{i+1} = \tau_{i, i+2}$, and so give one way of writing $\tau_{i, j}$ in terms of simple transpositions? Is $\tau_{i,j}$ an even or an odd permutation? What is $\det(M_{\tau_{i,j}})$? What is the inversion number of $\tau_{i,j}$?
%
%\item The minimal number of simple transpositions needed to express $\sigma$ is called the \emph{length}\index{Permutation!Length} of $\sigma$; for example the length of $\mu$ is 1 since $\mu = s_3$. Show that the length of $\sigma$ is equal to the inversion number of $\sigma$.
%
%\emph{Hint: Find an procedure which gives you a new permutation $\sigma^{\prime}$ where $\sigma = s_i \sigma^{\prime}$ for some $i$ and the inversion number for $\sigma^{\prime}$ is 1 less than the inversion number for $\sigma$.}
%
%\item Show that $(-1)^N = \sgn(\sigma) = \det(M_{\sigma})$, where $\sigma$ is a permutation with $N$ inversions. Note that this immediately implies that $\sgn(\sigma \rho) = \sgn(\sigma) \sgn(\rho)$ for any permutations $\sigma$ and $\rho$.
%\end{enumerate}

\item Let $M'$ be the matrix obtained from $M$ by swapping two columns $i$ and $j$. Show that $\det M'=-\det M $.

\item The scalar triple product of three vectors $u,v,w$ from $\Re^3$ is $u\cdot(v\times w)$. Show that this product is the same as the determinant of the matrix whose columns are $u,v,w$ (in that order). What happens to the scalar triple product when the factors are permuted? 

\item Show that if $M$ is a $3\times 3$ matrix whose third row is a sum of multiples of the other rows ($R_3=aR_2+bR_1$) then $\det M=0$. Show that the same is true if one of the columns is a sum of multiples of the others. 

\end{enumerate}

\phantomnewpage


\newpage
